\section{Track and primary vertex reconstruction}
The CMS tracker gets traversed by $\mathcal{O}~1000$ charged particles at each bunch crossing, produced by an average of roughly 34 proton-proton interactions happening ~simultaneously. This makes track reconstructions extremely challenging, and is the reason why a high granularity of the tracker is vital.
Track reconstruction describes the process of taking hits from the pixel and strip detectors, combining them and estimating the momentum and flight direction of the charged particle responsible for producing the hits. It is an extremely computationally heavy process and is based on what is called a combinatorial Kalman filter~\cite{BILLOIR1989390}. A Kalman filter is an algorithm that uses time-dependent observations in order to estimate unknown variables, by proceeding progressively from one measurement to the next, improving the knowledge of the
trajectory with each new measurement. The track reconstruction software in CMS (called the Combinatorial Track Finder (CTF)) constructs its collection of tracks by iteratively looping over the hits and reconstructing tracks, then removing those which are already used as inputs for a previous track. It starts from a seed in the inner most tracker layers, usually two or three hits, and then extrapolates the seed trajectories searching for additional hits to associate to that candidate. It then disregards tracks that fail certain criteria  based on a $\chi^2$ calculation taking both hit and trajectory uncertainties into account, as well as the number of missing hits.
The track reconstruction algorithm is effective over the full tracker coverage range up to $|\eta|<2.5$ and can reconstruct particles with momenta as low as 0.1 GeV or particles which are produced up to 60 \cm from the beam line. In the central region, particles with a momentum of 100 GeV have a \PT-resolution of roughly 2.8 \%, a transverse impact parameter resolution of 10 $\micron$ and a longitudinal impact parameter of 30 $\micron$. 


In order to define the location and uncertainty of every proton-proton interaction in an event, primary-vertex reconstruction in performed. Primary vertices lie within a radius of a few millimeters of the beam axis and are defined as the common origin of groups of tracks.
The reconstruction algorithm takes as input the reconstructed tracks from the previous step which pass certain selection criteria, clusters the tracks that share a common origin and then fit for the position of each vertex. Each track must have at least 2 hits in the pixel layers and no less than 5 hits in the pixel+strip as well as a $\chi^2<20$ from a fit to the particle trajectory to be considered as input for the vertex finder. The primary vertex resolution is around 12 \micron in x and 10 \micron in z for vertices with at least 50 tracks.

Offline, all events are required to have at least one primary vertex reconstructed within a 24 \cm window along the beam axis, with a transverse distance from the nominal interaction region of less than 2 \cm. The reconstructed vertex with the largest value of summed physics object $\PT^2$ is selected as the primary interaction vertex where the hard scattering process occurred.

\section{The Particle Flow Algorithm}


\section{Pile-up removal}

Particles originating from proton-proton interactions not associated with the hardest primary vertex, are denoted pileup events.
These distorts observables of interest from the hard scattering event and must be mitigated through dedicated pileup removal techniques

\subsection{Charged Hadron Subtraction}
As mentioned previously, primary vertices are reconstructed using tracks from charged hadrons. If a primary vertex does not correspond to the hard scattering vertex of the event, the charged hadrons associated to this vertex (called pileup vertex) are removed from the event collection of particles and will not participate in any further object reconstruction. This method is denoted charged hadron subtraction (CHS)

\subsection{Pile up per particle identification (PUPPI)}
In order to mitigate the effect of pileup on jet observables, we take advantage of pileup per particle identification (PUPPI)~\cite{Bertolini2014}.
This method uses local shape information, event pileup properties and tracking information
together in order to compute a weight describing the degree to which a particle is pileup-like.
A local variable $\alpha$ is computed which contrasts the collinear structure of QCD with the soft diffuse radiation coming from pileup.  
The local shape for charged pileup, assumed as a proxy for all pileup particles, is used on an event-by-event basis to calculate a weight for each particle. The weights describe the degree to which particles are pileup-like and are used to rescale their four-momenta, superseding the need for jet-based corrections.

As discussed in Ref.~\cite{Bertolini2014}, various definitions of the discriminating variable $\alpha$ are possible. 
We adopted a configuration to obtain the best discriminating power between pileup and particles from the hard scattering vertex in the pileup scenario under study. Different definitions of $\alpha$ are used for particles in the central ($|\eta| < $ 2.5) and forward region ($|\eta| > $ 2.5) of the detector, where tracking information is not available. However, here we only use particles in the central region and therefore focus only on the usage in this region.\\
In the central region, the shape variable for a given particle $i$ is defined as
%
\begin{linenomath}
\begin{equation}
  \alpha_i = \log \sum_{\substack{j \in \mathrm{Ch,PV} \\ j \neq i}} \left(\frac{p_{T,j}}{\Delta R_{ij}}\right)^{2} \Theta(R_0 - \Delta R_{ij}),
\end{equation}
\end{linenomath}
%
where $\Theta$ is the step function, $i$ refers to the particle in question and $j$ to the neighboring charged particles from the primary vertex within a cone of radius $R_0$. We consider charged particles as coming from the primary vertex if their track is associated to the leading vertex of the event or is unassociated but with $d_z < $0.3~cm, where $d_z$ is the distance along the $z$ axis with respect to the leading vertex. 

%We found that in the forward region, characterized by a lower detector granularity, 
%the usage of both the variables defined in Eq.~\ref{eq:puppiFwd1} and Eq.~\ref{eq:puppiFwd2} improves the performance.
A $\chi^{2}$ approximation
\begin{linenomath}
\begin{equation}
%\chi^{2}_{i} = \frac{(\alpha_i -  <\alpha_{PU}>)^{2}}{RMS_{PU}^{2}},
\chi^{2}_{i} = \frac{(\alpha_i -  \bar{\alpha}_{PU})^{2}}{RMS_{PU}^{2}},
\end{equation}
\end{linenomath}
where $\bar{\alpha}_{PU}$ is the median value of the $\alpha_i$ distribution for pileup particles in the event and $RMS_{PU}$ is the corresponding RMS,
is used to determine the probability of a particle to be from pileup. 
The variables $\bar{\alpha}_{PU}$ and $RMS_{PU}$ are calculated using all charged pileup particles (i.e. all charged particles not from PV).
%The pseudorapidity dependence of $\alpha_{PU}$ and $RMS_{PU}$ is accounted for by computing 
%their values separately in three pseudorapidity bins ($0<|\eta|<2.5$, $2.5<|\eta|<3$ and $|\eta|>3$).
%In the forward region, where the two variables defined in Eq.~\ref{eq:puppiFwd1} and Eq.~\ref{eq:puppiFwd2} are used, 
%the corresponding $\chi^2$ are summed.
Particles are then assigned a weight given by $w_i = F_{\chi^2,NDF=1}(\chi^2_i)$ where $F_{\chi^2,NDF=1}$ is the cumulative distribution function of the $\chi^2$ distribution with one degree of freedom.

The algorithm parameter choices are similar to what is recommended in Ref.~\cite{Bertolini2014}. 
The radius of the cone $R_0$ is set to 0.4.
Particles with weights $w_i$ smaller than 0.01 are rejected.
In addition a cut on the minimum scaled \pt of the neutral particles is applied: 
%$w_i \cdot \pt_i >  (0.2 + 0.02 \cdot n_{PV})$~GeV, 
$w_i \cdot \pt_i >  (A + B \cdot n_{PV})$~GeV, 
where $n_{PV}$ is the reconstructed vertex multiplicity in the event, 
and A and B are tuneable parameters. % which are tuned separately in three pseudorapidity bins. 
%In the pseudorapidity regions $0<|\eta|<2.5$ and $2.5<|\eta|<3$ 
%the parameters are chosen such that jet mass and $\pt$ resolution are optimized, 
%and in the forward region $|\eta|>3$ the parameters are chosen such that MET resolution is optimized. 
No additional pileup corrections are applied to jets clustered from these weighted inputs. 

\section{Jet reconstruction}
\subsection{Jet clustering}
\subsection{Jet substructure reconstruction}
\subsubsection{Grooming}
\subsubsection{N-subjettiness}
\section{Monte Carlo Simulation}
\subsection{Matrix Element Generators}
\subsection{Shower Generators}

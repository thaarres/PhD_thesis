\section{The Standard Model}
Everything this thesis is built on, has its roots in the Standard Model. The Standard Model of particle physics (SM) addresses the question \emph{What is matter made of?} on the smallest possible scale. It maps the fundamental constituents of the universe together through the forces that bind them, hoping to provide a complete picture of the laws of nature. The Standard Model is formulated as a quantum field theory, where the fundamental particles are spin-1/2 fermions which interact with one another through the exchange of bosons. These interaction comes in three forms, each mediated by three different types of gauge bosons: The electromagnetic force, mediated through photons, the weak force, mediated through W and Z bosons, and the strong force, mediated by gluons. How the fundamental particles interact, also defines which properties they exhibit. In addition, the Standard Model includes a field very different from the others, the Higgs field. The Higgs field is felt by both fermions and bosons and is what gives all particles their mass.\newline
One thing the Standard Model fails to incorporate, is the force of gravity. This shortcoming is one of the main motivations for looking for alternative models beyond the Standard Model (BSM), which is the main topic of this thesis.

\subsection{Fundamental particles: Quarks and leptons}
It appears that all matter in the universe can be described by a very small collection of fundamental particles, six leptons and six quarks. These are collectively called fermions and are, as far as we can tell, truly elementary (not composed of any other particles).
Leptons are particles with integer or zero electric charge (defined in units of electron charge). They come in three flavours, or generations, and their mass increases with generation. Each generation of leptons consists of two particles: one charged lepton and one neutrally charged particle denoted \emph{neutrino ($\nu$)}. The three generations can be arranged in a doublet structure, and are as follows
\begin{equation}
\label{eqn:lepton_flavor_doublets}
\begin{pmatrix} e       \\ \nu_e      \end{pmatrix} \qquad
\begin{pmatrix} \mu     \\ \nu_{\mu}  \end{pmatrix} \qquad
\begin{pmatrix} \tau    \\ \nu_{\tau} \end{pmatrix}
\end{equation}
The leptons come in two states; positively charged and negatively charged, where charged is defined in unit of electron charge $e$. The base state is negatively charged, $e^{-}$, $\mu^{-}$, and $\tau^{-}$,  whereas the positively charged leptons,  $e^{+}$, $\mu^{+}$, and $\tau^{+}$ are considered their anti-particles states.
% Each lepton generation is assigned its own quantum number $L$ which must be conserved after any process.
A summary of the lepton properties is listed in Table~\ref{table:theory:lepprop}.
\begin{table}[h!]
\begin{center}
\begin{tabular}{|c|c|c|}% c c c|}
\hline
Lepton        & Mass           & Charge \\%& $L_{e}$ & $L_{\mu}$ & $L_{\tau}$ \\
\hline
$e^{-}$      & $0.5 \mbox{ MeV}$      & $e$ \\%& 1       & 0         & 0 \\
$\mu^{-}$    & $106 \mbox{ MeV}$      & $e$ \\%& 0       & 1         & 0 \\
$\tau^{-}$   & $1777 \mbox{ MeV}$     & $e$ \\%& 0       & 0         & 1 \\
\hline                                      
$\nu_{e}$    & $< 3 \mbox{ eV}$       & $0$ \\%   & 1       & 0         & 0 \\
$\nu_{\mu}$  & $< 0.19 \mbox{ MeV}$   & $0$ \\%   & 0       & 1         & 0 \\
$\nu_{\tau}$ & $< 18.2 \mbox{ MeV}$   & $0$  \\%   & 0       & 0         & 1 \\
\hline
\end{tabular}
\end{center}
\caption{Lepton Properties}
\label{table:theory:lepprop}
\end{table}
Leptons interact with one another through the \emph{electromagnetic and weak force}, which will be explained in more detailed in Section~\ref{sec:theory:ew}.\newline
The other six fundamental matter particles are the \emph{quarks}. They are distinguished from the leptons in that they interact with one another through the \emph{strong force}, described in Section~\ref{sec:theory:qcd}. This force binds the quarks together to form baryons (like protons and neutrons) or mesons (like pions), and in addition keeps the quarks from being observed as free particles (they are only visible through their baryon/meson bound states) Also organized in three generations, the six quarks are called \textit{up}, \textit{down}, \textit{charm}, \textit{strange}, \textit{top} and \textit{bottom}, and are organized in flavor doublets as follow
\begin{equation}
\label{eqn:quark_flavor_doublets}
\begin{pmatrix} u \\ d \end{pmatrix} \qquad
\begin{pmatrix} c \\ s \end{pmatrix} \qquad
\begin{pmatrix} t \\ b \end{pmatrix}
\end{equation}
Each quark comes with a fractional charge of $-\frac{1}{3}$ (u, c and t) and $\frac{2}{3}$ (d, s and b) of one electron charge. As with the leptons, they also come with distinct particles of opposite charge, anti-quarks. As mentioned above, quarks can interact with one another through the strong force. However, they also interact through the weak and electro-magnetic forces.
Some of the quark properties are listed in Table~\ref{table:theory:quarkprop}.
\begin{table}
\begin{center}
\begin{tabular}{|c|c|c|}%c|}
\hline
Quark & Mass & Charge \\% & Properties \\
\hline
u & $1-5 \mbox{ MeV}$         & $\phantom{-}\frac{2}{3} e$  \\%& $I_{z} = \frac{1}{2}$ \\
d & $3-9 \mbox{ MeV}$         & $-\frac{1}{3} e$            \\%& $I_{z} = -\frac{1}{2}$ \\
c & $1.15-1.35 \mbox{ GeV}$   & $\phantom{-}\frac{2}{3} e$  \\%& Charm = +1 \\
s & $75-170 \mbox{ MeV}$      & $-\frac{1}{3}e$             \\%& Strangeness = -1 \\
t & $\approx 174 \mbox{ GeV}$ & $\phantom{-}\frac{2}{3} e$  \\%& Top = +1 \\
b & $4.0-4.4 \mbox{ GeV}$     & $-\frac{1}{3} e$            \\%& Bottom = -1 \\
\hline
\end{tabular}
\end{center}
\caption{Quark Properties}
\label{table:theory:quarkprop}
\end{table}
These 12 fermions, together with their corresponding anti-particles, represent the fundamental particles of the Universe and build up all matter we see around us. We categorize them through which forces they interact with, the fundamental forces, which also has a large impact on the particles properties.
There are four fundamental forces that we know of: Gravity, electromagnetism, the weak force and the strong force. Gravity is so weak compared to the other forces, that it can safely be ignored in the energy domain we probe in this thesis. 
All particles which are electrically charged, interact through the electromagnetic force. In our tables above, that includes the charged leptons ($e$, $\mu$ and $\tau$) and all of the quarks. These interactions are governed by the laws of Quantum Electro Dynamics (QED), and is mediated through the massless and electrically neutral spin-1 photons. All of the fermions, now also including the neutral neutrinos, feel the weak force and undergo weak interactions. The weak force is mediated through vector bosons (W and Z), heavy charged particles of spin-1. Finally, we have the strong force, mediated by the massless and electrically natural spin-1 gluon. Only quarks interact via the strong force, and it is that interaction that makes the quarks so fundamentally different from electrons and neutrinos. The strong force keeps us from observing quarks as free particles, and keep them in bound states referred to as \emph{hadrons}. Their interaction is governed by the laws of Quantum Chromodynamics (QCD). All of these interactions can be represented in one common gauge theory, the Standard Model.
\subsection{The Standard Model Lagrangian}
\label{sec:theory:gauge}
The Standard Model is a quantum field theory where each particle is described as a dynamical field that spreads through all of space-time. These fields are governed by a Lagrangian density function, the Standard Model Lagrangian, where all interactions between the fundamental particles due to the effects of the fundamental forces (not including gravity), can be described as changes in this Lagrangian of quantum fields: The spin-0 Higgs field, the spin-1/2 fermion fields and the spin-1 gauge boson fields. The Standard Model Lagrangian is required to be invariant under local transformations (transformations which are different at every point in spacetime) of its gauge group. All the particle fields are representations of this symmetry group, which obeys global Poincaré symmetry: Invariance under translations in spacetime, rotations in space and boosts. In order to describe the electromagnetic, strong and weak interactions, the gauge group of the Standard Model is the direct product
\begin{equation}
  SU(3)_C \otimes SU (2)_L \otimes U(1)_Y
  \end{equation}
Each of the three subgroups come with a corresponding gauge field.
\subsection{The Quantum Chromodynamics sector}
\label{sec:theory:qcd}
The group $SU(3)_C$ represents the strong force, described by the quantum gauge theory Quantum Chromodynamics (QCD). The gauge field of the group is the gluon field tensor 
\begin{equation}
G_{\mu\nu}^a=\partial_{\mu} \mathcal{A}_{\nu}^a-\partial_{\nu} \mathcal{A}_{\mu}^a+g_s f^{abc}\mathcal{A}_{\mu}^b\mathcal{A}_{\nu}^c
\end{equation}
where $f^{abc}$ is the $SU(3)_C$ structure constant, $g_s$ is the strong coupling and $a$ runs over the eight generators of the group, which correspond to eight massless spin-1 gluons. The conserved charge in QCD is \emph{color} charge $C$, which can be red, green or blue. As $SU(3)_C$ is a non-Abelian group (where a group operation on two group elements depend on the order they are written), the gluons are charged (with one unit of color, one unit of anti-color) and display self-interactions. These self-interactions have severe consequences: Due to the high amount of interaction
Quarks, the only fundamental particles interacting with the strong force, are the simplest representation of $SU(3)$ and come with one unit of color/anti-color which gets rotated by the generators during an interaction 
\subsection{The electroweak sector}
The electromagnetic and weak interactions arise from the breaking of $SU (2)_L \otimes U(1)_Y$ symmetry. The gauge field tensor of $SU (2)_L$,the group of weak left-handed isospin $L$, is $W_{\mu\nu}^a$ where $a$ runs over the 3 generators of the group. The final group, $U(1)_Y$ of weak hypercharge $Y$, is abelian and hence has no self-interaction. The interactions are mediated by a neutral particle with the gauge field tensor $B_{\mu\nu}^a$.\newline
All the fundamental fermions have a \emph{chirality}, defined as the projection of the particles spin along its direction of motion. From observations, the weak interactions is observed to only interact with fermions of left-handed chirality (vector minus axial coupling, V-A). The left-handed fermion fields are therefore in the simplest doublet representation of $SU(2)$ with weak isospin $I=1/2$, where the doublets are as defined in Equation~\ref{eqn:lepton_flavor_doublets} and~\ref{eqn:quark_flavor_doublets}, while the fermions of right-handed chirality are in the singlet representation with weak isospin $I=0$, meaning they do not interact with the gauge bosons of $W_{\mu\nu}^a$.

\subsection{The Higgs Mechanism}
\section{Beyond Standard Model Physics}
\subsection{The hierarchy problem and the gravitational force}
\subsection{Theories of New Physics}
\subsubsection{Warped extra dimensions}
\label{sec:theory:wed}
\subsubsection{Compositeness}
\subsection{Heavy Vector Triplet formalism}
\label{sec:theory:hvt}
	
\chapter{The Large Hadron Collider}
In March 1984, the European Organization for Nuclear Research CERN) and the European Committee for Future Accelerators (ECFA) 
held a workshop in Lausanne entitled "Large Hadron Collider in the LEP Tunnel". 
This is history's first written mention of the Large Hadron Collider (LHC) and the topic under discussion 
was exactly how to build a new type of high-energy collider, capable of bringing hadrons
to collide rather than leptons.
The LHC would be housed in a tunnel which, at the time, was under excavation to host the Large Electron-Positron Collider (LEP) designed to collide leptons with center-of-mass-energies up to around 200 \GeV.
LEP was a circular collider with a circumference of 27 km and the tunnel hosting it was located roughly 100 meters underground beneath France and Switzerland, at the outskirts of Geneva. 
The justification for building a machine like the LHC was that, once LEP got to its maximum center-of-mass energy, a new and more powerful collider would be needed in its place in order to probe higher energies.
While collisions of electrons with positrons provide exceptionally clean and precise measurements due to their being point particles,
 their lightness prevent them from being accelerated to higher energies in circular colliders due to synchrotron radiation. Collisions of hadrons, however, would allow for center-of-mass energies two orders of magnitude higher than that of LEP. Therefore, after obtaining sufficient statistics when running at a center-of-mass-energy of twice the W boson mass (160 \GeV) and reaching a maximum center-of-mass energy of 209 \GeV, in a search for the Higgs boson, LEP was dismantled in 2000 in order to make room for the LHC.
 
The Large Hadron Collider first circulated protons in September 2008 and, while having the same 27-kilometer radius as the LEP collider, is capable of accelerating protons up to a center-of-mass energy of around 14 TeV, 70 times that of LEP. The accelerator consists of two oppositely circulating proton beams, isolated from each other and under ultrahigh vacuum. The protons are accelerated up to speeds close to the speed of light through radio frequency (RF) cavities, before being focused to collide at four different interaction points along the ring.
These four collision points correspond to the location of the four LHC particle detectors; ATLAS, CMS, LHCb and ALICE.
While ATLAS and CMS are general-purpose detectors, built in order to study a large range of different physics processes, 
LHCb and ALICE are built for dedicated purposes; LHCb for b-physics processes and ALICE for heavy ion collisions.
The journey of a proton from a gas to one of the LHC collision points is as follows. First, hydrogen nuclei are extracted from a small tank of compressed hydrogen gas and stripped of their electrons. The remaining protons are then
injected into the LINAC2, a linear accelerator responsible for increasing the proton energy to about 50 MeV through RF cavities that push charged particles forward by switching between positive and negative electric fields. The LINAC2 additionally divides the constant stream of particles into equally spaced "bunches" by careful tuning of the frequency of the field switch.
The accelerated protons are then injected into the Proton Synchrotron Booster (PSB), where their energy is increased by thirty fold, to an energy of roughly 1.4 \GeV. The two final acceleration stages before the protons reach the LHC ring are the Proton Synchrotron and Super Proton Synchrotron, eventually producing protons with a total energy of 450 \GeV. The protons are now ready for the final stage of their travel and are injected into the two beam pipes of the LHC in opposite directions. They are injected in trains of 144 bunches each (on order of $10^{11}$ protons per bunch), where each bunch is roughly 7.5 meters apart (or 25 ns). There are some larger beam gaps present in each beam in order to give special magnets sufficient time to switch on in order to inject or dump the beam. The largest beam abort gap is roughly 3 ms or 900 m long. The ring is filled with proton bunches until these are equally distributed throughout the two rings, a process taking roughly 4 minutes. This is called a "fill". Here, the protons are accelerated to their current maximum energy of 6.5 TeV, a process taking roughly 20 minutes, through eight RF cavities. These RF cavities are also responsible for keeping the proton bunches tightly bunched. A complete sketch of the CERN accelerator complex is shown in Figure~\ref{fig:cms:LHC}.
\begin{figure}[h] 
    \centering
    \includegraphics[width=1.0\textwidth]{figures/cms/LHC.jpg}
    \caption{The Large Hadron Collider accelerator complex. The four collision points along the ring correspond to the location of the LHC particle detectors CMS, LHCb, ATLAS and ALICE~\cite{LHC}.}
    \label{fig:cms:LHC}
\end{figure}
After the beams have reached their maximum energy and are stably circulating in the LHC ring, the bunches are brought to collide. The goal of such a collision, which occurs every 25 nanoseconds, is that some of the protons in each bunch will inelastically collide, allowing the quark and gluon constituents of each proton to interact with one another and produce new and interesting particles.
The number of times such an interaction will take place inside a detector per area and time is quantified through the instantaneous luminosity $\mathcal L$, which is the proportionality factor between the number of observable events per second and the cross section $\sigma$ of the process you are interested in,
\begin{equation}
  \frac{dN_{events}}{dt} =\mathcal L \sigma .
\end{equation}
The cross section is the probability that an event (like one which would produce new and interesting particles) will occur and is measured in barns, where $1 \textrm{ barn} = 10^{-28} \textrm{ m}^2$. This luminosity should therefore be as high as possible. It depends only on parameters of the accelerator beams and can, in the case of the LHC, be defined through accelerator quantities as
\begin{equation}
  \mathcal L = \frac {N_b^2 n_b f_{rev} \gamma_r} {4 \pi \epsilon_n \beta *}F,
\end{equation}
where $N_b$ is the number of particles per bunch, $n_b$ is the number of bunches, $f_{rev}$ is their revolution frequency, $\gamma_r$ is the relativistic gamma factor, $\epsilon_n$ is the normalized transverse beam emittance (how confined the particles are in space and momentum), $\beta *$ is the beta function at the collision point (how narrow, or "squeezed", the beam is) and F is a reduction factor to account for the case where the beams do not collide head-on but at slight crossing angles.
From this, it becomes clear that the main goal of the LHC is to maximize the number of particles ($N_b$,$n_b$), their frequency ($f_rev$) and their energy ($\gamma_r$), while at the same time ensuring the protons are packed together as tightly as possible (lower $\epsilon_n$ and $\beta *$).
Using the nominal values for the LHC, the peak instantaneous luminosity is roughly $\mathcal L \sim 10^{34} \textrm{cm}^{-2} s^{-1}$. 


The peak luminosity of the LHC by the end of Run 2 in 2018 was about $2.0 \cdot 10^{34} \textrm{cm}^{-2} s^{-1}$, corresponding to 2 times the nominal design instantaneous luminosity.

To quantify the size and statistical power of a given LHC dataset, the integrated luminosity is used. This it the integral of the instantaneous luminosity over time and is defined as
\begin{equation}
  \mathcal L_{int} = \int \mathcal L dt.
\end{equation}
It is usually defined in units of inverse cross section, $\textrm{b}^{-1}$.


Despite the LHC starting up in 2008, there would be another year before data taking began due to technical difficulties with the magnets. In March 2010, the LHC saw its first collision with a center-of-mass energy of 7 TeV, and continued running at this energy, collecting around 5 inverse femtobarns of data by the end of 2011. In 2012, the energy was increased to 8 TeV and the LHC continued running until a planned long shutdown scheduled to begin in February 2013, collecting a total of $\sim 20 \textrm{ fb}^{-1}$, which allowed the Higgs boson to be discovered. This marked the end of Run 1 and the beginning of a two-year maintenance project intended to prepare the LHC for running at a center-of-mass energy of 13 TeV, a period referred to as Run 2.

Run 2 started in June 2015, and provides the dataset used in thesis. With the accelerator now running at 90\% of its nominal energy, and with a peak luminosity between 1-2 times the design luminosity, the LHC managed to collect an impressive $\sim 160 \textrm{ fb}^{-1}$ with center-of-mass energy of 13 TeV up until its planned shutdown at the end of 2018. Some key LHC accelerator parameters that were in use for the datasets analyzed in this thesis are quoted in Table~\ref{tab:LHCparameters}.

\begin{table}[]
\begin{tabular}{| l | lllll |}
\hline
Parameter               & Units                                      & Nominal & 2015 & 2016 & 2017     \\
\hline
Energy                  & {[}TeV{]}                                  & 7.0     & 6.5  & 6.5  & 6.5      \\
Bunch spacing           & {[}ns{]}                                   & 25      & 25   & 25   & 25       \\
Bunch intensity         & $\times10^{11}${[}protons/bunch{]}         & 1.15    & 1.15 & 1.15 & 1.2-1.45 \\
Bunches per train       &                                            & 144     & 144  & 96   & 144      \\
Total number of bunches &                                            & 2808    & 2244 & 2220 & 2556     \\
$\beta*$                & {[}cm{]}                                   & 55      & 80   & 40   & 27/25    \\
Peak luminosity         & $\times 10^{34} [\textrm{cm}^{-2} s^{-1}]$ & 1.0     & 0.5  & 1.4  & 2.0      \\
Integrated luminosity   &                                            &         & 4.2  & 39.7 & 50.2     \\
\hline
\end{tabular}
\caption{Some key LHC detector parameters achieved during the first years of data taking with a center-of-mass energy of 13 TeV.}
\label{tab:LHCparameters}
%https://indico.cern.ch/event/663598/contributions/2782540/,https://indico.cern.ch/event/580313/contributions/2359285/attachments/1396590/2135891/Operation_in_2016_v1_1.pdf, http://iopscience.iop.org/article/10.1088/1748-0221/3/08/S08001/pdf
\end{table}


\chapter{The CMS detector}
The Compact Muon Solenoid (CMS) detector is true to its name. With a diameter of 15 meters and a weight of 14000 tons, it is 60\% smaller but twice as heavy as its counterpart, the ATLAS detector.
Its large weight is due to its solenoid: A superconducting niobium titanium coil circulating 18500 Amps and capable of generating a magnetic field of 3.8 Tesla, making it the worlds largest and most powerful magnet. Together with its corresponding iron return yoke, responsible for returning the escaping magnetic flux, it accounts for 90\% of the total detector weight.
The CMS detector is cylindrically symmetric and organized such that the inner tracking system begins at a radius of around 3 cm from the beam pipe. It consists of an inner silicon pixel detector and an outer silicon strip tracker, stretching out to a radius of roughly 1.2 meters. Following the tracker are two calorimeter layers: the electromagnetic calorimeter (ECAL) consisting of lead tungstate scintillating crystals and responsible for measuring the energy of electromagnetically interacting particles, followed by the hadronic calorimeter (HCAL) that measures the energy of hadrons.
Contrary to "standard" configurations for general purpose detectors, the CMS calorimeters are located inside the superconducting solenoid. This allows the detector to be rather compact, by reducing the necessary radii of the calorimeters, and results in a strong magnetic field due to the large coil radius. The muon detectors are alternated with three layers of steel return yoke, responsible for containing and returning the magnetic field. Only muons and weakly interacting particles are expected to transverse the full detector volume without being stopped. Since the muons bend in the magnetic field of the return yoke, an additional momentum measurement can be made. A schematic overview of the CMS detector is shown in Figure~\ref{fig:cms:CMS}. In the following, the different sub-detectors will be described in detail.
\begin{figure}[h] 
    \centering
    \includegraphics[width=1.0\textwidth]{figures/cms/CMS.pdf}
    \caption{The CMS detector and its subsystems: the silicon tracker, electromagnetic and hadron calorimeters, superconducting solenoid and the muon chambers inter-layered with the steel return yoke~\cite{CMS}.}
    \label{fig:cms:CMS}
\end{figure}

\section{Coordinate system}
To describe locations within the CMS detector, a Euclidian space coordinate system is used. Here, the positive z axis points along the beam pipe towards the west, the positive x axis points towards the center of the LHC ring, and the positive y axis up towards the surface of the earth. Due to the cylindrical symmetry of the detector, polar coordinates are more convenient and most frequently encountered. In this scheme, the azimuthal angle $\phi$ is measured in the xy-plane, where $\phi=0$ correspond to the positive x axis and $\phi=\pi/2$ correspond to the positive y axis. The polar angle $\theta$ is measured with respect to the z axis, $\theta=0$ aligning with the positive and $\theta=\pi$ with the negative z axis. To define a particles' angle with respect to the beam line, the pseudorapidity $\eta = -\ln{}\textrm{tan}(\theta/2)$ is preferred over $\theta$. This is due to the fact that particle production is approximately constant as a function of pseudorapidity and, more importantly, because differences in pseudorapidity are Lorentz invariant under boosts along the z-axis when assuming massless particles.
To measure angular difference between particles in the detector, the variable $\Delta \textrm{R}=\sqrt{\Delta \eta^2+\Delta \phi^2}$ is used, which is also Lorentz invariant under longitudinal boosts. A summary of the CMS coordinate system together with some example values are shown in Figure~\ref{fig:cms:coordinates}.
\begin{figure}[h] 
    \centering
    \includegraphics[width=0.45\textwidth]{figures/cms/img_cms_coordinates.png}
    \caption{The CMS coordinate system~\cite{Lenzi:2013xpa}, shown with some values of $\theta$ and $\eta$.}
    \label{fig:cms:coordinates}
\end{figure}

\section{Tracking detectors}
The CMS tracker is responsible for accurately reconstructing the momentum of charged particles and consists of two sub-detectors. Closest to the interaction point, and where the particle intensity is the highest, the silicon pixel detector is located. Upgraded in 2017, from the so-called Phase-0 to the Phase-1 detector, it is structured with four cylindrical barrel layers at radii 2.9, 6.8, 10.9 and 16.0 \cm (the barrel pixel detector) and three disks in each of the forward regions placed at a distance from the nominal interaction point of 29.1, 39.6 and 51.6 \cm (the forward pixel detector). A sketch of the current Phase-1 pixel detector compared to the Phase-0 detector is shown in Figure~\ref{fig:cms:pixel}. The sensors located closest to the beam pipe are subject to hit intensities of $\mathcal{O}( \textrm{MHz}/\textrm{mm}^2 )$ such that strict constraints on the sensor pixel size are required in order to minimize occupancy in the detector. The sensor pixels are 100 \micron x 150 \micron with a thickness of 285 \micron, and when counting both barrel and forward pixel detectors, sum up to a total of 124 million channels. The sensors are mounted on detector modules each with 16 read-out chips, where the type of readout chip depends on how close the module is to the beam pipe. The inner layer uses readout chips with a rate capability of 600 MHz/$\cm^2$, while for the outer layers, readout chips with a rate capability of up to 200 MHz/$\textrm{cm}^2$ are sufficient.
\begin{figure}[h] 
    \centering  
    \includegraphics[height=4cm,keepaspectratio]{figures/cms/20120828_01_pixel_phase1_largesharp.png}
    \includegraphics[height=4cm,keepaspectratio]{figures/cms/20120827_01_pixel_phase1_04.png}
    \caption{Left: The pixel detector layout before (bottom) and after (top) the Phase-1 upgrade. Right: The barrel pixel detector before (left) and after (right) the Phase-1 upgrade~\cite{Dominguez:1481838}.}
    \label{fig:cms:pixel}
\end{figure}
Since the hit intensity reduces as you go further away from the beam pipe, the pixel sensors are replaced by silicon strip sensors of larger size, making up the second of the two tracker sub-systems, the silicon strip tracker. There are ten strip layers in total, stretching out to a radius of roughly 130 cm. These are divided into four sections: The inner barrel (TIB) with four strip layers, the two inner endcaps (TID) consisting of three disks each, the outer barrel (TOB) consisting of 6 cylindrical layers, and the two endcaps (TEC) with 9 strip layers each. A schematic overview of the strip tracker layout is shown in Figure~\ref{fig:cms:tracker}.
The strips in the TIB and TID are 10 cm long, with a width of 80 \micron and a thickness of 320 \micron. The TOB and TEC sections consist of slightly larger strips of 25 \cm x 180 \micron and a thickness of 500 \micron. The strip tracker has a total of 10 million detector strips and covers an area of $\sim 200 \textrm{ m}^2$.
To prolong the silicon detector lifetime, the entire tracker (pixel and strip) is kept at a temperature of \SI{-20}{\celsius} with a dedicated cooling system.
The tracker has a coverage of up to $|\eta|<2.6$ and a resolution of roughly $\sigma / \PT \approx 1.5 \times 10^{-5} \PT \oplus 0.005$.
\begin{figure}[h!] 
    \centering
    \includegraphics[width=1.0\textwidth]{figures/cms/fig_cmstracker.png}
    \caption{Schematic of the CMS silicon strip tracker and its four subsections: the inner barrel (TIB), inner endcaps (TID), the outer barrel (TOB), and the two endcaps (TEC)~\cite{Chatrchyan:2008aa}.}
    \label{fig:cms:tracker}
\end{figure}

\section{Electromagnetic calorimeter}
Surrounding the tracking detectors is the electromagnetic crystal calorimeter (ECAL). Consisting of 75 848 laterally segmented scintillating lead tungstate ($\textrm{PbWO}_4$) crystals, it was designed to have the best possible photon energy and position resolution in order to resolve a Higgs boson decaying into two photons, one of the cleanest discovery channels of the Higgs boson. 
With a design energy resolution of 0.5\% above 100 \GeV for photons and electrons, the choice of detector material for the ECAL has been its most crucial design feature. 
In order to withstand the high doses of radiation and the high magnetic field present within the detector, while at the same time generating well-defined signal responses within the 25 nanoseconds between particle collisions, an extremely dense and transparent material capable of producing fast and clean photon bursts when hit, is required. 
Metal-heavy lead tungstate crystals, each taking roughly two days to artificially grow (and a total of about ten years to grow all of them), were chosen. With a density of $\delta=8.28 \textrm{ g}/\textrm{ cm}^3$ (slightly higher than for stainless steel), the crystals are compact enough to yield excellent performance without taking up too much volume, allowing the ECAL to fit within the CMS superconducting solenoid. The homogeneous medium allows for a better energy resolution as it minimizes the effects from sampling fluctuations, and it additionally contains enough oxygen in crystalline form to make it highly transparent to the entire scintillation emission spectrum. With an extremely short radiation length and small Moliére radius ($\textrm{X}_0=0.85$ \cm, $\textrm{R}_M=2.19$ \cm), the required homogeneity, granularity, and compactness is obtained, while at the same time emitting 80\% of generated light within the 25 ns timeframe required. The largest drawbacks with a lead tungstate detector are the low light yield (100 $\gamma$ per MeV), requiring dedicated avalanche photodiodes to increase the gain, and the light yield, which strongly depends on the temperature. The detector response to an
incident electron changes by 3.8 $\pm$ 0.4 \% per degree Celsius which requires the ECAL temperature to be kept stable at $18.0 \pm 0.5$ degree Celsius, which is obtained through an intricate water cooling system.
The ECAL is completely hermetic and sorted into a barrel part (EB), covering pseudorapidities up to $|\eta|<1.48$, and two endcap parts (EE) extending the total coverage to $|\eta|<3.0$, in order to match the tracker coverage of $|\eta|<2.5$. In order to improve the separation power between the $\gamma$ and $\pi_0$, a pre-shower detector (ES) using lead absorbers and silicon sensors covers the forward region between $1.65<|\eta|<2.6$.
The  crystals in the barrel are organized into supermodules, each consisting of about 1700 crystals, while the endcap is divided into two half disks consisting of 3662 crystals each (so-called "Dees").
Each $\textrm{PbWO}_4$ crystal weighs around 1.5 kg and has a slightly tapered shape with a front face of 2.2 x 2.2 $\cm^2$ in the barrel and 2.86 x 2.86 $\cm^2$ in the endcaps. The crystals are 23 and 22 \cm long in the barrel and endcaps, respectively. The total volume of the calorimeter including barrel and endcaps is 11 $\textrm{m}^2$ and it weighs a total of 92 tonnes.
The ECAL detector layout is illustrated in Figure~\ref{fig:cms:ecal}.
\begin{figure}[h] 
    \centering
    \includegraphics[width=0.7\textwidth]{figures/cms/ecal.jpg}
    \caption{A schematic of the CMS electromagnetic calorimeter showing the barrel supermodules (yellow), the individual barrel crystals (black, top left), the endcap modules (green), and the pre-shower detectors (pink)~\cite{Chatrchyan:2008aa}.}
    \label{fig:cms:ecal}
\end{figure}

Having no longitudinal segmentation, the ECAL relies on an accurate reconstruction of the event primary vertex, provided by the tracker, in order to reconstruct the photon angle correctly.

The obtained energy resolution of the ECAL can be parametrized in three parts: a stochastic, a noise, and a constant term~\cite{Adzic:2007mi}. It is given as

\begin{equation*}
  \frac{\sigma_{\textrm{E}}}{\textrm{E}} = \frac{2.8 \%}{\sqrt{\textrm{E}}}\oplus\frac{12 \%}{\textrm{E}}\oplus 0.3 \%,
\end{equation*}

in which the constant values were estimated using an electron test beam. The constant term of $0.3 \%$ is dominated by the non-uniformity in longitudinal light collection~\cite{1742-6596-587-1-012001}, and one of the main goals of the detector design was to get this term below 1\%. The energy resolution as a function of electron energy is shown in Figure~\ref{fig:cms:ecal-res}.
\begin{figure}[h] 
    \centering
    \includegraphics[width=0.6\textwidth]{figures/cms/ECAL-energy-resolution.pdf}
    \caption{The ECAL energy resolution as a function of electron energy as measured in an electron test beam.~\cite{Adzic:2007mi}}
    \label{fig:cms:ecal-res}
\end{figure}


\section{Hadronic calorimeter}
Outside the electromagnetic calorimeter is the hadronic calorimeter (HCAL). The HCAL is a sampling calorimeter, meaning it consists of alternating layers of dense brass absorber material and plastic scintillators. When a particle hits an absorber plate, it interacts with the absorber material and generates a shower of secondary particles which themselves generate new particle showers. These particles then generate light in the scintillating material which is proportional to their energy, and summing up the total amount of generated light over consecutive layers within a region, called a "tower", is representative of the initial particles energy. It is the combined response of the ECAL and the HCAL that are responsible for measuring the energy of quarks, gluons and neutrinos through the reconstruction of particle jet energy and missing transverse energy. The hadron calorimeter is split into four regions: the inner (HB) and outer (HO) barrel, the endcap (HE) and the forward region (HF).A schematic of the CMS HCAL is shown in Figure~\ref{fig:cms:hcal}.
\begin{figure}[h] 
    \centering
    \includegraphics[width=0.6\textwidth]{figures/cms/HCAL.pdf}
    \caption{The four regions of the CMS hadron calorimeter:  the inner (HB) and outer (HO) barrel, the endcap (HE) and the forward region (HF)~\cite{Chatrchyan:2008aa}}
    \label{fig:cms:hcal}
\end{figure}
The inner barrel lies within the superconducting solenoid volume and covers the pseudorapidity range $|\eta|<1.3$.
It consists of 36 identical wedges, each of which weigh 26 tonnes, split into two half barrels (HB+ and HB-).
A photograph of the wedges taken during installation is shown in Figure~\ref{fig:cms:hcal-wedges}. 
\begin{figure}[h] 
    \centering
    \includegraphics[width=0.45\textwidth]{figures/cms/hcal-2000-010_02.jpg}
    \caption{The installation of the barrel HCAL wedges, consisting of alternating layers of brass absorber plates and plastic scintillator, each weighing roughly 26 tonnes~\cite{Veillet:41645}.}
    \label{fig:cms:hcal-wedges}
\end{figure}
The wedges are made up of flat brass absorber plates oriented parallel to the beam axis. These plates consist of a 4-\cm thick front steel plate followed by eight 5-\cm thick brass plates, six 5.6-\cm thick brass plates and ending with a 7.5-\cm thick steel back plate. The absorber plates alternate with 4-mm thick plastic scintillator tiles, which are the active medium of the detector, and which are read out using wavelength-shifting plastic fibers. The effective thickness of the barrel hadron calorimeter in terms of interaction lengths increases with the polar angle $\theta$, starting out at about 5.8 $\lambda_I$ at an angle of 90 degrees, and increasing to 10.6 $\lambda_I$ at $|\eta|<1.3$.
As the energy resolution of the calorimeter depends on how much of the particles shower can be absorbed by the calorimeter, the quality of the energy measurement depends on its thickness. Due to the CMS design, the HB is confined to the volume between the ECAL (ending at a radius of 1.77 m) and the magnetic coil (starting at a radius of 2.95 m).
In the central $\eta$ region, the combined ECAL and HCAL interaction length is too small to sufficiently contain hadron showers. 
In order to ensure adequate sampling, especially of late starting showers, an additional layer of scintillator has therefore been added outside of the solenoid coil. This is the outer barrel (HO). It uses the coil itself as an absorbing material, increasing the total barrel calorimeter interaction length to 11.8 $\lambda_I$.
The hadron calorimeter endcaps (HE) are located in the forward region close to the beam pipe and cover the pseudorapidity range $1.3 < |\eta|< 3.0$, a region containing about 35\% of the particles produced in collisions. Due to its close proximity to the beam pipe, the endcaps need to handle extremely high rates as well as have a high radiation tolerance.
As the resolution in the endcap region is degraded due to pile-up and magnetic field effects, the hadron calorimeter endcaps were designed to minimize the cracks between HB and HE, rather than having the best single-particle resolution (as is the case for the barrel).
The absorber plates in the endcaps are mounted in a staggered geometry rather than on top of each other as is done in the barrel, in order to contain no dead material and provide a hermetic self-supporting construction.
The HCAL is read out in individual towers with a size $\Delta \eta \times \Delta \phi = 0.087 \times 0.087$ in the barrel, and $0.17 \times 0.17$ at larger pseudorapidities. 
In order to obtain a completely hermetic calorimeter, an additional hadron forward calorimeter (HF) is added in the very forward region.
Stretching out to a pseudorapidity of $|\eta|= 5.2$, this detector is located so close to the beam pipe that the particle rate exceeds $10^{11}$ per $\cm^2$, receiving roughly 760 GeV per proton-proton collision compared to an average of 100 GeV for the rest of the detector. It consists of two cylindrical steel structures, each with an outer radius of 130 \cm and an inner radius of 12.5 \cm, located 11.2 meters on either side of the interaction point. Also a sampling calorimeter, it consists of grooved 5-mm thick steel absorber plates, where the quartz fiber active medium is inserted into these grooves.
The energy resolution of the CMS ECAL and HCAL detectors for pions is measured in a test beam as a function of energy and is shown in Figure~\ref{fig:cms:hcal-res}.
\begin{figure}[h!] 
    \centering
    \includegraphics[width=0.49\textwidth]{figures/cms/hcal_res.pdf}
    \includegraphics[width=0.49\textwidth]{figures/cms/hcal_ecal_res.pdf}
    \caption{The calorimeter energy resolution as a function of pion energy using the HB only or HB+HO (left), and when adding ECAL and HCAL measurements (right)~\cite{Sharma2007}.}
    \label{fig:cms:hcal-res}
\end{figure}
The typical electronics noise of the HCAL is measured to be $~200 \MeV$ per tower. The inclusion of the HO improves the resolution by 10\% for a pion energy of 300 GeV. The final energy resolution parametrization when using ECAL+HB+HO is given by a stochastic and a constant term, as was the case for the ECAL detector, and is
\begin{equation*}
  \frac{\sigma_{\textrm{E}}}{\textrm{E}} = \frac{84.7 \%}{\sqrt{\textrm{E}}}\oplus 7.4 \%.
\end{equation*}

\section{Muon chambers}
The outer part of the CMS detector is dedicated to performing muon identification, momentum measurements, and triggering.
The muon system is made up of three types of gaseous particle detectors: drift tube (DT) chambers, cathode strip chambers (CSCs), and resistive plate chambers (RPCs), all integrated into the magnetic return yoke structure.
In the barrel region, where particle rates are low and the magnetic field uniform, DT chambers are used and cover the pseudorapidity region $|\eta|< 1.2$. In the endcap regions, however, the muon rates and background levels are considerably higher and the magnetic field itself is large and non-uniform. Therefore, CSCs with finer segmentation, higher radiation resistance, and faster signal collection are used, and cover the region $0.9 < |\eta|< 2.4$.
To ensure accurate muon triggering, RPCs are used as a complimentary dedicated muon triggering system, which has been added both in the barrel and in the endcaps. These provide an excellent time resolution and cover the region $|\eta|< 1.6$. These chambers also assist in resolving ambiguities if multiple hits are present within a CSC or DT chamber. A schematic overview of the muon system is shown in Figure~\ref{fig:cms:muon-sys}.
\begin{figure}[h] 
    \centering
    \includegraphics[width=0.49\textwidth]{figures/cms/MuonSys.png}
    \caption{A schematic overview of the muon chambers: the DT chambers in the barrel, the CSCs in the endcaps, and the redundant RPC system stretching out to $|\eta|< 1.6$, which are used for triggering purposes~\cite{Kim:1477844}.}
    \label{fig:cms:muon-sys}
\end{figure}

\chapter{The CMS trigger system}
With bunches in CMS colliding at a rate of 40 MHz, there are only 25 nanoseconds between collisions available to process event data. For typical instantaneous luminosities, one billion collisions take place every second, and with an event size of roughly 1 MB, it is impossible for all of these events to be read out and stored to disk. The CMS triggering system is therefore designed to make ultra fast high-quality decisions about which events are interesting and which events are not.
The first stage of triggering, called Level 1 (L1), is designed to reduce the event rate to a maximum of 100 kHz through custom-designed hardware. It uses coarse data from the muons system and calorimeters in order to make a decision on whether the event should be recorded or not, a decision that needs to happen within $3.2 \mu s$. During this time, the full event data from each sub-detector is stored in detector front-end electronics, awaiting the L1 decision. The information used by L1 is gathered in three steps. First, trigger primitives are created. For the muon system, these consist of track segments from each of the three types of muon detectors. For the calorimeter, trigger primitives are generated by calculating the transverse energy in a trigger tower ( $\Delta \eta- \Delta \phi$ of 0.087 x 0.087) and assigning it to the correct bunch crossing. Trigger primitives from the calorimeter are then passed on to a regional trigger, which defines electrons, muons and jet candidates. Some of this information is passed to the regional muon trigger in order to provide information about whether the particle is minimum ionizing. The global muon trigger then combines track information with calorimeter information and selects a maximum of four muon candidates and calculates their momentum, position, charge and quality. The output from the regional calorimeter trigger is also passed to a global calorimeter trigger which provides information about the jets, total transverse energy and missing energy in the event. Combining the information from the global muon trigger and the global calorimeter trigger, the L1 trigger decides whether to keep the event or not by combining several decisions by simple logic operations (AND/OR/NOT), forming up to 128 algorithms.

If the event is accepted, the full event information is read out at a rate of 100 kHz and passed to the High Level Trigger (HLT), a farm of commercially available computers. Here, the full precision of the detector data is used in order to make decisions based on offline-quality reconstruction algorithms. The goal of the HLT is to eventually reduce the event rate to an average of 400 Hz, which will be stored on tape.

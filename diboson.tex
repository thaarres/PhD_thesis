\begin{singlespace}
\setstretch{1.25}
\begin{centering}
\chapter{Search I: First search for diboson resonances at 13 TeV}
\label{searchI}
\textit{
\noindent When the LHC started its Run II data taking period in summer 2015, it would be the first time ever for a particle collider to produce collisions with center-of-mass energies as high as 13 \TeV. The Higgs boson, for which the LHC was designed to observe, had been discovered at the end of the previous data taking era, leaving us with a Standard Model that we know is either in need of extensions or only an effective theory valid in a certain energy domain. The Run II search program would therefore be oriented around two main efforts: Precision measurements of the newly discovered Higgs boson and searches for physics beyond the standard model.
\newline
\newline
I started my PhD four months before the first 13 TeV collisions took place and had to consider the following:
What was the most interesting search that could be done on a short time scale (to be presented 6 months after first collisions, which would be at the CERN end-of-year "Jamboree"), whose physics objects could be reconstructed and understood on a relatively short time scale, and would be robust enough in case there were issues with the never-before-validated 13 \TeV Monte Carlo?
\newline
\newline
The attention of the high-energy physics community has in the past years been focused on certain "hot topics": In 2018 and currently in 2019, the excitement is over leptoquarks, which could explain anomalies observed by LHCb and b-factories; in 2016 and 2017 it was diphoton resonances, with $>3 \sigma$ excesses observed at the same mass in both CMS and ATLAS. And in 2015 during the 13 \TeV LHC start-up, attention was centered on diboson resonances in the all-hadronic final state. The choice was therefore clear: My first analysis would be a search for diboson resonances in the boosted dijet final state. With a background model based on a smooth fit to data in the signal region, eliminating the need for accurate QCD MC predictions, this was a simple one-background only (QCD) analysis, feasible to finalize in one year, given dedication and sufficient effort. Despite its straightforwardness, due to observed 8 TeV excesses, it was in addition considered a high-profile analysis.
\newline
\newline
This search became one of the first "boosted" searches published with data collected with a 13 TeV center-of-mass energy, as well as the first search to take advantage of dedicated "grooming" triggers. It was published with 2.7 \fbinv of 2015 data.
}
\begin{figure}[b!] 
    \centering
    \includegraphics[height=6.5cm]{figures/analysis/search1/misc/first_coll.png}
    \vspace*{10mm}
    \caption*{\footnotesize{\textit{Published in Journal of High Energy Physics (2017), DOI: 10.1007/JHEP03(2017)162}}}
\end{figure}
\end{centering}
\end{singlespace}
\clearpage
\subsection{A small bump}
On June 2nd 2015, the day before CMS recorded its first ever 13 TeV event, a pre-print appeared on the arXiv "Search for high-mass diboson resonances with boson-tagged jets in proton-proton collisions at $\sqrt{s} = 8$ \TeV with the ATLAS detector"~\cite{Aad2015}.
It was an analysis of the full ATLAS Run 1 dataset, corresponding to 20.3 \fbinv, searching for heavy resonances decaying to vector bosons in the all-hadronic state. The analysis documented a 3.4 $\sigma$ excess for a heavy resonance decaying to \PW\PZ around 2 \TeV.
The corresponding CMS analysis, published the previous year, had a 1.3 $\sigma$ excess at roughly the same resonance mass, but mostly compatible with a \PW\PW final state hypothesis~\cite{Khachatryan:1700394}. Figure~\ref{fig:searchI:8tev} shows the corresponding dijet invariant mass spectrum as seen by ATLAS (left) and the upper limit on the production times the cross section for a $G_{Bulk}$ decaying to \PW\PW (right)  as documented by CMS.

\begin{figure}[h!] 
    \centering
    \includegraphics[width=0.4\textwidth]{figures/analysis/search1/misc/atlas_8tev.png}
    \includegraphics[width=0.4\textwidth]{figures/analysis/search1/misc/EXO-12-024_gWW.pdf}
    \caption{A "bump" corresponding to 3.4 $\sigma$ in the dijet invariant mass spectrum around 2 \TeV (left) observed by ATLAS when analyzing the full 8 \TeV dataset~\cite{Aad2015}, together with a similar excess (1.3 $\sigma$) observed in the corresponding CMS analysis~\cite{Khachatryan:1700394}.}
    \label{fig:searchI:8tev}
\end{figure}

The two measurements were found to be compatible, favoring a heavy resonance with a production cross section of around 5 \fbinv and a mass between 1.9 and 2.0 TeV decaying to either \PW\PW, \PW\PZ or \PZ\PZ~\cite{Dias:2015mhm}. Figure~\ref{fig:searchI:8tevcombo} show the obtained p-value of the ATLAS (red) and CMS (blue) search as well as their combination (black).  

\begin{figure}[h!] 
    \centering
    \includegraphics[width=0.25\textwidth]{figures/analysis/search1/misc/CMS_ATLAS_BulkWW_JJ_dijetfit_p.png}
    \includegraphics[width=0.25\textwidth]{figures/analysis/search1/misc/CMS_ATLAS_BulkZZ_JJ_dijetfit_p.png}
    \includegraphics[width=0.25\textwidth]{figures/analysis/search1/misc/CMS_ATLAS_WZ_JJ_dijetfit_p.png}
    \caption{p-values as a function of resonance mass obtained with an emulation of the ATLAS (red) and CMS (blue) searches as well as the combination of the two (black). Here for a \PW\PW (left), \PW\PZ (middle) and \PZ\PZ (right) hypothesis~\cite{Dias:2015mhm}.}
    \label{fig:searchI:8tevcombo}
\end{figure}

The combination of the two excesses and the timing of the ATLAS paper, naturally lead to some excitement and in the coming weeks, the arXiv was flooded with theory papers seeking an explanation for the measurements.\newline
In addition, one of the main benefits of increasing the LHC center-of-mass energy from 8 to 13 \TeV, was that the partonic luminosity would increase.
That meant that you could expect them same number of signal events as you would expect for the full 8 \TeV dataset ($\sim 20 \fbinv$), for a considerably smaller 13 \TeV dataset. Figure~\ref{fig:searchI:8vs13reach} shows the system mass that can be probed with 3 \fbinv of 13 \TeV data (y-axis), the expected 2015 integrated luminosity, as a function of the probe-able system mass with 20 \fbinv of 8 \TeV data (x-axis) for different partonic channels. The probable 13 \TeV mass is defined by finding the system mass which results in the same number of expected events at 8 \TeV, if assuming cross sections scale with partonic luminosity and $1/m^2$. Three different partonic scattering channels are considered: qq, qg and gg. We see that, for instance for a resonance with a mass of 2000 \GeV, the reach at 13 \TeV is 2241(gg), 2091(gq), 1851(qq, one type) and 2046(qq, all types) \GeV.

\begin{figure}[h!] 
    \centering
    \includegraphics[width=0.50\textwidth]{figures/analysis/search1/misc/colliderReach.png}
    \caption{The system mass that can probed with 3 \fbinv of 13 \TeV data (y-axis) as a function of the probe-able system mass with 20 \fbinv of 8 \TeV data (x-axis) for different partonic channels (Generated with~\cite{collreach}).}
    \label{fig:searchI:8vs13reach}
\end{figure}

What this meant was that, if we saw hints of a 2 \TeV \BulkG (mainly produced through gluon fusion, gg) with the 8 \TeV dataset, we should bee able to confirm it with nothing but the expected 3 \fbinv of data expected in 2015.
The pressure on seeing early results with 13 TeV data in the VV all-hadronic final state was therefore extremely high, and it was agreed with CMS Physics Coordination that a preliminary analysis would be ready in December that same year, only 6 months after the first 13 \TeV collision.

\subsection{Analysis strategy}

When a resonance X with a mass above 1 TeV decays into a vector boson pair, the bosons have a very high energy ($\tilde\PT=\mX/2=500 \GeV$, assuming X is produced at rest). The boson is co-called "boosted". The decay products of a hadronically decaying boosted vector boson, will therefore not appear as back-to-back in the lab frame but rather be very collimated, as described in Section~\ref{sec:objreco:substructure}. This results in a final state with two large high-\PT jets, where an AK R=0.8 jet is expected to fully contain the two quarks coming from the vector boson decay. This is illustrated in Figure~\ref{fig:searchI:merged}.

\begin{figure}[h!] 
    \centering
    \includegraphics[width=0.70\textwidth]{figures/event_reconstruction/WWqqqq_merged_small.pdf}
    \caption{If a heavy ($>1 \TeV$) resonance decays into vector bosons, the transverse momentum of each boson will be large and its decay products are merged into one single large cone AK8 jet.}
    \label{fig:searchI:merged}
\end{figure}

The two jets are both expected to have a mass around the \PW of \PZ mass, and some intrinsic substructure stemming from their two-prong origin. The invariant mass of the dijet system, \mjj, should be roughly equal to the resonance mass \mX. This dijet system is the final state under scrutiny and the dijet invariant mass is the parameter of interest. Both \WW and \ZZ, as well as \WZ final states are of interest. \par

The main background for such an analysis, is QCD multijet events. As mentioned in Section~\ref{sec:objreco:substructure}, quark/gluon jets can obtain a high mass due to diffuse radiation and QCD processes have such a large cross section that the number of QCD jets with a mass compatible with the W mass can be large. In order to discriminate between the two, we take advantage of three properties: 1. The groomed mass of signal and background jets should be very different, 2. signal jets should appear two-prong like, quark/gluon jets not, and 3. the dijet invariant mass for a signal process should peak around the resonance mass while the QCD spectrum is predicted to be smoothly falling (we will get back to why this assumption is justified in Section~\ref{sec:searchI:bkg}). The strategy therefore consists of performing a smoothness test on \mjj of the observed data, a so-called "bump-hunt", by assuming that the signal will appear as a bump on top of a smooth distribution. This is illustrated in Figure~\ref{fig:searchI:bumphunt}.

\begin{figure}[h!] 
    \centering
    \includegraphics[width=0.49\textwidth]{figures/analysis/search1/misc/sigExtraction.pdf}
    \caption{The search strategy consists of looking for signal "bumps" in the dijet invariant mass on top of a smoothly falling QCD multijet background.}
    \label{fig:searchI:bumphunt}
\end{figure}

The benefit of such a method is that there is no need for any background simulation and the strategy is simple and robust. The disadvantage is that the analysis is intrinsically limited to regions where the dijet invariant mass spectrum is smooth, hence must avoid regions with continuities due to trigger turn-ons or kinematic selections.

\subsection{Data and simulated samples}
\label{sec:searchI:samples}
The data analyzed in this search correspond to a total integrated luminosity of 2.7\fbinv collected at a center-of mass energy of 13 \TeV between June and December 2015. The instantaneous luminosity of the LHC during this run was around half of the design luminosity ($0.5 \times 10^{34} \percms$), with an average number of primary vertices per event of $<\mu>=13$. \par\par

The bulk graviton model (see Section~\ref{sec:theory:wed}) and the HVT model (\PWpr{} and \PZpr{}, see Section~\ref{sec:theory:hvt}) are used as benchmark signal processes. In these models, the vector gauge bosons are produced with a longitudinal polarization in more than 99\% of the cases, which leads to a 24\% higher acceptance per boson for reasons explained in Section~\ref{sec:objreco:pol}. For the HVT model, a scenario (model B) with $g_{\rm V}=3$, $c_{\rm H}=-0.976243$, and $c_{\rm F}=1.02433$ is chosen, where the heavy resonance predominantly couple to bosons and the coupling to fermions is suppressed. The bulk graviton samples were generated with $\ktilde = 0.5$.
The resonance masses considered lie in the range 1.2 to 4\TeV and has a width of 0.1\% of the resonance mass. The narrow width allows us to neglect detector effect as the natural width of the resonance is smaller than the detector resolution, making the modeling of detector effects on the signal shape independent of the model. All signal samples are generated at leading order with \amcatnlo{} v2.2.2~\cite{Alwall:2014hca} \par\par

Simulated samples of the production of QCD multijet events are generated to leading order using \PYTHIA version 8.205~\cite{Sjostrand:2007gs} with the CUETP8M1 tune~\cite{Khachatryan:2015pea}.


\subsection{Event selection}

\subsubsection{Triggering}
\label{sec:searchI:trigger}
The first selection to be confronted in any analysis, is the trigger selection. Due to an overwhelming QCD background in all-hadronic final states, the threshold for fully-hadronic triggers is very large in order to keep the trigger rate low (preferably around 10-30 Hertz). In this analysis, we therefore decided to take advantage of triggers that place requirements on the jet groomed mass in addition to the "standard" triggers based on the scalar sum of jet transverse energy \HT. These "boosted" triggers were never before tested in data, and this analysis was the first published result taking advantage of grooming at the trigger level in CMS. The following \HT-based triggers (called inclusive triggers in the following) are used
\begin{itemize}
\item \texttt{HLT\_PFHT650\_WideJetMJJ900DEtaJJ1p5}
\item \texttt{HLT\_PFHT650\_WideJetMJJ950DEtaJJ1p5},
\item \texttt{HLT\_PFHT800}
\end{itemize}
where the two first triggers apply an additional cut on the $|\Delta \eta|$ between the two jets for reasons that will be explained below. In addition, two grooming triggers cutting on the jet trimmed mass (see Section~\ref{sec:objreco:trimming}) of 30 and 50 GeV are used
\begin{itemize}
\item \texttt{HLT\_AK8PFJet360\_TrimMass30}
\item \texttt{HLT\_AK8PFHT700\_TrimR0p1PT0p03Mass50}.
\end{itemize}
The tuneable parameters for the trimming algorithm at HLT are $r_{sub}=0.2$ and $p_{T,frac}=0.03$. The \texttt{HLT\_AK8PFJet360\_TrimMass30} trigger is seeded by single-object Level 1 triggers with jet $p_T$ thresholds of 176 or 200 GeV (\texttt{L1\_SingleJet176} or \texttt{L1\_SingleJet200}), and the remaining triggers requires an online \HT{}$>$150 or 175 GeV (\texttt{L1\_HTT150} or \texttt{L1\_HTT175}).\par

In order to avoid any kinks in the dijet invariant mass spectrum due to the presence of a trigger turn-on, we need to define for which dijet invariant mass the analysis triggers are fully efficient ($>99\%$), then cut away everything below that point.

In order to estimate the trigger efficiency, we use a lower threshold prescaled \HT{} trigger \texttt{HLT\_PFHT650} as reference trigger. This trigger has a prescale of 40, meaning it only stores information for every 40 events that trigger it, and is seeded by L1 triggers \texttt{L1\_HTT150} or \texttt{L1\_HTT175}. We then define the efficiency as
\begin{equation*}
\textrm{Efficiency} = \frac{N_{trigger+ref}}{N_{ref}}  
\end{equation*}
where $N_{trigger+ref}$ corresponds to events passing the trigger under study as well as the reference trigger and $N_{ref}$ corresponds to all events passing the reference trigger. Figure~\ref{fig:searchI:trigger-fits} shows the trigger turn-on curves as a function of dijet invariant mass for jets where one of the jets is required to have a pruned mass larger than 65 GeV (in other words, compatible with a W jet). A sharp turn-on for the inclusive triggers (top left) is observed, reaching the 100\% efficiency plateau for dijet masses of around 1.0--1.1 TeV. The grooming triggers, however, turn on more slowly and are not fully efficient before dijet invariant masses of around 1.2 TeV (top right). The real power of the grooming triggers become clear when adding them in OR with the \HT-based triggers. The bottom plot in Figure~\ref{fig:searchI:trigger-fits} compares the trigger turn-on curves as a function of dijet invariant mass for jets passing one of the three inclusive triggers only, one of the grooming triggers only and when combining all of them. Here, one can see that the 99\% efficiency threshold is lowered by 75 \GeV when including the substructure triggers, once substructure is required at analysis level.
This allowed for the analysis to start at a dijet invariant mass of 1 TeV.

\begin{figure}[h!]
\centering
\includegraphics[width=0.4\textwidth]{figures/analysis/search1/AN-15-211//triggereffMjj-HT.png}
\includegraphics[width=0.4\textwidth]{figures/analysis/search1/AN-15-211//triggereffMjj-SUBST.png}\\
\includegraphics[width=0.4\textwidth]{figures/analysis/search1/AN-15-211/triggereffMjj-ALL.png}
\caption{Top: Efficiency for the inclusive triggers (top left) and the grooming triggers (top right) as a function of dijet invariant mass for jet pairs where one jet has a pruned mass larger than 65 GeV. Bottom: Comparison of trigger efficiencies for jets passing one of the HT-triggers only (red), for jets passing one of the grooming-triggers only (blue) and for jets passing one of the HT-triggers or one of the grooming triggers (green). Here as a function of dijet invariant mass for all jet pairs passing loose selections and where one jet has a pruned mass larger than 65 GeV. The 99\% efficiency threshold is lowered by 75 \GeV when including substructure taggers.}
\label{fig:searchI:trigger-fits}
\end{figure}


As a measure of the performance of the grooming triggers, we have in addition looked at the trigger efficiencies as a function of the offline groomed mass (pruned and softdrop, see Sections~\ref{sec:objreco:pruning} and ~\ref{sec:objreco:softdrop}), for the grooming trigger with the lowest mass threshold (30 \GeV). This is shown in Figure~\ref{fig:searchI:grooming-mj-trigger}, where an additional cut on the jet transverse momentum of one of the jets of 600 GeV is required and no other mass cut is applied. The trigger plateau is reached for offline groomed-jet masses around 50 GeV, an impressively sharp turn-on for a trigger paths first test i data (as reference trigger for this study, the prescaled trigger \texttt{HLT\_PFJet320} was used). 

\begin{figure}[h!]
\centering
\includegraphics[width=0.4\textwidth]{figures/analysis/search1/AN-15-211//triggereff-prunedmass600.pdf}
\includegraphics[width=0.4\textwidth]{figures/analysis/search1/AN-15-211//triggereff-sdmass.pdf}
\caption{Efficiency for the lowest threshold grooming trigger as a function of pruned-jet (left) and softdrop-jet (right) mass for jets with $\PT > \unit{600}{\GeV}$.}
\label{fig:searchI:grooming-mj-trigger}
\end{figure}


\subsubsection{Preselection} 
\label{sec:searchI:preselection}
After trigger selections and the corresponding requirement of a dijet invariant mass above 1 \TeV to ensure a smoothy falling background, the process of maximizing the signal significance while keeping the background low can begin. This is done through a set of jet requirements. The jets used in this analysis are clustered with the anti-\kt{} jet clustering algorithm with a clustering parameter of $R=0.8$ (see Section ~\ref{sec:objreco:jets}) to allow containment of the full vector boson decay products. As we know that a minimum transverse of 200 \GeV is required for the decay products of a \PW/\PZ to be fully contained within an R=0.8 jet, events are further selected by requiring at least two jets with $\PT > \unit{200}{\GeV}$. These are in addition required to be central, with an $|\eta| < 2.4$. \par
The two highest \PT jets in the event passing these criteria are selected as potential vector boson candidates.
As our main background is QCD multijet events, we further take advantage of the fact that the angular distribution between these, mainly t-channel, processes are very different from the s-channel signal processes under study. The crossection for QCD t-channel processes as a function of the opening angle with respect to the beam axis ($\theta*$), exhibit a pole around $\cos \theta*=1$, meaning QCD t-channel jets are mostly forwardly produced, with an opening angle with respect to the beam axis close to zero. The signal jets on the other hand, produced through an s-channel process, are concentrated in the barrel region. We therefore require the jets to have a separation of $|\Delta\eta|<1.3$ in order to reduce the QCD multijets background.
The distribution of $|\Delta\eta|$ between the two highest-\PT jets for QCD as well as for different signal scenarios, is shown in Figure~\ref{fig:searchI:detaopt}

\begin{figure}[h!]
\centering
\includegraphics[width=0.4\textwidth]{figures/analysis/search1/misc/deta_opt.png}
\caption{ $|\Delta\eta|$  between the two highest-\PT jets for QCD jets and jets stemming from different signal scenarios.}
\label{fig:searchI:detaopt}
\end{figure}
 
A cut of $|\Delta \eta|_{jj}<1.3$ makes sure to remove the t-channel pole at $\cos \theta* = 1$ and is in addition found to yield the best separation between signal and the QCD background.
%
% A summary of the applied preselections is as follows:
%
% \begin{itemize}
% \item PF jet tight ID applied
% \item Jet $\eta < 2.4$
% \item Jet \pt $> 200$ GeV
% \item $|\Delta\eta|_{jj} < 1.3$
% \item Dijet invariant mass $> 1$ TeV
% \end{itemize}

In addition the these requirements on the jets themselves, an overlap veto with leptons in the event is applied. Here the overlap $\Delta R(\text{jet},\text{lepton})$ between the jet candidate and a lepton is required to be larger than 0.8. Leptons used for this veto are required to pass the identification requirements described in Section~\ref{sec:objreco:electrons} and~\ref{sec:objreco:muons}, have a transverse momentum larger than 35 (30) GeV and a pseudorapidity smaller than 2.5 (2.4) in case of electrons (muons).

The \PT, $\eta$, dijet invariant mass and $|\Delta \eta|_{jj}$ distribution for the two leading jets in the event after the above preselections have been applied is shown in Figure~\ref{fig:kinematics-all}.

\begin{figure}[h!]
\centering
\includegraphics[width=0.4\textwidth]{figures/analysis/search1/AN-15-211/controlplots/silverjson/Pt_WSignal.pdf}
\includegraphics[width=0.4\textwidth]{figures/analysis/search1/AN-15-211/controlplots/silverjson/Eta_WSignal.pdf}\\
\includegraphics[width=0.4\textwidth]{figures/analysis/search1/AN-15-211/controlplots/silverjson/Mjj_WSignal.pdf}
\includegraphics[width=0.4\textwidth]{figures/analysis/search1/AN-15-211/controlplots/silverjson/DeltaEta_WSignal.pdf}
\caption{Jet \PT (top left), $\eta$ (top right), dijet invariant mass (bottom left) and $|\Delta \eta|_{jj}$ (bottom right) distribution for the two leading jets in the event after loose preselections are applied. The signal is scaled by an arbitrary number.}
\label{fig:kinematics-all}
\end{figure}

\subsubsection{Vector boson tagging}
\label{sec:searchI:wtagging}
After preselections, we take advantage of the jet substructure algorithms described in Section~\ref{sec:objreco:substructure} to further separate boosted W/Z jets from the QCD multijets background. In the 8 \TeV analysis~\cite{Khachatryan:1700394} published the previous year, the pruning algorithm was the groomer of choice. However, recent progress had been made in the development of alternative groomers which had favorable properties from a theoretical point of view (see Sections~\ref{sec:objreco:grooming} and ~\ref{sec:searchII:puppisoftdrop}). We therefore studied two different grooming algorithms: pruning and softdrop (with $\beta=0$ and $z_{cut} = 0.1$). A comparison of the softdrop (dotted lines) and pruned (solid lines) jet mass for \PW, \PZ and \PH jets is shown in Figure~\ref{fig:searchI:sdvspruning}.

 \begin{figure}[h!]
 \centering
 \includegraphics[width=0.49\textwidth]{figures/analysis/search1/misc/SDvsPruned.pdf}
 \caption{The softdrop (dotted lines) and the pruned (solid lines) jet mass for \PW, \PZ and \PH jets.}
 \label{fig:searchI:sdvspruning}
 \end{figure}
 
One of the first observations we made comparing the two groomers, was that there appeared to be a strong dependence of softdrop mass on the jet \PT. Figure~\ref{fig:searchI:grommedmassshift} shows the pruned (left) and softdrop (right) mass distributions for \PW jets coming from the decay of a \BulkG with a resonance mass of $0.8 \TeV < \mX < 4 \TeV$. While the pruned jet mass mean appeared stable as the jet transverse momenta of the jet increased ($\PT\sim\mX/2$), the softdrop jet mass mean shifted towards lower values as jet \PT increased.



\begin{figure}[h!]
\centering
\includegraphics[width=0.49\textwidth]{figures/analysis/search1/misc/pruned_mass_shift.pdf}
\includegraphics[width=0.49\textwidth]{figures/analysis/search1/misc/softdrop_mass_shift.pdf}
\caption{The jet mass distribution for W jets coming from a $\textrm{G}_{\textrm{bulk}}$ of masses in the range $0.8 \TeV < \mX < 4 \TeV$ decaying to \WW, here with pruning applied (left) and softdrop (right). A strong shift in the jet mass mean as a function of \PT ($\sim\mX/2$), is observed for jets groomed with the softdrop algorithm. Charge hadron subtraction is applied to all jets before clustering.}
\label{fig:searchI:grommedmassshift}
\end{figure}

In order to investigate whether this was a reconstruction effect or an algorithmic effect, we additionally looked at the pruned and softdrop mass for generator level jets (jets clustered with generator level particles before they are passed through the detector simulation). Figure~\ref{fig:searchI:grommedmassshift_genvsreco} shows the reconstructed (solid line) and generator level (dotted line) jet mass distributions after pruning (left) or softdrop (right) have been applied. Again, the distributions are compared for jets with very different \PT profiles, here for W jets coming from a $\BulkG \rightarrow \WW$ of mass $\mX = 0.8 \TeV$ (red), roughly $\PT\sim 400 \GeV$, and $\mX = 2.0 \TeV$ (blue), $\PT\sim 1 \TeV$.
Interestingly, we observe a \PT-dependent mass shift already for generator level softdrop jets (comparing the dotted lines in the right plot); an effect further enhanced at reconstruction level. This effect is not present for pruned jets, neither at generator level nor reconstruction level.


\begin{figure}[h!]
\centering
\includegraphics[width=0.49\textwidth]{figures/analysis/search1/misc/pruned_mass_shift_genvsreco.pdf}
\includegraphics[width=0.49\textwidth]{figures/analysis/search1/misc/softdrop_mass_shift_genvsreco.pdf}
\caption{The reconstructed (solid line) and generator level (dotted line) jet mass distribution for W jets coming from a $\BulkG \rightarrow \WW$ of mass $\mX = 0.8 \TeV$ (red), roughly $\PT\sim 400 \GeV$, and $\mX = 2.0 \TeV$ (blue), $\PT\sim 1 \TeV$. Here for the pruned (left) and softdrop (right) jet mass.}
\label{fig:searchI:grommedmassshift_genvsreco}
\end{figure}

The observed softdrop mass \PT-dependence was problematic, due to the fact that it would require a \PT dependent mass window. This would again require several different measurements of data to simulation tagging efficiency scale factors, for the respective mass windows, or a significantly higher uncertainty on the signal yield. Due to these observations, the grooming algorithm of choice for this analysis is pruning, with $65 \GeV < m_{pruned} < 105 \GeV$. However, this would be a study we would return to in Search II (Section~\ref{searchII}).
\newline\newline
The shape tagger we chose for this analysis was the n-subjettiness ratio \nsubj. \nsubj is strongly correlated to the pruned jet mass, and the discriminating power of the variable is reduced when applying a pruned mass cut. The \nsubj distribution for the QCD background and \PW jets from a signal decay before (left) and after (right) a pruned mass cut of $65 \GeV < m_{pruned} < 105 \GeV$ have been applied, is shown in Figure~\ref{fig:searchI:tau21_groomedvsungroomed}.

\begin{figure}[h!]
\centering
\includegraphics[width=0.49\textwidth]{figures/analysis/search1/misc/tau21_ungroomed.pdf}
\includegraphics[width=0.49\textwidth]{figures/analysis/search1/misc/tau21_groomed.pdf}
\caption{The \nsubj distribution for QCD background and signal jets before (left) and after (right) a pruned mass window is applied. The discriminating power of \nsubj is strongly reduced after grooming.}
\label{fig:searchI:tau21_groomedvsungroomed}
\end{figure}

We therefore perform a cut optimization on \nsubj after all analysis selections, including a pruned mass window of $65 \GeV < m_{pruned} < 105 \GeV$, have been applied. This is done by scanning the \nsubj cut, and for each cut computing the Punzi significance~\cite{Punzi:2003bu} defined as
\begin{equation*}
\textrm{S} = \frac{\epsilon_S}{1+\sqrt{\textrm{B}}}  
\end{equation*}  
where $\epsilon_S$ is the signal efficiency and B is the total background. The cut with the highest significance is defined as the optimal cut value. The signals under consideration are W jets coming from the decay of a \BulkG with $\mX = 1-4 \TeV$, against a background of light flavored QCD jets. Only jets with a dijet invariant mass in a 20\% window around the resonance mass are considered. The Punzi significance as a function of the upper cut value on \nsubj is shown on the left in Figure~\ref{fig:searchI:tau21_punzi}.

\begin{figure}[h!]
\begin{center}
\includegraphics[width=0.49\textwidth]{figures/analysis/search1/AN-15-196/tau21optimisation/HP_Punzi_BulkWW.png}
%\includegraphics[width=0.49\textwidth]{figures/analysis/search1/misc/tau21_punzi.pdf}
\includegraphics[width=0.49\textwidth]{figures/analysis/search1/AN-15-196/tau21optimisation/HP_CutSignificance_bulkWW.png}\\
\caption{Optimal upper cut on $\tau_{21}$ as a function of \BulkG mass (left).  The optimal $\tau_{21}$ selection for W' (HTV model) resembles the Bulk graviton selection.}
\label{fig:searchI:tau21_punzi}
\end{center}
\end{figure}

The optimal cut gets looser as the dijet invariant mass increases, something which can be understood when looking at the QCD dijet invariant mass spectrum in Figure~\ref{fig:kinematics-all}. The number of QCD jets falls of exponentially with \mjj, meaning that the background at 4 \TeV is considerably lower than at 1 \TeV. This allows for a looser cut on \nsubj as \mjj increases. In order to choose a single cut which works reasonably well for all masspoints, we look at the ratio of a given \nsubj cut over the significance of the best cut at that mass points. This is shown in the right plot of Figure~\ref{fig:searchI:tau21_punzi}. The cut $\tau_{21}<0.45$ has the most stable performance out of the investigated cut values and is due to that, and due to the desire of keeping the background as low as possible at low \mjj, chosen as the nominal cut. In order to account for the fact that background is lower at high-\mjj, we add an additional analysis category, $0.45 < \nsubj < 0.75$, which contains $>95\%$ of the signal and enhances the analysis sensitivity where the background is low.
These categories are hereafter referred to as the "high purity" (HP) category, for jets with $0<\tau_{21} \leq 0.45$, and the low purity (LP) caegory, for jets with $0.45<\tau_{21}\leq0.75$.
\newline
\newline
The W-tagging efficiency and QCD light-flavored jet mistagging rate for a W-tagger consisting of $0<\tau_{21} \leq 0.45$ and  $65 \GeV < m_{pruned} < 105 \GeV$ is shown in Figure~\ref{fig:searchI:wtageff}, both as a function of jet \PT and as a function of number of primary vertices in the event.


\begin{figure}[h]
\begin{center}
\includegraphics[width=0.49\textwidth]{figures/analysis/search1/misc/WtagSigEff_vpT_pruned.pdf}
\includegraphics[width=0.49\textwidth]{figures/analysis/search1/misc/WtagSigEff_vnPVs_pruned.pdf}
\caption{The W-tagging efficiency (green) and light jet mistag rate (grey) for a pruned jet mass cut only and pruned jet mass + \nsubj cut as a function of \PT (left) and number of primary vertices (right). }
\label{fig:searchI:wtageff}
\end{center}
\end{figure}


The signal efficiency for a pruned jet mass cut only, is around 80 \% with a mistag rate of $\sim 15\%$. After adding a \nsubj cut, the signal efficiency drops to around 55\% and the mistagging rate to $\sim 2\%$. Another interesting feature is the dependence of \nsubj on \PT on pileup, compared to the resilience of the groomed mass as a function of the same variables. This will be another feature we explore in Search II (Section~\ref{searchII}). Figure~\ref{fig:wtag} shows the pruned-jet mass (left) and the n-subjettiness $\tau_{21}$ distribution (right) for signal and background Monte Carlo, as well as the distributions measured in data. 

\begin{figure}[h!]
\centering
\includegraphics[width=0.4\textwidth]{figures/analysis/search1/AN-15-211/controlplots/silverjson/PrunedMass_WSignal.pdf}
\includegraphics[width=0.4\textwidth]{figures/analysis/search1/AN-15-211/controlplots/silverjson/Tau21_punzi_WSignal.pdf}\\
\caption{Pruned jet mass distribution (left) and n-subjettiness $\tau_{21}$ (right) distribution for data and simulated samples. Simulated samples are scaled to match the distribution in data. The $\tau_{21}$ distribution is shown for jets after a cut of $65 {\GeV} < M_{p} < 105 {\GeV}$ has been applied.}
\label{fig:wtag}
\end{figure}


\subsubsection{Analysis categorization}
As the analysis requires two W/Z-tags, we always require one HP tagged jet and then divide into LP and HP categories depending on whether the other jet is of high or low purity. In addition, in order to further enhance the analysis sensitivity, we further split the pruned jet mass window into a W and a Z boson window where the W window is defined as $65 {\GeV} < m_{pruned} < 85 {\GeV}$ and the Z boson window as $85 {\GeV} < m_{pruned} < 105 {\GeV}$. This has the added benefit of allowing us to make a discrimination between a \BulkG decaying to \WW or \ZZ, and a \PWpr decaying into \WZ through counting events in each category. We, for instance, expect a higher signal yield in \WZ category for a \PWpr decaying to a \PW and \PZ boson then for a \BulkG decaying to \WW or \ZZ. Figure~\ref{fig:searchI:massCatWpr} shows the relative expected signal yield (left) and expected limits (left) in the different mass categories for a 2 \TeV \PWpr.
 
 \begin{figure}
 \centering
 \begin{minipage}{0.5\textwidth}
 \centering
 \includegraphics[width=0.99\textwidth]{figures/analysis/search1/misc/massCategories.pdf}
 \end{minipage}
 \begin{minipage}{0.29\textwidth}
 \centering
 \captionsetup{type=table} %% tell latex to change to table
 \begin{tabular}{| l | c |}
 \hline
 \multicolumn{2}{|c|}{$\PWpr (2 \TeV) \rightarrow \WZ$}\\
 \hline
 Category & Expected limit \\
 \hline
 WWHP & 2.1984 \\ 
 WWLP & 2.3261 \\ 
 WZHP & 1.2419 \\ 
 WZLP & 1.7157 \\ 
 ZZHP & 7.0855 \\ 
 ZZLP & 9.2012 \\ 
 \hline
 \end{tabular}
 % \caption{table caption goes  here}\label{tab:searchI:massCatWpr}
 \end{minipage}
 \caption{The expected signal yield per mass category for \PWpr (2 \TeV) decaying to a \PW and \PZ (left) together with the expected limit per mass category for the same signal (right).}
 \label{fig:searchI:massCatWpr}
 \end{figure}
All categories are combined in the end, leading to the same or better sensitivity than when using the whole pruned mass window. 
Figure~\ref{fig:searchI:massCategories} shows the expected 95\% CL upper limits on the production cross section of a \PWpr decaying to \WZ (left) and a \BulkG decaying to \WW (right) as function of the resonance mass in the HP category. The blue line corresponds to the expected limits obtained when not splitting into mass categories and the red line corresponds to the limit using the combination of two categories. The dotted and solid black lines are the limits in the \PW and \PZ categories, respectively. The combination of two mass categories leads to a slightly better (~10\%) or to the same sensitivity as when using one large mass window.

\begin{figure}[h!p]
 \centering
 \includegraphics[width=0.49\textwidth]{figures/analysis/search1/AN-15-196/massCategories/compare-HP-HPV-Wprime.png}
  \includegraphics[width=0.49\textwidth]{figures/analysis/search1/AN-15-196/massCategories/compare-HP-HPV-BulkG.pdf}
 \caption{Expected 95\% CL upper limits on the production cross section of a \PWpr (left) and \BulkG (right) signal as function of the resonance mass for the different mass categories for events passing the high-purity $\tau_{21}$ selections.}
 \label{fig:searchI:massCategories}
 \end{figure}

The real benefit of splitting into mass categories becomes obvious when defining a test statistics based on the likelihood ratios of each signal hypothesis, $q = -2 \ln(L_{\BulkG}/L_{\PWpr})$, shown in Figure~\ref{fig:searchI:signalsep}. For a signal with a signal strength corresponding to a 3-4 $\sigma$ excess, the test statistics for each signal hypothesis are well separated ($\sim3.5 \sigma$), allowing us to make a statement of how \BulkG or \PWpr like a possible signal is.
\begin{figure}[h!p]
 \centering
 \includegraphics[width=0.49\textwidth]{figures/analysis/search1/AN-15-196/massCategories/sig_sep.pdf}
 \caption{Distribution of the test statistic  $q = -2 \ln(L_{\BulkG}/L_{\PWpr})$ for a \BulkG (blue) and \PWpr signal hypothesis.}
 \label{fig:searchI:signalsep}
 \end{figure}
With the high-purity and low-purity categories as defined above for each mass window combination, this leaves us with six different signal categories. They are as follows:
\begin{itemize}
\item High-purity, 3 mass categories: \WW, \ZZ and \WZ
\item Low-purity , 3 mass categories: \WW, \ZZ and \WZ
\end{itemize}
In parallel to the mass-category based analysis, we perform an analysis without categorization in mass (similar to the 8 \TeV analysis) as a cross-check, and found the sensitivity with mass categories to be better.
The final tagging efficiency for different signal hypothesis (top) together with the QCD mistag rate (bottom) in the different signal categories is shown in Figure~\ref{fig:search1:sigeff}. The solid lines represent the tagging efficiency in the full mass window ($65 {\GeV} < M_{p} < 105 {\GeV}$) before splitting into mass categories. A lower signal efficiency the ZZ mass category is observed in all cases. This can be explained from the pruned jet mass distribution on the left in Figure~\ref{fig:wtag}, where a cut at 85 GeV leaves a large fraction of the Z peak in the W mass window. As the main benchmark models under consideration preferably decays to W bosons (in the Bulk Graviton model the branching ratio BR($G_{Bulk}$ $\rightarrow$ \PW\PW) = 2* BR($G_{Bulk}$$\rightarrow$ ZZ), and in the HVT model W'/Z' $\rightarrow$ WZ/WW (but not ZZ) ), a high tagging efficiency for the W boson is preferred. In the limit-setting procedure all the categories are combined and the overall signal efficiency is conserved. For the combined mass-categories (solid line) the signal efficiency is between 16 and 23 \% in the double-tag categories, and between 20 and 34 \% in the single-V tag categories. The mistagging rate in the double-V tag categories is below 1 \% in the high-purity category.
\begin{figure}[h!]
\centering
\includegraphics[width=0.49\textwidth]{figures/analysis/search1/AN-15-211/HP_VV_SigEff.png}
\includegraphics[width=0.49\textwidth]{figures/analysis/search1/AN-15-211/LP_VV_SigEff.png}\\
\includegraphics[width=0.49\textwidth]{figures/analysis/search1/AN-15-211/QCD_HP_VV_MistaggingRateEff.pdf}
\includegraphics[width=0.49\textwidth]{figures/analysis/search1/AN-15-211/QCD_LP_VV_MistaggingRateEff.pdf}
\caption{Tagging efficiency (top) and mistagging rate (bottom) in the different pruned mass categories in the high-purity category (left) and in the low-purity category (right)}
\label{fig:search1:sigeff}
\end{figure}
The full analysis selections and final categories are listen in Table~\ref{tab:search1:selection}.
\begin{table}[!h!]
\footnotesize
\begin{center}
\renewcommand{\arraystretch}{1.2}
\begin{tabular}{lc}
\hline 
\multicolumn{1}{c}{Selection} & Value\\
\hline \hline
\multicolumn{1}{c}{Boson selections}\\
\cline{1-1}
V $\to\qqbar$ (2 AK8 jets) & $\pt >200\GeV$\\
  & $|\eta| < 2.4$\\
Pruned jet mass & $65 < {\mJ}_1,{\mJ}_2  < 105\GeV$\\
Topology    & $|\Delta \eta_\mathrm{jj}| < 1.3$\\
Dijet invariant mass     & $\mjj >1\TeV$\\ 
2- to 1-subjettiness ratio    & $\nsubj < 0.75$\\
\hline
%\hline
\multicolumn{1}{c}{\mJ{} categories}\\
\cline{1-1}
WW & $ 65 < {\mJ}_{1} < 85\GeV$, $ 65 < {\mJ}_{2} < 85\GeV$\\
WZ & $ 65 < {\mJ}_{1} < 85\GeV$, $ 85 < {\mJ}_{2} < 105\GeV$\\
ZZ & $ 85 < {\mJ}_{1} < 105\GeV$, $ 85 < {\mJ}_{2} < 105\GeV$\\
\hline
\multicolumn{1}{c}{\nsubj{} categories}\\
\cline{1-1}
High-purity   & $\tau_{\rm{21, jet1}} < 0.45$, $\tau_{\rm{21, jet2}} < 0.45$\\
Low-purity    & $\tau_{\rm{21, jet1}} < 0.45$, $0.45 < \tau_{\rm{21, jet2}} < 0.75$\\
\hline						       
\end{tabular}
\caption{The full analysis selections, mass and \nsubj categories.}
\label{tab:search1:selection}
\end{center}
\end{table}
\clearpage

\subsection{Background modeling}
\label{sec:searchI:bkg}

The background modeling in this analysis is based on a smoothness test performed directly on unblinded data, similar to what is done in previous CMS analyses looking for bumps in the dijet invariant mass spectrum~\cite{Chatrchyan:2012ypy,CMS-PAS-EXO-12-059}. We assume that the QCD multijet background in the different analysis categories can be described by smooth, monotonically decreasing functions of 2 or 3 parameters
\begin{equation}
\label{eq:dijet1}
\frac{dN}{d\mjj}= \frac{ P_0 } { (\mjj/\sqrt{s})^{P_2} }\quad\quad\quad{\rm and}
\quad\quad\quad\quad
\frac{dN}{d\mjj}= \frac{ P_0(1-\mjj/\sqrt{s})^{P_1} } { (\mjj/\sqrt{s})^{P_2} }\:\:,
\end{equation}
where $m$ is the dijet invariant mass, $\sqrt{s}$ the centre of mass energy and $P_0$ is a normalization parameter for the probability density function and $P_1$ and $ P_2$ describe the shape. The number of fit parameters is decided through a Fishers F-test~\cite{RePEc:bla:istatr:v:80:y:2012:i:3:p:491-491}. In this test, we start from the 2 parameter function and compare the goodness of fit ($\chi^2$ divided by degrees of freedom) when fitting the data signal region with a 2, 3, 4 and 5 parameter function. We then check at 10\% confidence level (CL) if additional parameters are needed to model the background distribution. The 4 and 5 parameter functions are
\begin{align}
\label{eq:dijet2}
\frac{dN}{d \mjj} &= \frac{ P_0(1-\mjj/\sqrt{s})^{P_1} } {(\mjj/\sqrt{s})^{P_2+P_3\times\log(\mjj/\sqrt{s})} }\\
\frac{dN}{d \mjj} &= \frac{ P_0(1-\mjj/\sqrt{s})^{P_1} } {(\mjj/\sqrt{s})^{P_2+P_3\times\log(\mjj/\sqrt{s})+P_4\times\log(\mjj/\sqrt{s})^2} }
\end{align}
where $P_3$ and $P_4$ are additional free parameters. As an additional cross check, an alternative fit function is also tested:
\begin{equation}
\label{eq:dijet4}
\frac{dN}{d\mjj} = \frac{ P_0(1-\mjj / \sqrt{s}+P_3(\mjj / \sqrt{s})^2)^{P_1} } { (\mjj/\sqrt{s})^{P_2} }.
\end{equation}

The fit range is chosen such that it start where the trigger efficiency has reached its plateau to avoid bias from trigger inefficiency, and extends to the bin after the highest $m_{VV}$ mass point. The binning chosen for the fit follows the detector resolution as in~\cite{Chatrchyan:2012ypy,CMS-PAS-EXO-12-059}. Before unblinding the signal region, we check that the QCD dijet invariant mass spectrum is expected to be smooth from the distribution in QCD MC as well as exercise the F-test in QCD MC and in a data sideband.\par
The fits to data in the signal region using the different fit functions, are shown in Figure \ref{fig:searchI:fit-dataVV}, and the corresponding F-test output are given in Table \ref{tab:WW_enriched} through Table \ref{tab:ZZ_enriched}. The findings can be summarized as follows: for the WW enriched category a 2 parameter fit is sufficient to describe the data in both the high- and low-purity categories. In the WZ category, a two parameter fit is sufficient in the high-purity category, while three parameters are needed for the low-purity category. For the ZZ category, a 3 parameter fit is needed for both purity categories. The 2 and 3 parameters fit functions as defined in Equation \ref{eq:dijet2} will therefore be used to model the background component in the simultaneous signal and background fit.
\par

\begin{figure}[h!]
\centering
\includegraphics[width=0.43\textwidth]{figures/analysis/search1/AN-15-211/ftest/no5par/WWHP_fitComp.pdf}
\includegraphics[width=0.43\textwidth]{figures/analysis/search1/AN-15-211/ftest/no5par/WWLP_fitComp.pdf}\\
\includegraphics[width=0.43\textwidth]{figures/analysis/search1/AN-15-211/ftest/no5par/WZHP_fitComp.pdf}
\includegraphics[width=0.43\textwidth]{figures/analysis/search1/AN-15-211/ftest/no5par/WZLP_fitComp.pdf}\\
\includegraphics[width=0.43\textwidth]{figures/analysis/search1/AN-15-211/ftest/no5par/ZZHP_fitComp.pdf}
\includegraphics[width=0.43\textwidth]{figures/analysis/search1/AN-15-211/ftest/no5par/ZZLP_fitComp.pdf}\\
\caption{Fitted dijet mass spectrum in the different mass and purity categories in data for the double V-tag category. A 2 parameter fit is sufficient to describe the data for the WW (HP and LP) and WZ (LP) enriched categories. For the ZZ enriched (HP and LP) and WZ (HP) categories, a 3 parameter fit is needed.}
\label{fig:searchI:fit-dataVV}
\end{figure}

\begin{table}[h!]
\centering
\begin{tabular}{|l c c c |}
\hline
\multicolumn{4}{|c|}{WW enriched, HP}\\
\hline
Function & Residuals & $\chi^2$ & ndof \\
\hline
2 par & 0.034 & 9.279 & 11 \\
3 par & 0.034 & 9.160 & 10 \\
4 par & 0.040 & 8.030 & 9 \\
\hline
\hline
Fishers23  & -0.053 &CL &1.0\\
Fishers34  & -1.456 &CL &1.0\\
\hline
\end{tabular}
\quad
\begin{tabular}{|l c c c |}
\hline
\multicolumn{4}{|c|}{WW enriched, LP}\\
\hline
Function & Residuals & $\chi^2$ & ndof \\
\hline
2 par & 0.270 & 13.462 & 17 \\
3 par & 0.300 & 13.819 & 16 \\
4 par & 0.324 & 13.680 & 15 \\
\hline
\hline
Fishers23 & -1.723& CL & 1.0\\
Fishers34 & -1.191& CL & 1.0\\
\hline
\end{tabular}
\caption{Residuals, $\chi^{2}$, and degrees of freedom for the WW enriched HP and LP categories. A 2 parameter fit is needed to describe the data in both categories.}
\label{tab:WW_enriched}
\end{table}


\begin{table}[h!]
\centering
\begin{tabular}{|l c c c |}
\hline
\multicolumn{4}{|c|}{WZ enriched, HP}\\
\hline
Function & Residuals & $\chi^2$ & ndof \\
\hline
2 par & 0.039 & 9.105 & 16 \\
3 par & 0.047 & 7.915 & 15 \\
4 par & 0.048 & 8.370 & 14 \\
\hline
\hline
Fishers23 & -2.598& CL & 1.0\\
Fishers34 & -0.491& CL & 1.0\\
\hline
\end{tabular}
\quad
\begin{tabular}{|l c c c |}
\hline
\multicolumn{4}{|c|}{WZ enriched, LP}\\
\hline
Function & Residuals & $\chi^2$ & ndof \\
\hline
2 par & 1.016 & 17.602 & 20 \\
3 par & 0.270 & 11.424 & 19 \\
4 par & 0.269 & 11.421 & 18 \\
\hline
\hline
Fishers23 & 55.258& CL & 0.0\\
Fishers34 & 0.078& CL & 0.783\\
\hline
\end{tabular}
\caption{Residuals, $\chi^{2}$, and degrees of freedom for the WZ enriched HP (left) and LP (right) categories. A 2 parameter fit is sufficient to describe the data in the high-purity category, while three parameters are needed for the low-purity category.}
\label{tab:WZ_enriched}
\end{table}



\begin{table}[h!]
\centering
\begin{tabular}{|l c c c |}
\hline
\multicolumn{4}{|c|}{ZZ enriched, HP}\\
\hline
Function & Residuals & $\chi^2$ & ndof \\
\hline
2 par & 0.220 & 9.901 & 11 \\
3 par & 0.140 & 9.511 & 10 \\
4 par & 0.124 & 9.781 & 9 \\
\hline
\hline
Fishers23 & 6.302& CL & 0.029\\
Fishers34 & 1.246& CL & 0.290\\
\hline
\end{tabular}
\quad
\begin{tabular}{|l c c c |}
\hline
\multicolumn{4}{|c|}{ZZ enriched, LP}\\
\hline
Function & Residuals & $\chi^2$ & ndof \\
\hline
2 par & 0.448 & 18.832 & 15 \\
3 par & 0.121 & 17.463 & 14 \\
4 par & 0.118 & 17.394 & 13 \\
\hline
\hline
Fishers23 & 40.438& CL & 0.0\\
Fishers34 & 0.356& CL & 0.56\\
\hline
\end{tabular}
\caption{Residuals, $\chi^{2}$, and degrees of freedom for the ZZ enriched LP and HP categories. A 3 parameter fit is sufficient to describe the data in both categories.}
\label{tab:ZZ_enriched}
\end{table}

\clearpage
\subsection{Signal modeling}
\label{sec:searchI:sig}

The signal shape is extracted from signal MC with masses in the range from 1 to 4 TeV. A linear interpolation provides shapes for the mass points in between in steps of 100 GeV. From these shapes, pdf models are constructed as composite models with a Gaussian core due to detector resolution and an exponential tail to account for parton distribution function effects. Parametric shape uncertainties due to jet energy scale and resolution uncertainties are inserted by variations of the Gaussian peak position and width. The dijet invariant mass shape for different benchmark model signals are shown in Figure \ref{fig:searchI:sigfit}. The signal and background components are then simultaneously fitted to the data points.
\begin{figure}[h!]
\centering
\includegraphics[width=0.49\textwidth]{figures/analysis/search1/B2G-16-004/Figure_005-a.pdf}
\caption{Dijet invariant mass from signal MC used to extract the signal shape. Here for 1.2, 2, 3 and 4 TeV resonances.}
\label{fig:searchI:sigfit}
\end{figure}
\clearpage

\subsection{V-tagging scale factors}
\label{sec:searchI:vtag}

As seen in Figure~\label{fig:wtag}, some discrepancy is observed in the \nsubj distribution between data and MC. This can lead to a bias in the signal efficiency estimation and we therefore measure the real data signal efficiency in an orthogonal data sample.
The W-tagging efficiency is measured using real boosted \PW-jets in a semi-leptonic $\textrm{t}\bar{\textrm{t}}$ enriched data sample. This region is mainly quark-enriched, as opposed to the QCD gluon-enriched region we saw previously, and substructure variables are better described here. The sample is obtained through requiring a final state compatible with two b-jets and two \PW bosons, where one of the bosons decay leptonically and the other one hadronically. There are several good reasons to use this channel: Top quark pair production events are plentifully produced at the LHC, we can ensure a high purity of the sample through high-energy lepton, b-tag and missing energy requirements and lastly we can ensure that the \PW jets are boosted by requiring the leptonic leg, together with the hadronic \PW candidate, to have high transverse momentum. The final state is illustrated in Figure~\ref{fig:search2:ttsemilep}, with the object of interest being the AK R=0.8 jet containing the two quark daughters of the hadronically decaying \PW.
\begin{figure}[h!]
\centering
\includegraphics[width=0.49\textwidth]{figures/analysis/search2/misc/semileptt.pdf}
\caption{A top quark pair decaying into two b quarks and two \PW bosons, one of which decays leptonically and one on which decays hadronically}
\label{fig:search2:ttsemilep}
\end{figure}

\subsubsection{Event selection}
\label{sec:searchI:vtag:evsel}
The \PW can decay either to an electron or a muon, both final states ("channels") are used in the analysis. We select events through triggering and selections on the leptonic leg. First, we require a high-energy lepton at trigger level, with an online \PT above 45 \GeV for the muon and 135 \GeV for the electron. This requires an offline muon(electron) \PT threshold of 53(120) \GeV. The leptons are further required to pass the lepton requirements defined in Section~\ref{sec:objreco:muons} and Section~\ref{sec:objreco:electrons}, and events containing additional leptons (passing the same ID requirements, but looser cuts as defined in Table~\ref{tab:searchII:cutsummary}) are vetoed. Offline, we further require a high missing energy of 40(80) \GeV in the muon(electron) channel. To insure a high signal (boosted hadronic \PW) purity, the leptonic \PW four-vector is reconstructed such that we can put tight momentum requirements on the leptonic leg (ensuring that both tops, and therefore vector bosons, have a high momentum). The leptonic \PW is reconstructed in two steps: First, the unknown z component of the neutrino momentum must be solved for through a second order equation assuming the real \PW mass
\begin{equation*}
M_\mathrm{W}^2 = m_\ell^2   + 2(E_\ell E_\nu - p_{x_\ell}p_{x_\nu} - p_{y_\ell}p_{y_\nu} - p_{z_\ell}p_{z_\nu} ) = (80.4)^2.  
\end{equation*}
This results in a completely defined neutrino four-vector, which is then added to the lepton four-vector. The sum of the two defines the leptonic \PW and its momentum is required to be greater than 200 \GeV. \newline
Further, we require at least one AK R=0.4 jet to be b-tagged with the Combined Secondary Vertex (CSV) algorithm~\cite{1748-0221-8-04-P04013,1748-0221-13-05-P05011}. This algorithm exploits the relatively long lifetime of b quarks leading to the presence of a displaced vertex, in order to distinguish between jets originating from b quarks to those originating from light flavor quarks. More information on the CSV algorithm can be found in ~\cite{1748-0221-8-04-P04013,1748-0221-13-05-P05011}. The reason for requiring only one b-tagged jet is to ensure a high selection efficiency.\newline
Finally, we require at least one AK R=0.8 jet in the event with a momentum greater than 200 \GeV which will be the hadronic \PW candidate. It's pruned jet mass is required to be between 40 \GeV and 150 \GeV. After reconstructing and selecting all our objects, a set of angular selections are applied to ensure a diboson like topology. These are the following:
\begin{itemize}
\itemsep0em 
  \item $\Delta R(\l,W_{AK8}) > \pi/2$
  \item $\Delta \phi(W_{AK8},\ETmiss) > 2$
  \item $\Delta \phi(W_{AK8},W_{lep}) > 2$
\end{itemize}
With these requirements, we have a nearly pure sample of \ttbar events, with a small contamination from
single top, W+jets and \VV events.  A summary of the final selection criteria is presented in Table~\ref{tab:searchII:cutsummary}.The pruned jet mass and $\tau_{21}$ variables in data and in MC are shown in Figure~\ref{fig:searchI:ttbarcp}.
\begin{table}[h!]
\footnotesize
\centering
\begin{tabular}{lcc}
\hline 
\multicolumn{1}{c}{\textbf{Selection}} & \textbf{Value} & \textbf{Comments}\\
\hline
\multicolumn{1}{c}{\texttt{Tight} Lepton selection}\\
\cline{1-1}
Electron $\PT$ & $\PT > 120 \GeV$    & \\
Muon $\PT$ & $\PT > 53 \GeV$ & \\
Electron $\eta$ & $|\eta|_{\text{SC}} <2.5$ except [1.4442, 1.566] & Veto ECAL barrel-endcap transition.\\
Muon $\eta$  & $|\eta|<2.1$  & \\
\hline
\multicolumn{1}{c}{\texttt{Loose} Lepton selection}\\
\cline{1-1}
Electron $\PT$ & $\PT > 35 \GeV$    & \\
Muon $\PT$ & $\PT > 20 \GeV$ & \\
Electron $\eta$ & $|\eta|_{\text{SC}} <2.5$ except [1.4442, 1.566] & Veto ECAL barrel-endcap transition.\\
Muon $\eta$  & $|\eta|<2.4$  & \\
\hline
\multicolumn{1}{c}{AK8 jet selections}\\
\cline{1-1}
Jet $\PT$ &  $\PT >200~\GeV$ & For hadronic \\
Jet $\eta$  & $|\eta|<2.4$ & W reconstruction \\
\hline
\multicolumn{1}{c}{AK4 jet selections}\\
\cline{1-1}
Jet $\PT$ &  $\PT >30~\GeV$ & Used for b-tag \\
Jet $\eta$  & $|\eta|<2.4$ & jet selection\\
\hline
\multicolumn{1}{c}{\ETmiss selections}\\
\cline{1-1}
\ETmiss (electron channel) &  \ETmiss$>80~\GeV$ & \\
\ETmiss (muon channel) & \ETmiss$>40~\GeV$ & \\
\hline
\multicolumn{1}{c}{Boson selections}\\
\cline{1-1}
Pruned jet mass & $ 40 < m_{p} < 150 \GeV$ &  \\
Leptonic W $\PT$      &  $\PT > 200 \GeV$     & \\
Hadronic W $\PT$      &  $\PT > 200 \GeV$     & \\
\hline
\multicolumn{1}{c}{Veto}\\
\cline{1-1}
Number of \texttt{loose} electrons & 0    &  \\
Number of \texttt{loose} muons & 0    & \\
Number of b-tagged jets           & $>0$    & CSV medium working point \\
\hline
\multicolumn{1}{c}{Angular selections}\\
\cline{1-1}
$\Delta R(\l,W_{AK8})         $ & $> \pi/2$ & \\
$\Delta \phi(W_{AK8},\ETmiss) $ & $> 2$     & \\
$\Delta \phi(W_{AK8},W_{lep}) $ & $> 2$     & \\
\hline
\end{tabular}
\caption{Summary of the final semi-leptonic t$\bar{t}$ selections.}
\label{tab:searchII:cutsummary}
\end{table}

\begin{figure}[ht!]
\centering
\begin{tabular}{cc}
\includegraphics[width=0.4\textwidth]{figures/vtagging/AN-16-215/Whadr_pruned_mu.pdf}
\includegraphics[width=0.4\textwidth]{figures/vtagging/AN-16-215/Whadr_tau21_mu.pdf}\\
% \includegraphics[width=0.5\textwidth]{figures/vtagging/AN-16-215/Whadr_puppi_softdrop_mu.pdf}
% \includegraphics[width=0.5\textwidth]{figures/vtagging/AN-16-215/Whadr_puppi_tau2tau1_mu.pdf}\\
\end{tabular}
\caption{Distribution of pruned jet mass (left) and n-subjettiness (right) in the \ttbar control sample.} 
\label{fig:searchI:ttbarcp}
\end{figure}

\subsubsection{Fitting procedure}
For this measurement, what we are interested in is to extract and compare the W-tagging efficiency of the combined jet mass and \nsubj selection in data and in MC. We are additionally interested in the difference in jet mass scale (mean of the \PW jet mass peak) and jet mass resolution (width of \PW jet mass peak), as this also affects the signal jet mass shape and therefore efficiency. In order to study these variables, we look at the pruned jet mass spectrum between 40 and 150 \GeV in two regions: 
\begin{itemize}
\itemsep0em 
  \item Pass region: $0 <  \nsubj \leq 0.45 \sim$ high purity
  \item Fail region: $0.45 < \nsubj \leq 0.75\sim$ low purity
\end{itemize}
Our goal is to understand what the real fraction of merged \PW jets is in the pass category and in the fail category, assuming that the sum of the two correspond to a 100\% selection efficiency (the amount of \PW jets falling outside of this region is negligible).
The strategy is the following: We first derive probability density functions (PDFs) which describe the distribution of fully merged \PW jets and non-\PW jets in \ttbar, both in the pass and in the fail region. The PDFs describing real \PW jets and non-\PW jets are added with a fraction which is left floating: the fit decides what the fraction of real \PW to non-\PW jets is in the pass and in the fail region. As simultaneous fit of pass and fail is then performed (using the two composite \PW+non-\PW PDFs), where the fraction of real \PW jets in both pass and fail is constrained such that, if the signal efficiency in pass is $\epsilon_S$, the signal efficiency in fail is ($1-\epsilon_S$). This is done by letting the normalization of the PDF describing real \PW jets in the pass category, be defined as the \textit{total} real \PW yield in pass and fail combined multiplied by some fraction, $\epsilon_S$. The normalization of the PDF describing real \PW jets in the fail category is then the total real \PW yield multiplied by ($1-\epsilon_S$).\newline
To understand which part of the \ttbar jet mass distribution contains ``real'' merged Ws and which are only pure combinatorial background, non-PWs, we start from \ttbar MC.
By matching the AK8 jet with quarks coming from the hadronic W at generator level, in a cone of $\Delta R < 0.8$, we can access the real merged \PW and non-merged \PW shapes.
The real \PW and non-\PW PDFs for jets that pass and fail the N-subjettiness selection $\nsubj < 0.45$, are found to be well described by the following functions:
\begin{align*} 
f_{\rm bkg}(m_{j}) &= F_{\textrm{ExpErf}} = e^{c_0m_{j}} \cdot \frac{1 + {\rm Erf}((m_{j}-a)/b)}{2}  &\sim\textrm{for non-\PW jets in both pass and fail }\\
f^{\rm sig}(m_{j}) &= F_{\rm Gaus}(m_{j}) + F_{\rm ExpErf}(m_{j})                                    &\sim\textrm{for real \PW jets in both pass and fail}
\end{align*}
Figure~\ref{fig:searchI:ttfitpruned} shows the fitted PUPPI softdrop mass spectrum for \ttbar real \PW (top) and non-\PW (bottom) distributions for jets that passed (left) and failed (right column) the N-subjettiness selection PUPPI $\tau_{21}~<$~0.4.
The corresponding plots for the jet pruned mass can be found in Figure~\ref{app:sf16}.

\begin{figure}[h!]
  \centering
      \includegraphics[width=0.3\textwidth]{figures/vtagging/AN-16-215/plots_76X/fits_TTMC/plots_em_HP0v45powheg_76X_MCfits/{_TTbar_realWExoDiBosonAnalysis.WWTree_TTbar_powheg_76X_GausErfExp_ttbar_with_pull}.pdf}
      \includegraphics[width=0.3\textwidth]{figures/vtagging/AN-16-215/plots_76X/fits_TTMC/plots_em_HP0v45powheg_76X_MCfits/{_TTbar_realW_failtau2tau1cutExoDiBosonAnalysis.WWTree_TTbar_powheg_76X_GausErfExp_ttbar_failtau2tau1cut_with_pull}.pdf}\\
      \includegraphics[width=0.3\textwidth]{figures/vtagging/AN-16-215/plots_76X/fits_TTMC/plots_em_HP0v45powheg_76X_MCfits/{_TTbar_fakeWExoDiBosonAnalysis.WWTree_TTbar_powheg_76X_ErfExp_ttbar_with_pull}.pdf}
      \includegraphics[width=0.3\textwidth]{figures/vtagging/AN-16-215/plots_76X/fits_TTMC/plots_em_HP0v45powheg_76X_MCfits/{_TTbar_fakeW_failtau2tau1cutExoDiBosonAnalysis.WWTree_TTbar_powheg_76X_ErfExp_ttbar_failtau2tau1cut_with_pull}.pdf}
    \caption{Fit to the real \PW (top) and non-\PW (bottom) pruned jet mass distribution for jets that pass (left) and fail (right) the cut on $\tau_{21}~<$~0.45.}
  \label{fig:searchI:ttfitpruned}
\end{figure}
These shapes constitute the fit functions used for the simultaneous fit. As can be seen from the fit to real \PW jets in the pass region, the distribution is not purely Gaussian and have a tail at higher groomed masses. This tail depends on the matching requirements used to define real merged \PW jets and is unphysical. We therefore assume that the distribution of real W-jets can be described by a Gaussian only, allowing the exponential error function  used to describe non W-jets to cover the contribution from the tails, hereby taking the number of real W-jets as the integral of the Gaussian shape only. This eliminates two additional fit functions, corresponding to six free parameters from the fit. 
In older estimations of the W-tagging scale factor based on the same procedure~\cite{CMS-PAS-B2G-16-021}), the functions used to describe the tail of the real W-jet distributions were also taken into account as contributing to the real W-jet tagging efficiency. These two calculations tests two extremes:
The new method assumes a Gaussian peak, absorbing the tails into the background function making the fit more robust, while the old method assumes a Gaussian peak with tails estimated from matched MC. The latter uses a more precise definition of real \PW jets, but a less robust fit. Both methods were investigated and we found that the absorption of tails into the background function resulted in a decrease in the relative uncertainty on the final scale factor of 50 $\%$ and an overall improvement on the fit quality, reducing the fit $\chi^2$ by 15 $\%$. The fit parameters of the functions used to describe non W-jets in both the pass and in the fail region, are further constrained using the values obtained from matched \ttbar MC. The W-tagging scale factors ($SF_{HP}$), for the high purity selection ($\nsubj<0.45$), are then extracted estimating the cut efficiency ($\epsilon_{HP}$) on both data and simulated samples fitting, simultaneously, pass and fail samples:
\begin{align*}
\footnotesize
    L_{\rm pass} &= \prod_{i}^{N_{\rm evt}^{pass}} \bigg[N_{\rm W}\cdot\epsilon_{HP}\cdot f_{\rm pass}^{\rm sig}(m_{j}) + N_{\rm 2}\cdot f_{\rm pass}^{\rm bkg}(m_{j})+ \sum_{j=\textrm{ST,VV,WJet}} N^{j}_{\rm pass}\cdot f_{\rm pass}^{j}\bigg]\\
    L_{\rm fail} &= \prod_{i}^{N_{\rm evt}^{fail}} \bigg[N_{\rm W}\cdot(1-\epsilon_{HP})\cdot f^{\rm sig}_{\rm fail}(m_{j}) + N_{\rm 3}\cdot f_{\rm fail}^{\rm bkg}(m_{j})+ \sum_{j=\textrm{ST,VV,WJet}} N^{j}_{\rm fail}\cdot f_{\rm fail}^{j}\bigg]
\end{align*}
where $N_{W}$ is the number of real W jets, $N_{2}$ and $N_{3}$ are the number of combinatorial background events passing and failing the \nsubj cut respectively. $N_{j}$ and $f_{j}$, with $j$ = ST, VV, WJet, are the normalizations and shapes of the minor backgrounds (single top, VV, W+jets) which are fixed from simulation. The fit functions used are
\begin{alignat*}{3}
    f_{\rm pass}^{\rm sTop} &= F_{\rm ErfExpGaus}(x) = &&\frac{1 + {\rm Erf}((x-a)/b)}{2} \cdot e^{-(x-x_{0})^{2}/2\sigma^{2}}\\
    f_{\rm fail}^{\rm sTop} &= F_{\rm ExpGaus}(x)    = &&e^{ax} \cdot e^{-(x-b)^{2}/2s^{2}}\\
    f_{\rm pass}^{\rm VV}   &= F_{\rm ExpGaus}(x)    = &&e^{ax} \cdot e^{-(x-b)^{2}/2s^{2}}\\
    f_{\rm fail}^{\rm VV}   &= F_{\rm ExpGaus}(x)    = &&e^{ax} \cdot e^{-(x-b)^{2}/2s^{2}}\\
    f_{\rm pass}^{\rm wjet} &= F_{\rm ErfExp}(x)     = &&e^{c_0x} \cdot \frac{1 + {\rm Erf}((x-a)/b)}{2}\\
    f_{\rm fail}^{\rm wjet} &= F_{\rm ErfExp}(x)     = &&e^{c_0x} \cdot \frac{1 + {\rm Erf}((x-a)/b)}{2}
\end{alignat*}
with the corresponding distributions shown in Figure~\ref{fig:searchII:minorbkr}.
\begin{figure}[h!]
   \centering
    \includegraphics[width=0.30\textwidth]{figures/vtagging/AN-16-215/plots_76X/plots_0v45/plots_em_HP0v45powheg_76X_MCfits/{_STopExoDiBosonAnalysis.WWTree_STop_76X_ErfExpGaus_sp}.pdf}
    \includegraphics[width=0.30\textwidth]{figures/vtagging/AN-16-215/plots_76X/plots_0v45/plots_em_HP0v45powheg_76X_MCfits/{_WJets0ExoDiBosonAnalysis.WWTree_WJets_76X_ErfExp}.pdf}
    \includegraphics[width=0.30\textwidth]{figures/vtagging/AN-16-215/plots_76X/plots_0v45/plots_em_HP0v45powheg_76X_MCfits/{_VVExoDiBosonAnalysis.WWTree_VV_76X_ExpGaus}.pdf}\\
    \includegraphics[width=0.30\textwidth]{figures/vtagging/AN-16-215/plots_76X/plots_0v45/plots_em_HP0v45powheg_76X_MCfits/{_STop_failtau2tau1cutExoDiBosonAnalysis.WWTree_STop_76X_ExpGaus}.pdf}
    \includegraphics[width=0.30\textwidth]{figures/vtagging/AN-16-215/plots_76X/plots_0v45/plots_em_HP0v45powheg_76X_MCfits/{_WJets0_failtau2tau1cutExoDiBosonAnalysis.WWTree_WJets_76X_ErfExp}.pdf}
    \includegraphics[width=0.30\textwidth]{figures/vtagging/AN-16-215/plots_76X/plots_0v45/plots_em_HP0v45powheg_76X_MCfits/{_VV_failtau2tau1cutExoDiBosonAnalysis.WWTree_VV_76X_ExpGaus}.pdf}
  \caption{Fits to the pruned jet mass spectrum for the non-dominant backgrounds (Single top, W+jets and VV respectively) in the pass (top) and fail (bottom) regions.}
  \label{fig:searchII:minorbkr}
\end{figure} 
The floating parameters of the fit (besides the PDF shape parameters themselves) are the rates $N_{W}$, $N_{2}$ and $N_{3}$, and the mean and sigma of the W-mass distribution defined in
$f^{\rm sig}_{\rm pass}(m_{j})$ and $f^{\rm sig}_{\rm fail}(m_{j})$. The ratio between data and simulation efficiencies are then taken as the W-tagging scale factor:
\begin{equation}
  \label{SF}
  SF_{HP}= \frac{\epsilon_{HP}(\textrm{data})}{\epsilon_{HP}(\textrm{sim})}
\end{equation}
Considering that, both for data and simulation, $\epsilon_{HP}+\epsilon_{LP}+\epsilon_{fail} = 1$, the scale factor for low purity category can be defined as:
\begin{equation*}
  SF_{LP} = \frac{1-\epsilon_{HP}(\textrm{data})-\epsilon_{fail}(\textrm{data})}{1-\epsilon_{HP}(\textrm{sim})-\epsilon_{fail}(\textrm{sim})}
\end{equation*}
where $\epsilon_{fail}$ is the ratio between the number of events with $\tau_2/\tau_1 > 0.75$ and the total number of events. As mentioned previously, the number of real \PW jets with $\tau_2/\tau_1 > 0.75$  is negligible and the definition of the low purity scale factor simplifies to
\begin{equation}
  SF_{LP} = \frac{1-\epsilon_{HP}(\textrm{data})}{1-\epsilon_{HP}(\textrm{sim})}
\end{equation}

\subsubsection{Systematic uncertainties}
\label{sec:searchI:wtagsystematic}
As systematic uncertainties, we consider effects due to differences in \ttbar simulation as well as effects due to choice of fit method. The former is evaluated by comparing the extracted scale factor when using \ttbar MC samples produced with different matrix element (ME) and shower generators: \POWHEG (NLO) interfaced with \PYTHIA{8} , \MADGRAPH (LO) QCD interfaced with \HERWIG{++} and \POWHEG interfaced with \HERWIG{++}.
The uncertainty due to different ME generators (\POWHEG versus \MADGRAPH) correspond to 3(13)\% and are listed in Table~\ref{tab:searchI:WtagSFs} as the first quoted systematic uncertainty. The uncertainty due to parton showering (\PYTHIA{8} versus \HERWIG{++}) is 8.6\%, but are not relevant for analyses where no \HERWIG{++} based simulation is used, as is the case for the search presented in this chapter. 
For the systematic uncertainty accounting for effects due to choice of fit method, we compare the estimated extracted efficiency in \ttbar MC using the two different fit models described above: The new model, where the signal is modeled by a Gaussian peak and the tails of the distribution are absorbed in the background fit model, and the old model, including the tails when calculating the fraction of real \PW jets. Figure~\ref{fig:searchII:gausvstails} shows the fits obtained in the pass and fail regions using the two different models. With the new model only the Gaussian component of the fit contributes to the W-tagging efficiency while, with the old model, a Chebyshev component is additionally contributing to the total W-tagging efficiency.
\begin{figure}[ht!]
  \centering
    \includegraphics[width=0.33\textwidth]{figures/vtagging/AN-16-215/2Gauss.pdf}
    \includegraphics[width=0.33\textwidth]{figures/vtagging/AN-16-215/GausErfExpPass.pdf}\\
    \includegraphics[width=0.33\textwidth]{figures/vtagging/AN-16-215/GausChebysgev.pdf}
    \includegraphics[width=0.33\textwidth]{figures/vtagging/AN-16-215/GausErfExpFail.pdf}
  \caption{Fits obtained in the pass (top) and fail (bottom) regions using two different models: An alternative model with tails (top and bottom, left) where the tail component is contributing to the total W-tagging efficiency. When using the default model (top and bottom, right), only the Gaussian component of the fit contributes to the W-tagging efficiency.}
  \label{fig:searchII:gausvstails}
\end{figure} 
The estimated efficiencies obtained using both methods, after being corrected for the fraction of \PW jets in the tails, agree within 0.3(0.8)\% and are listed as systematic uncertainty in Table~\ref{tab:searchI:WtagSFs}.\newline
One additional uncertainty is added. As the W-tagging scale factor is evaluated in a \ttbar sample, the transverse momentum range is rather limited. When the \PW \PT reaches $\sim 400 \GeV$, the AK8 jet becomes a fully merged top jet with a mass of 170 \GeV and a scale factor measurement becomes impossible. However, the jets used in the analyses presented in this thesis have very high transverse momenta, up to 2-3 \TeV, and we therefore need an estimate of how the uncertainty on the W-tagging scalefactor changes as a function of \PT. This is estimated by comparing the difference in tagging between $\BulkG \rightarrow \PW \PW$ signal MC showered by \PYTHIA{}8 and \HERWIG{++} as a function of \PT, relative to the difference in tagging efficiency between the two at a $\PT \sim 200$~\GeV. This measurement was performed by a separate analysis team, and found to be $5.90\% \times \ln(\pt/200\GeV)$.

% \begin{figure}[h!]
%   \centering
%     \includegraphics[width=0.50\textwidth]{figures/vtagging/AN-16-215/wtag-ptdependence-tau21tight.pdf}
%   \caption{Uncertainty on the \pt dependence of the scale factor as a function of \pt, approximated with a logarithmic function.}
%   \label{fig:ptdependence}
% \end{figure}

Systematic uncertainties from other sources (lepton identification, b tagging etc.) are less than 0.5\% and therefore negligible.

\subsubsection{Results}
The simultaneous fit as described above is then performed both for data and for simulation, where we take the ratio of data and MC efficiencies as efficiency scale factors.
The corresponding fits are shown in Figure~\ref{fig:searchII:simfit}, with the corresponding extracted efficiencies and scale factors summarized in Table~\ref{tab:searchI:WtagSFs}.

\begin{table}[h!]
   \centering
   \footnotesize
   \begin{tabular}{| l | c | c | c | c |}
   \hline
   Category & Working point & Eff. data & Eff. simulation & Scale factor\\
   \hline
   HP&$\tau_2 / \tau_1 < 0.45$& $0.775 \pm 0.041 $& $0.822 \pm 0.033$ &$0.94 \pm 0.05~\rm{(stat)} \pm 0.03~\rm{(sys)} \pm 0.003~\rm{(sys)}$\\
   LP&$0.45 < \tau_2 / \tau_1 < 0.75$& $0.225 \pm 0.041 $& $0.178 \pm 0.033$ &$1.27 \pm 0.25~\rm{(stat)} \pm 0.13~\rm{(sys)} \pm 0.008~\rm{(sys)}$\\
   \hline
   \end{tabular}
   \caption{Efficiencies in data and in MC together with the corresponding W-tagging scale factors for the high purity and low purity categories. }
   \label{tab:searchI:WtagSFs}
\end{table}


\begin{figure}[h!]
\centering
\includegraphics[width=0.44\textwidth]{figures/vtagging/AN-16-215/_HP0v45powheg_76X_em_pTbin_200_5000.pdf}
\includegraphics[width=0.44\textwidth]{figures/vtagging/AN-16-215/_HP0v45powheg_76X_em_fail_pTbin_200_5000.pdf} \\
\caption{Pruned jet mass distribution that pass (left) and fail (right) the $\tau_2 / \tau_1 < 0.45$ selection. Results of both the fit to data (blue) and simulation(red) are shown. The background components of the fit are shown as short-dashed lines.}
\label{fig:searchII:simfit}
\end{figure}

We additionally extract the jet mass scale and jet mass resolution, used to scale and smear the jet mass signal shape in the limit setting procedure. These values are taken from the mean $\langle m \rangle$ and width $\sigma$ of the Gaussian
component of the simultaneous fit in the pass region and are summarized in Table~\ref{tab:searchI:params}. Both the jet mass scale as well as the jet mass resolution is larger in simulation than in data with a relative difference of 2 and 10\%, respectively.
Howver, the jet mass resolution scale factor has a large uncertainty attached to it and is statistically insignificant (in agreement with unity within uncertainty).

\begin{table}[!htb]
 \begin{center}

 \begin{tabular}{|l|c|c|c}
  Parameter & Data & Simulation & Data/Simulation \\
  \hline
  Pruning $\langle m \rangle$ &$80.9 \pm 0.6~{\rm \GeV}$   & $82.5 \pm 0.1~{\rm \GeV}$  & $0.980 \pm 0.007$ \\
  Pruning $\sigma$            & \ $6.7 \pm 0.7~{\rm \GeV}$ & \ $7.5 \pm 0.3~{\rm \GeV}$ & $0.89 \pm 0.10$ \\
  \hline
  % PUPPI softdrop $\langle m \rangle$ &$86.8 \pm 0.8~{\rm \GeV}$ & $87.9 \pm 0.2~{\rm \GeV}$ & $0.988 \pm 0.010$ \\
%   PUPPI softdrop $\sigma$ & \ $9.2 \pm 1.0~{\rm \GeV}$ & \ $8.7 \pm 0.4~{\rm \GeV}$ & $1.07 \pm 0.09$ \\
  % PUPPI softdrop $\langle m \rangle$ &$80.3 \pm 0.8~{\rm \GeV}$ & $81.9 \pm 0.01~{\rm \GeV}$ & $0.98 \pm 0.01$ \\%New mass corrections
  % PUPPI softdrop $\sigma$ & \ $9.0 \pm 0.9~{\rm \GeV}$ & \ $8.5 \pm 0.4~{\rm \GeV}$ & $1.07 \pm 0.12$ \\%New mass corrections
 \end{tabular}
 \caption{Jet mass scale and resolution in data and in simulation together with the relevant data-simulation scale factors.}
 \label{tab:searchI:params}
 \end{center}
\end{table}

\subsubsection{Impact on search variables}
\label{sec:searchI:wtagimpact}

The obtained W-tagging scale factors are used as a scale of the signal yield. As we require two W-tagged jets, either HPHP or HPLP, the actual scale factors for the high-purity signal yield is $\textrm{SF}_{HP}\times\textrm{SF}_{HP}$ and for the low-purity category $\textrm{SF}_{HP}\times\textrm{SF}_{LP}$. The signal yields are then
\begin{align*}
N_{S}^{HP} &= N_{\textrm{HP tot. yield}} \times \textrm{SF}_{HP} \times \textrm{SF}_{HP}\\
N_{S}^{LP} &= N_{\textrm{LP tot. yield}} \times \textrm{SF}_{HP} \times \textrm{SF}_{LP}\\
\end{align*}
The uncertainties on the scale factors are considered as anti-correlated between the HP and the LP categories.
The jet mass scale and resolution are used to scale and smear the signal Monte Carlo. An uncertainty on the signal yield based on the uncertainty on jet mass scale and resolution is also considered by scaling and smearing the jet mass up and down within the quoted uncertainties and then recomputing the signal efficiency. The results are listed in Table~\ref{tab:searchI:sys}.
% With a HP relative uncertainty of 6\% and LP relative uncertainty of 22\%, this becomes
% \begin{align*}
% \sigma_{rel}^{HPHP} &= 1.06 \times 1.06 = 1.12\\
% \sigma_{rel}^{HPLP} &= 1.28 \times 1.06 = 1.36\\
% \end{align*}
  
\subsection{Systematic uncertainties}
\label{sec:searchI:sys}

The uncertainty on the background parametrization is statistical only and is taken as the covariance matrix of the dijet fit function. As demonstrated in the F-test, we study different background parameterizations and we have found these to be within the fit uncertainty of the nominal fit. The remaining uncertainties concern the signal shape and yield and are listen in Table~\ref{tab:searchI:sys}. Jet reconstruction uncertainties affect both the signal yield and she signal shape. These are evaluated by rescaling the jet four-momenta according to uncertainties on the jet energy scale and resolution and recomputing the signal efficiency. The difference in efficiency with and without smearing/scaling is taken as systematic uncertainties, as described above.
The jet mass/energy scale and resolution also affect the signal shape, and are added as uncertainties in the peak position and width of the Gaussian component of the signal PDFs.

\begin{table}[h!]
  \centering
  \begin{tabular}{lccc}
    \hline
    Source                           & Relevant quantity    & HP uncertainty (\%)  & LP uncertainty (\%)\\
    \hline
    Jet energy scale                 & Resonance shape      & 2                    & 2 \\
    Jet energy resolution            & Resonance shape      & 10                   & 10 \\
    \hline
    Jet energy and \mJ{} scale       & Signal yield         & \multicolumn{2}{c}{0.1--4}\\ 
    Jet energy and \mJ{} resolution  & Signal yield         & \multicolumn{2}{c}{0.1--1.4}\\
    Pileup                           & Signal yield         & \multicolumn{2}{c}{2}\\
    Integrated luminosity            & Signal yield         & \multicolumn{2}{c}{2}\\
    PDFs (\PWpr)                     & Signal yield		      & \multicolumn{2}{c}{4--19}\\
    PDFs (\PZpr)                     & Signal yield		      & \multicolumn{2}{c}{4--13}\\
    PDFs (\BulkG)                    & Signal yield		      & \multicolumn{2}{c}{9--77}\\
    Scales (\PWpr)                   & Signal yield		      & \multicolumn{2}{c}{1--14}\\
    Scales (\PZpr)                   & Signal yield		      & \multicolumn{2}{c}{1--13}\\
    Scales (\BulkG)                  & Signal yield		      & \multicolumn{2}{c}{8--22}\\
    \hline
    Jet energy and \mJ{} scale       & Migration            & \multicolumn{2}{c}{1--50}\\
    V tagging \nsubj{}               & Migration            & 14                    & 21\\
    V tagging \pt-dependence         & Migration            & 7--14                & 5--11\\
    \hline
  \end{tabular}
  \caption{Summary of  systematic uncertainties and the quantities they affect. Migration uncertainties result in events switching between the purity/mass categories and changes the efficiency in each category, but do not affect the total signal efficiency.}
  \label{tab:searchI:sys}
\end{table}


\clearpage

\subsection{Results}
\label{sec:searchI:results}
The background fits for each analysis category in the data signal region are shown in Figure \ref{fig:search1:bkgfitMassCat}. Here a background only fit is performed while, as described above, a simultaneous fit is used for the limit setting procedure. The filled area correspond to the 1 sigma error band of the background fit, obtained using linear error propagation.

\begin{figure}[h!]
\centering
\includegraphics[width=0.44\textwidth]{figures/analysis/search1/AN-15-211/fits/MLfits/BkgFit_DijetMassHighPuriWW.pdf}
\includegraphics[width=0.44\textwidth]{figures/analysis/search1/AN-15-211/fits/MLfits/BkgFit_DijetMassLowPuriWW.pdf}\\
\includegraphics[width=0.44\textwidth]{figures/analysis/search1/AN-15-211/fits/MLfits/BkgFit_DijetMassHighPuriWZ.pdf}
\includegraphics[width=0.44\textwidth]{figures/analysis/search1/AN-15-211/fits/MLfits/BkgFit_DijetMassLowPuriWZ.pdf}\\
\includegraphics[width=0.44\textwidth]{figures/analysis/search1/AN-15-211/fits/MLfits/BkgFit_DijetMassHighPuriZZ.pdf}
\includegraphics[width=0.44\textwidth]{figures/analysis/search1/AN-15-211/fits/MLfits/BkgFit_DijetMassLowPuriZZ.pdf}\\
\caption{Fit to data in the signal region using the background fit only for the different mass and purity categories. The filled red area correspond to the 1 sigma statistical error of the fit.}
\label{fig:search1:bkgfitMassCat}
\end{figure}

We proceed by setting limits on the cross section of the process $\text{X} \to \VV$, using the asymptotic $\textrm{CL}_\textrm{S}$ method as described in Section~\ref{sec:theory:statmet}. The binned likelihood is defined as
\begin{equation}
L = \prod_i\frac{\mu^{n_i}_ie^{-\mu_i}}{n_i!}
\end{equation}
with
\begin{equation}
\mu_i=\sigma \cdot N_i(S)+N_i(B)
\end{equation}
Here $\sigma$ is the signal strength scaling the expected number of signal events in the $i$-th dijet invariant mass bin $N_i(S)$, $N_i(B)$ is the expected number of background events in dijet invariant mass bin $i$ and $n_i$ is the observed number of events in the $ith$ dijet invariant mass bin. The background per bin $N_i(B)$ is estimated from the background component of the best signal+background fit to the data points with the signal cross section set to zero. The number of signal events in the $i$-th dijet invariant mass bin, $N_i(S)$, is then estimated from the signal templates, where only a dijet invariant mass in a 20\% window around the resonance mass is considered, containing most of the signal contribution while making sure to keep a good description of the core.

\subsection{Limits: All-hadronic analysis}
\label{sec:searchI:results4q}
As mentioned in Section~\ref{sec:searchI:samples}, we set limits on three different signal scenarios: $\BulkG \rightarrow \WW$, $\BulkG \rightarrow \ZZ$ and $\PWpr \rightarrow WZ$, with a $\ktilde = 0.5$ for the \BulkG. Figure \ref{fig:searchI:Limits_CombNew} shows the asymptotic limits at 95 \% confidence level on signal cross section as a function of the resonance mass obtained with 2.7 \fbinv of 13 \TeV CMS data after combining all mass and purity categories (top). The corresponding p-values are shown in the bottom panel.

\begin{figure}[h!]
\centering
\includegraphics[width=0.32\textwidth]{figures/analysis/search1/AN-15-211/limits/brazilianFlag_BulkWW_new_combined_13TeV.pdf}
\includegraphics[width=0.32\textwidth]{figures/analysis/search1/AN-15-211/limits/brazilianFlag_WZ_new_combined_13TeV.pdf}
\includegraphics[width=0.32\textwidth]{figures/analysis/search1/AN-15-211/limits/brazilianFlag_BulkZZ_new_combined_13TeV.pdf}\\
\includegraphics[width=0.32\textwidth]{figures/analysis/search1/AN-15-211/pvalues/pvalue_BulkWWin_combined_new.pdf}
\includegraphics[width=0.32\textwidth]{figures/analysis/search1/AN-15-211/pvalues/pvalue_WZin_combined_new.pdf}
\includegraphics[width=0.32\textwidth]{figures/analysis/search1/AN-15-211/pvalues/pvalue_BulkZZin_combined_new.pdf}\\
\caption{Expected and observed limits with corresponding p-values obtained using 2.6 $\textrm{fb}^{-1}$ of CMS data after combining all mass and purity categories. Here for a Bulk $G\rightarrow WW$ (left), $W'\rightarrow WZ$ (middle) and $G\rightarrow ZZ$ (right) signal.}
\label{fig:searchI:Limits_CombNew}
\end{figure}


The statistics are too low to exclude the excess around 2 \TeV observed in the corresponding Run 1 analysis and in addition an under-fluctuation in data is present in this region. The largest excess is observed for a $\BulkG \rightarrow \ZZ$ hypothesis at a resonance mass of 2.8-3 TeV, around 2.3 $\sigma$.
This is driven by the ZZ high-purity category, the category with the lowest statistics, where one event at 3 TeV yields a local significance of 2.8 $\sigma$. A 3 parameter fit is the default background fit function for this category, however, a 2 parameter fit could also be used to describe these data. In Figure \ref{fig:searchI:Limits_ZZHP} we compare the limits and p-values obtained using a 2 parameter and a 3 parameter fit to describe the background in this category. The significance at 3 TeV is reduced from 2.8 to 1.5 $\sigma$ with a 2 parameter fit, reflecting the fact that the fit is poorly constrained in the high mass tail due to low statistics. The fit to data using both a 2 and 3 parameter fit in the ZZHP category is shown in Figure~\ref{fig:app:ZZHP2vs3p} and we in addition see that the 2 parameter fit lies within the fit uncertainties of the nominal fit.\newline
\newline

\begin{figure}[h!]
\centering
\includegraphics[width=0.49\textwidth]{figures/analysis/search1/AN-15-211/limits/brazilianFlag_BulkZZ_ZZHP_13TeV.pdf}
\includegraphics[width=0.49\textwidth]{figures/analysis/search1/AN-15-211/limits/brazilianFlag_BulkZZ_ZZHP_2parFit__13TeV.pdf}\\
\includegraphics[width=0.49\textwidth]{figures/analysis/search1/AN-15-211/pvalues/pvalue_BulkZZinZZ_high_purity.pdf}
\includegraphics[width=0.49\textwidth]{figures/analysis/search1/AN-15-211/pvalues/pvalue_BulkZZinZZ_high_purity_2par.pdf}
\caption{Expected/observed limits and corresponding p-values obtained in the ZZHP category using a 3 (left) and two (right) parameter fit to describe the background. The significance at 3 TeV is reduced from 2.8 to 1.5 $\sigma$.}
\label{fig:searchI:Limits_ZZHP}
\end{figure}

\begin{figure}[h!]
\centering
\includegraphics[width=0.49\textwidth]{figures/analysis/search1/misc/CMS-PAS-EXO-15-002_Figure_004-e.pdf}
\caption{Background fit to data in the ZZHP category using the default 3 (red) and an alternate 2 (blue) parameter fit to describe the background.}
\label{fig:app:ZZHP2vs3p}
\end{figure}


The lack of constraint on the fit in the dijet invariant mass tail when statistics are very low, is a drawback of a method relying fully on a parametric fit and reduces the analysis sensitivity in the high-\mjj region. In Search II (Section~\ref{searchII}) we will keep taking advantage of the dijet fit, however, the integrated luminosity is $\sim 15$ times higher, resulting in more datapoints in the \mjj tail which further constrains the fit. In Search III (Section~\ref{searchIII}), we will explore alternate methods which allow more control over the background shape across the full mass spectrum. 

\subsection{Limits: Semi-leptonic and all-hadronic combination}
\label{sec:searchI:resultsComb}

To maximize the search sensitivity, we combine the results obtained above with those of the corresponding semi-leptonic analysis. We assume the uncertainties on luminosity, V-tagging efficiency, jet mass scale and resolution to be fully correlated.

The obtained exclusion limits are shown in Figure~\ref{fig:searchI:limitCombined} shows the resulting expected and observed
exclusion limits. As before, we consider a scenario where only either a \PWpr or \PZpr resonance is expected, called the singlet hypothesis (upper two plots). In addition, we set limits on the triplet hypothesis, assuming the \PWpr and \PZpr bosons to be degenerate in mass (bottom left plot).
Due to larger branching fracion, the all-hadronic analysis sets stronger upper limits than the semi-leptonic analysis above 1.7\TeV for \PZpr and $>$ 1.3\TeV for \PWpr ($\cal B$($\PW\PW\to\qqbar\qqbar$) = 44\%, $\cal B$($\PW\PW\to\ell\Pgn\qqbar$) = 31\%, $\cal B$($\PW\PZ\to\qqbar\qqbar$) = 46\%, and $\cal B$($\PW\PZ\to\ell\Pgn\qqbar$) = 16\%). 
The analysis sensitivity for \BulkG is too weak to set limits, but cross sections between 3--1200\unit{fb} are excluded.
For the HVT model A and B, \PWpr is excluded below $< 2.0$ and $2.2 \TeV$, respectively. \PZpr resonances are excluded below $< 1.6~(1.7)\TeV$ for HVT model B(A). If assuming a HVT Model A(B) triplet hypothesis, resonances below $< 2.3$($< 2.4$) \TeV are excluded.


\begin{figure}[!htb]
\centering
     \includegraphics[width=0.49\textwidth]{figures/analysis/search1/B2G-16-004//EXOVVhvt_JJLVJZPRIME13_UL_Asymptotic_log.pdf}%
     \includegraphics[width=0.49\textwidth]{figures/analysis/search1/B2G-16-004//EXOVVhvt_JJLVJWPRIME13_UL_Asymptotic_log.pdf}\\     
     \includegraphics[width=0.49\textwidth]{figures/analysis/search1/B2G-16-004//EXOVVhvt_JJLVJHVT13_UL_Asymptotic_log.pdf}%
     \includegraphics[width=0.49\textwidth]{figures/analysis/search1/B2G-16-004//EXOVVbulkg_ALL13_UL_Asymptotic_log.pdf}
\caption{Observed (black solid) and expected (black dashed) 95\% CL upper limits on the production of a narrow-width resonance decaying to 
a pair of vector bosons for different signal hypotheses. In the upper plots, limits are set in the context of a spin-1 neutral \PZpr (left) and charged \PWpr (right)
resonances, and compared with the prediction of the HVT Models A and B. In the lower left plot, limits are set in the same model under the triplet hypothesis (\PWpr and \PZpr).
In the lower right plot, limits are set in the context of a bulk graviton with $\ktilde =0.5$ and compared with the prediction.
}
\label{fig:searchI:limitCombined}
\end{figure}

\par
The combined results would therefore just exclude a $\PWpr$ with a mass around 2 \TeV, the favored candidate to explain the 8 \TeV diboson excess.
However, Bulk Graviton signals were still far from excluded and, with the expected ten times increase in luminosity in 2016, we were excited to keep on searching.

% Figure~\ref{fig:hvtscan} shows a scan of the coupling parameters and the corresponding observed 95\% CL exclusion
% contours in the HVT model for the combined analyses. The parameters
% are defined as $g_{\rm V}c_{\rm H}$ and $g^2c_{\rm F}/g_{\rm V}$,  related to the coupling strengths of the new resonance to
% the Higgs boson and to fermions. The range of the scan is limited by the assumption that the
% new resonance is narrow.  A contour is overlaid, representing the region where the theoretical
% width is larger than the experimental resolution of the searches, and hence where the narrow-resonance assumption is not satisfied.
% This contour is defined by a predicted resonance width of 5\%, corresponding to the narrowest resonance mass resolution of the searches.

% \begin{figure}[!htb]
% \centering
%      \includegraphics[width=0.49\textwidth]{figures/analysis/search1/B2G-16-004/hvt-couplings.pdf}
% \caption{
% Exclusion regions in the plane of the HVT couplings ($g^2c_{\rm F}/g_{\rm V},g_{\rm V}c_{\rm H}$) for three
% resonance masses, 1.5, 2.0, and 3.5\TeV. Model points A and B of the benchmarks used in the analysis are also shown.
% The solid, dashed, and dashed-dotted lines represent the boundaries of the regions excluded by this search for different resonance masses (the region outside these lines is excluded).
% The areas indicated by the solid shading correspond to
% regions where the resonance width is predicted to be more than 5\% of the resonance mass and
% the narrow-resonance assumption is not satisfied.}
% \label{fig:hvtscan}
% \end{figure}










\clearpage

\begin{singlespace}
\setstretch{1.25}
\vspace*{\fill}
\begin{centering}
\chapter{Search II: A new pileup resistant and perturbative safe tagger}
\label{searchII}
\textit{
\noindent With the first 13 \TeV diboson resonance search published, we could conclude that more data would be needed in order to fully exclude the observed Run 1 excess. Luckily, 2016 was right around the corner and, with the LHC planning to reduce $\beta^*$ from 80 \cm to 40, the machine was expected to deliver an instantaneous luminosity three times that of the 2015 peak luminosity. Higher instantaneous luminosity, however, meant double the pileup.
\newline
\newline
We knew that a novel pileup subtraction algorithm had been developed, which provided far better pileup and underlying event rejection than the current default (CHS). We also knew that there had been made progress on the theory side in the development of a groomer which was insensitive to the soft divergences of QCD and allowed to accomplish jet grooming in a theoretically calculable way, SoftDrop (mMDT). With more time at hand than in 2015, I therefore decided to pursue a novel W-tagger for this second search. This included work like optimization, development of dedicated jet mass corrections (in use today and recommended by the jet physics object group) as well as validation of the new tagger. The tagger, together with the mass corrections, afterwards became the default W-tagging algorithm in CMS.
\newline
\newline
Search II became the first published analysis to use the novel PUPPI softdrop grooming algorithm, now default for W-tagging in CMS. Through this search, the tagger was optimized, commissioned and validated, making it available for several analysis to come. In addition, the search was extended to setting limits on three additional signal hypothesis. Two of these were in a final state never before explored at 13 \TeV, the $\textrm{q}^* \rightarrow \textrm{qV}$ single V-tag analyses. Published with 35.9(12.9) \fbinv of 2016 data.
}
\end{centering}
\begin{figure}[b!]
    \centering
    \includegraphics[height=6.5cm]{figures/vtagging/JME-16-003/BoostedW/WtagSigEffvsNPV.pdf}
    \vspace*{10mm}
    \caption*{\footnotesize{\textit{Published in PRD, DOI: 10.1103/PhysRevD.97.072006; CMS-PAS-B2G-16-021; CMS-PAS-JME-16-003}}}
\end{figure}
\vspace*{\fill}
\end{singlespace}

\clearpage
\subsection{Towards robust boosted jet tagging}
\label{sec:searchII:intro}
When we first studied W-tagging at 13 \TeV in context with the analysis of the 2015 dataset, Section~\ref{sec:searchI:wtagging}, two interesting correlations were observed:

\bigskip
1) A strong dependence of the AK8 CHS softdrop ($\beta = 0$) jet mass on jet \PT and

2) a strong dependence of the AK8 CHS $\tau_{21}$ cut efficiency on pileup.
\bigskip

The reason we studied the softdrop algorithm as an alternative to pruning in 2015 was, besides the possibility it would result in a higher signal efficiency, that we knew it had certain favorable qualities compared to other groomers: Softdrop removes all sensitivity to the soft divergences of QCD, by removing all soft emission, more specifically the non-global logarithmic terms (NGLs) in the jet mass~\cite{Dasgupta:2013ihk}. These arise from constellations where, for instance, a soft gluon is radiated into the jet cone, as illustrated in Figure~\ref{fig:searchII:ngls}. 
\begin{figure}[h!]
\centering
\includegraphics[width=0.69\textwidth]{figures/analysis/search2/misc/ngls.pdf}
\caption{The pruning algorithm does not remove all soft emission and therefore has non-global logarithmic terms in the jet mass. Softdrop ($\beta = 0$) completely removes soft emissions and is therefore free of non-global logarithms.}
\label{fig:searchII:ngls}
\end{figure}
The consequence of this is that you can calculate the softdrop jet mass to way higher precision than what is possible for other grooming algorithms or for the plain jet mass (NGLs are the main reason a full resummation of the plain jet mass beyond NLL (considering e.g multiple-emission effects) accuracy does not exist). Despite this not being a precision measurement analysis, we had theoretically well-motivated reasons for wanting the baseline CMS V-tagger to be softdrop-based. However, despite being less sensitive to soft radiation for QCD jets, signal jets groomed with softdrop were found to be far more sensitive to the underlying event than pruned jets~\cite{Dasgupta:2015yua}. Figure~\ref{fig:searchII:ue} shows the signal efficiency for pruning (left) and softdrop (right) as a function of jet transverse momenta when including FSR only, FSR+ISR, hadronization and hadronization + underlying event.
\begin{figure}[h!]
\centering
\includegraphics[width=0.79\textwidth]{figures/analysis/search2/misc/pruningvssd_ue.pdf}
\caption{The signal efficiency for pruning (left) and softdrop (right) as a function of jet \PT when adding FSR, ISR, hadronization and UE. THe UE has a severe impact on the softdrop efficiency for signal jets~\cite{Dasgupta:2015yua}. }
\label{fig:searchII:ue}
\end{figure}
On parton level, as well as after hadronization, the two algorithms perform very similar as a function of \PT. However, once UE contamination is added, the softdrop tagging efficiency is severely affected. This can be explained by the larger effective radius considered by the softdrop algorithm ( $\propto \mV/\PT \sqrt{z_{cut}(1-z_{cut})}$ ) in comparison to pruning ( $\propto \mV/\PT$ ). This observation corresponds very well with the shift in jet mass we observed for softdrop as a function of \PT in Section~\ref{sec:searchI:wtagging}: As the jet \PT decreases the softdrop effective radius increases and the jet mass mean shifts to higher values, due to absorbing more background radiation. If softdrop would be our new default tagger, a better rejection of pileup and UE contamination would be needed. In parallel to the ongoing theoretical work on groomers, a novel pileup removal algorithm had been proposed: Pileup per particle identification (PUPPI)~\cite{Bertolini2014}. Described in detail in Section~\ref{subsub:objreco:puppi}, PUPPI considers not only charged pileup but rather reweights each particle in the jet with its probability of arising from pileup. PUPPI had proven it self far superior to the current CHS algorithm in terms of jet observables for large radius jets, and therefore seemed like the obvious choice to address both issues listed above: The sensitivity of softdrop regarding UE contamination and the strong pileup dependence of $\tau_{21}$. The focus of Search II would therefore be on the commissioning of a novel W-tagger. There are interesting changes and inclusions in the analysis strategy as well: The inclusion of a $\PZpr \rightarrow \WW$ signal hypothesis and the addition of a completely new analysis, the single V-tag analysis.

\subsection{Analysis strategy}
The analysis strategy for this search is conceptually the same as for Search I. In addition, we'll take advantage of the n-subjettiness categorization and do an additional analysis in parallel: A search for excited quark resonances $\rm{q^*}$~\cite{Bauer1987,PhysRevD.42.815} decaying to qW or qZ.
We call this the single V-tag analysis, and the analysis selection only differs in that one jet is not required to pass the V-tag selection (groomed mass and n-subjettiness). The \VV analysis is hereby referred to as the double V-tag analysis. The difference between the two analyses is illustrated in Figure~\ref{fig:searchII:svsd}. 
\begin{figure}[h!]
\centering
\includegraphics[height=6.5cm]{figures/analysis/search2/misc/singlevsdoubletag.pdf}
\caption{The double (top) and single (bottom) W/Z-tag analysis.}
\label{fig:searchII:svsd}
\end{figure}
In addition, limits are set on a $\PZpr \rightarrow \WW$ signal hypothesis in the double V-tag analysis, another 13 \TeV first.\newline
This analysis was published in two steps: An early Physics Analysis Summary (PAS) based on 12.9 \fbinv of 2016 data~\cite{CMS-PAS-B2G-16-021}, describing the new  PUPPI+softdrop based V-tagger as well as the single V-tag analysis, and a second analysis topping up with the full 2016 data~\cite{PhysRevD.97.072006}. The commissioning of the new \PW\PZ-tagger has also been documented in a jet performance Physics Analysis Summary~\cite{CMS-PAS-JME-16-003}. As the new V-tagger was developed and commissioned in the context of the early analysis, which was also were the single V-tag analysis was first published with 13 \TeV data, the main emphasis will be on the work presented in CMS-PAS-B2G-16-021~\cite{CMS-PAS-B2G-16-021}. The second part of the results chapter, Section~\ref{sec:searchII:brg17001res}, includes the results obtained using the full 2016 dataset of 35.9 \fbinv.

\subsection{Data and simulated samples}
\label{sec:searchII:samples}
As mentioned above, the analysis of the 2016 dataset was done in two steps: One analysis based on 12.9 \fbinv of early 2016 data, describing the new W-tagger and single V-tag category, and a second paper topping up with the full 2016 dataset, corresponding to 35.9 \fbinv.\par
The \BulkG and HVT signal samples are modeled in precisely the same way as in 2015. For the single V-tag $\textrm{q}^*$ samples, we simulate unpolarized boson with a compositeness scale $\Lambda$ set equal to the resonance mass. These are generated to leading order using \PYTHIA version 8.212~\cite{Sjostrand:2007gs}. \par
The background Standard Model processes; QCD, W+jets and Z+jets are all simulated to leading order. V+jets is simulated with \amcatnlo~\cite{Alwall:2014hca,Alwall:2007fs}, while three different combinations of matrix element and shower generators is used for QCD as these predictions are known to differ: \PYTHIA only, the leading order mode of \amcatnlo{} matched with \PYTHIA, and \HERWIG{++}~2.7.1~\cite{Bahr:2008pv} with tune CUETHS1~\cite{Khachatryan:2015pea}.

\subsection{Event selection}
\subsubsection{Triggering}
The triggers used in this analysis are the same ones as in 2015 (see Section~\ref{sec:searchI:trigger}), however, due to the new single V-tag analysis, the trigger turn-ons have this time been re-evaluated separately requiring either one or two jets to have an offline softdrop jet mass above 65 \GeV.
\par Figure~\ref{fig:searchII:trigger-fits} shows the trigger turn-on curves as a function of dijet invariant mass for jets passing one of the three inclusive triggers only, one of the grooming triggers only and when combining all of them. The turn-on curves are shown for all jet pairs passing loose selections as described in Section \ref{sec:searchI:preselection}. Zero, one or two of the two jets is further required to have a softdrop mass larger than 65 GeV.

\begin{figure}[h!]
\centering
% \includegraphics[width=0.4\textwidth]{figures/analysis/search2/AN-16-398/plots/trigger/triggereffMjj-ALL_SingleTag_runAll.pdf}
% \includegraphics[width=0.4\textwidth]{figures/analysis/search2/AN-16-398/plots/trigger/triggereffMjj-ALL_DoubleTag_runAll.pdf}
% \includegraphics[width=0.4\textwidth]{figures/analysis/search2/AN-16-398/plots/trigger/triggereffMjj-ALL_noTag_runAll.pdf}
\includegraphics[width=0.49\textwidth]{figures/analysis/search2/AN-16-235/plots/triggereffMjj-ALL_SingleTag.pdf}
\includegraphics[width=0.49\textwidth]{figures/analysis/search2/AN-16-235/plots/triggereffMjj-ALL_DoubleTag.pdf}\\
\includegraphics[width=0.49\textwidth]{figures/analysis/search2/AN-16-235/plots/triggereffMjj-ALL_noTag.pdf}
\caption{Comparison of trigger efficiencies for jets passing one of the HT-triggers only (pink), for jets passing one of the grooming-triggers only (green) and for jets passing one of the HT-triggers or one of the grooming triggers (purple). Here as a function of dijet invariant mass for all jet pairs passing loose selections and where one jet has a softdrop mass larger than 65 GeV (top left), both jets have a softdrop mass larger than 65 GeV (top right) and where no mass cut is applied (bottom). }
\label{fig:searchII:trigger-fits}
\end{figure}
Including grooming triggers lowers the 99\% trigger efficiency threshold by around 50(80) \GeV in the single (double) tag category once substructure is requested on the analysis level. Using the or of all triggers, we are safely on the trigger plateau for dijet invariant masses above 955(986) \GeV in the double (single) tag category, setting the analysis threshold at a dijet invariant mass of 955 \GeV for the double tag analysis and 990 \GeV for the single tag analysis. For controlplots, where no groomed mass window is applied, a trigger threshold of 1020 GeV is used.
\par Trigger efficiencies as a function of the offline softdrop-jet mass for the \\ 
\texttt{HLT\_AK8PFJet360\_TrimMass30} trigger are shown in Figure~\ref{fig:searchII:grooming-mj-trigger}. Here the jet transverse momentum of one of the jets is required to be at least 600 GeV and no other mass cut is applied. This trigger requires one jet to have a trimmed mass above 30 GeV at HLT level and reaches the trigger plateau for groomed-jet masses around 50 GeV. As reference trigger, the prescaled trigger \texttt{HLT\_PFJet320} is used. 
\begin{figure}[htb]
\centering
\includegraphics[width=0.49\textwidth]{figures/analysis/search2/AN-16-235/plots/triggereff-prunedmass_fit.pdf}
\includegraphics[width=0.49\textwidth]{figures/analysis/search2/AN-16-235/plots/triggereff-sdmass_fit.pdf}
\caption{Efficiency for the \texttt{HLT\_AK8PFJet360\_TrimMass30} trigger as a function of pruned-jet (left) and softdrop-jet (right) mass for jets with $\PT > \unit{600}{\GeV}$.}
\label{fig:searchII:grooming-mj-trigger}
\end{figure}

\subsubsection{Preselection}
\label{sec:searchII:presel}
The same preselections as in Search I, described in \label{sec:search1:preselection}, have been applied: We require two AK R=0.8 jets with CHS applied pre-clustering, required to pass the tight jet ID requirement, $\PT>200 \GeV$ and $|\eta|<2.5$. The same QCD t-channel suppressing cut of $|\Delta \eta|<1.3$ is required together with the following trigger thresholds on the dijet invariant mass: $\mjj > \unit{955} {\GeV}$ for the double V-tag and $\unit{990} {\GeV}$ for the single V-tag analysis. The jet \PT (top left), $\eta$ (top right), $\Delta \eta_{jj}$ and dijet invariant mass (bottom left) for the two leading jets in the event after loose preselections are applied is shown in Figure~\ref{fig:searchII:kinematics-all}.
\begin{figure}[h!]
\centering
\includegraphics[width=0.49\textwidth]{figures/analysis/search2/AN-16-235/plots/qcdcp_Pt.pdf}
\includegraphics[width=0.49\textwidth]{figures/analysis/search2/AN-16-235/plots/qcdcp_Eta.pdf}\\
\includegraphics[width=0.49\textwidth]{figures/analysis/search2/AN-16-235/plots/qcdcp_DeltaEta.pdf}
\includegraphics[width=0.49\textwidth]{figures/analysis/search2/AN-16-235/plots/qcdcp_Mjj.pdf}
\caption{Jet \PT{} (top left), $\eta$ (top right), $\Delta \eta_{jj}$ and dijet invariant mass (bottom left) for the two leading jets in the event after loose preselections are applied. The signal is scaled by an arbitrary number.}
\label{fig:searchII:kinematics-all}
\end{figure}
A large difference in slope in the jet \PT and dijet invariant mass spectrum depending on the QCD matrix element or shower generator is observed. Pure \PYTHIA QCD MC describes the data best, while \HERWIG{++} and \amcatnlo{}+\PYTHIA tend to under- or over-estimate the number of high $\PT/\mjj$ jets, respectively. Pure \PYTHIA QCD MC is therefore used for all background checks in this analysis.


\subsection{Developing a new W-tagger}
\label{sec:searchII:puppisoftdrop}
As mentioned in the introduction to this chapter, early studies had shown that the PUPPI pileup subtraction algorithm yielded superior resolution on large-cone jet observables like the jet mass. We therefore wanted to check whether the softdrop jet mass, and its observed sensitivity to the Underlying Event and pileup, would be improved if a better pileup subtraction algorithm was applied pre-clustering.\par
Two interesting observations were made. Softdrop used together with PUPPI pileup subtraction displayed a much smaller \PT-dependent shift than CHS+Softdrop, as hoped. Figure~\ref{fig:searchII:sdmass} shows the PUPPI softdrop mass for W-jets from a 1 \TeV ($\PT\sim 500 \GeV$) and 4 \TeV ($\PT\sim 2 \TeV$) resonance, exhibiting the desired reduced \PT dependence in jet mass scale. 
\begin{figure}[htb]
\centering
\includegraphics[width=0.49\textwidth]{figures/analysis/search2/AN-16-235/plots/gen_SoftdropMassUnCorr.pdf}
\caption{The  PUPPI softdrop jet mass distribution with no jet energy corrections applied}
\label{fig:searchII:sdmass}
\end{figure}
However, when applying centrally provided L2 and L3 jet energy corrections (see Section~\ref{sec:objreco:jec}) to the jet groomed mass, as is recommended, a strong \PT dependence is re-introduced. This effect is not present for the pruned jet mass. Figure~\ref{fig:searchII:wtagmass} show the softdrop (top left) and pruned (top right) jet mass distribution with recommended L2L3 corrections applied. Here, the PUPPI+softdrop jet mass shift is significantly increased with respect to what was observed for the uncorrected mass, while CHS+pruned jet mass is stable. This points to the PUPPI jet energy corrections not being optimal for scalar jet mass variables, while they may be good for correcting jet 4-vectors. The jet energy corrections derived for CHS and PUPPI jets as a function of jet \PT is shown in the bottom plot in Figure~\ref{fig:searchII:wtagmass} . A significant slope in JEC as a function of \PT is measured for PUPPI, while not present for CHS.
\begin{figure}[htb]
\centering
\includegraphics[width=0.49\textwidth]{figures/analysis/search2/AN-16-235/plots/gen_SoftdropMass.pdf}
\includegraphics[width=0.49\textwidth]{figures/analysis/search2/AN-16-235/plots/gen_PrunedMass.pdf}\\
\includegraphics[width=0.49\textwidth]{figures/analysis/search2/AN-16-235/plots/JECvsPT.pdf}
\caption{Top: PUPPI softdrop mass distribution (top left) and pruned jet mass distribution (top right) with L2 and L3 corrections applied. Bottom: The projection of CHS and PUPPI jet energy corrections versus jet \PT.}
\label{fig:searchII:wtagmass}
\end{figure}

\subsubsection{Dedicated PUPPI softdrop mass corrections}
\label{sec:searchII:masscorr}
In order to minimize \PT dependence in the PUPPI softdrop jet mass, all jet energy corrections to the softdrop jet mass are removed. However, this still leaves a residual \PT dependence and, in addition, the uncorrected mass does not peak at the correct W-mass of 80.4~\GeV. Figure~\ref{fig:searchII:UncorrSD} shows the mean of a Gaussian fit to the uncorrected PUPPI softdrop mass as a function of jet $\pt$ in two different $\eta$ bins (smaller or greater than $|\eta|=1.3$) for W-jets coming from a Bulk Graviton signal sample. A mass shift both as a function of $\eta$ and \PT is observed, together with an average mean significantly lower than the W-mass.
\begin{figure}[htbp]
\centering
\includegraphics[width=0.49\textwidth]{figures/analysis/search2/AN-16-235/plots/RecoPuppiSoftdropMass_vspt.pdf}
\caption{The mean of a Gaussian fit to the W-jet PUPPI softdrop mass peak as a function of jet \PT in two different $\eta$ bins (smaller or greater than $|\eta|=1.3$). No corrections have been applied to the softdrop mass.}
\label{fig:searchII:UncorrSD}
\end{figure}
In order to use PUPPI+softdrop for W-tagging, we therefore derive dedicated jet mass corrections to compensate for two factors: A generator level \PT-dependence, as first observed in \label{sec:searchI:wtagging}, and a reconstruction level \PT- and $\eta$-dependence, most likely caused by UE effects and the growing effective sofdrop radius at low jet \PT. Figure~\ref{fig:searchII:sdmassshifts} shows the mean of the generated softdrop mass (left) and the normalized difference in reconstructed and generated softdrop mass (right) as a function of jet \PT. The shift in generated softdrop mass at lower \PT is of the order of 2-3$\%$ while the difference between reconstructed and generated softdrop mass is a 5-10$\%$ effect.
\begin{figure}[htbp]
\centering
\includegraphics[width=0.49\textwidth]{figures/analysis/search2/AN-16-235/plots/GenSoftdropMass_vspt.pdf}
\includegraphics[width=0.49\textwidth]{figures/analysis/search2/AN-16-235/plots/MassShift_vspt.pdf}
\caption{The mean of the fitted generator level W-jet softdrop mass distribution as a function of jet $\pt$ (left) and the normalized difference in reconstructed and generated softdrop mass (right).}
\label{fig:searchII:sdmassshifts}
\end{figure}
The mass shift introduced at generator level is corrected by a fit to $\rm{M_{PDG}/M_{GEN}}$ as a function of jet \PT, where $\rm{M_{PDG}}=80.4~\GeV$ and $\rm{M_{GEN}}$ is the fitted mean of the generator level mass as shown in the left plot in Figure~\ref{fig:searchII:sdmassshifts}. To correct for the residual shift between generated and reconstructed softdrop mass, a fit to $\rm{(M_{RECO}-M_{GEN})/M_{RECO}}$, where $\rm{M_{RECO}}$ is the reconstructed mass shown in the right plot in Figure~\ref{fig:searchII:sdmassshifts} and $\rm{M_{GEN}}$ is as defined above, as a function of jet \PT in two $\eta$ bins (smaller or greater than $|\eta|=1.3$) is performed.
Polynomial fit functions of the following forms are used
\begin{align*} 
% w(\pt) &=  [0]+[1]*pow(x*[2],-[3] \\
w(\pt) &=  A  +B(x^{2})^{-C}          &\sim\textrm{``gen correction''}\\
w(\pt) &=  A  +Bx+Cx^2+Dx^3+Ex^4+Fx^5 &\sim\textrm{``reco correction''} 
\end{align*}
The distribution and corresponding fits for the two weights is shown in Figure~\ref{fig:jmcfits} for the "gen correction" (left) and "reco correction" (right).
\begin{figure}[htbp]
\centering
\includegraphics[width=0.45\textwidth]{figures/analysis/search2/AN-16-235/plots/JMC_fit_gen.pdf}
\includegraphics[width=0.45\textwidth]{figures/analysis/search2/AN-16-235/plots/JMC_fit_reco.pdf}
\caption{Fit to $\rm{M_{PDG}/M_{GEN}}$ as a function of jet $\pt$ (left), where $\rm{M_{PDG}}=80.4~\GeV$ and $\rm{M_{GEN}}$ is the fitted mean of the generator level mass and a fit to $\rm{(M_{RECO}-M_{GEN})/M_{RECO}}$ (right), where $\rm{M_{RECO}}$ is the reconstructed softdrop mass, as a function of jet $\pt$ in two $\eta$ bins.}
\label{fig:jmcfits}
\end{figure}
The two corrections are then applied to the uncorrected PUPPI softdrop mass both in data and in MC as
\begin{equation}
M_{SD}=M_{\rm{SD, uncorr}} \times \rm{w_{GEN}} \times \rm{w_{RECO}}
\end{equation}
where $w_{GEN}$ and $w_{RECO}$ correspond to the gen and reco corrections respectively and $M_{\rm{SD, uncorr}}$ is the uncorrected PUPPI softdrop mass. \par
Finally, a closure test is performed in order to check that the corrected PUPPI+softdrop W-jet mass peaks at 80.4 \GeV and is stable with \PT and $\eta$. The fitted mean of the corrected PUPPI softdrop mass peak as a function of jet \PT in two different $\eta$ bins is shown in Figure~\ref{fig:searchII:wtagclosure}. Good closure is observed, with the corrected mass peaking around 80 GeV independent of the jet $\pt$ and $\eta$.The PUPPI softdrop jet mass peak for W/Z-jets from different signal samples after jet mass corrections have been applied is shown in Figure~\ref{fig:search2:corrMass}, for resonances with a mass of 1 and 4 TeV. The corrections applied to Z-jets yield a mass stable with \PT, peaking around the Z mass. 
\begin{figure}[htbp]
\centering
\includegraphics[width=0.49\textwidth]{figures/analysis/search2/AN-16-235/plots/ClosureTest_RecoMass.pdf}
\caption{The mean of the fitted W-jet corrected PUPPI softdrop mass peak as a function of jet $\pt$ in two different $\eta$ bins.}
\label{fig:searchII:wtagclosure}
\end{figure}
\begin{figure}[htbp]
\centering
\includegraphics[width=0.49\textwidth]{figures/analysis/search2/AN-16-235/plots/SoftdropMass_NEWCORR_wH0.pdf}
\caption{The W/Z/H-jet corrected PUPPI softdrop mass peak for jets from different signal samples with masses of 1 and 4 TeV.}
\label{fig:search2:corrMass}
\end{figure}

\subsubsection{W-tagging performance}
The new PUPPI+softdrop based \PW/\PZ-tagger uses a mass window of $65 \GeV < m_{SD} < 105 \GeV$ in combination with a cut of PUPPI $\tau_{21}<0.4$.
We compare its performance to that of the CHS+pruning based tagger used in Search I as well as to that of a "DDT-transformed" \nsubj based tagger~\cite{Dolen:2016kst}. The $\tau_{21}^{DDT}$ variable is a linear transformation of \nsubj given as
\begin{equation}
\label{eq:searchII:ddt}
\tau_{21}^{DDT} = \tau_{21} + M \times \log \bigg( \frac{m^2}{p_T \times 1 \textrm{ GeV}}\bigg)
\end{equation}
where $M=-0.063$ is obtained from a fit of $\tau_{21}$ against the variable $\rho^{'}=\log(m^2/\PT/\mu)$, where $\mu = 1 \GeV$.
The purpose of this is to de-correlate $\tau_{21}$ from the softdrop mass and \PT, yielding a mass and dijet invariant mass spectrum minimally sculpted by
a cut on the $\tau_{21}^{DDT}$ tagging variable. This is tagger that will be further explored and explained in detail in the context of Search III, Section~\ref{sec:searchIII:ddt}.\par
The background rejection efficiency for QCD light flavor jets as a function of W-jet signal efficiency is shown in Figure~\ref{fig:searchII:roc}
The efficiency is measured requiring a fixed jet mass window of 65-105 \GeV, while scanning the cut on $\tau_2/\tau_1$.
\begin{figure}[ht!]
\centering
\includegraphics[width=0.49\textwidth]{figures/vtagging/JME-16-003/BoostedW/roc_WqqvsQCD_2bins.pdf}
\caption{The background rejection efficiency for QCD light flavor jets as a function of W-jet signal efficiency. A cut on CHS pruned or PUPPI softdrop jet
mass of $65<m_{\mathrm{jet}}<105$~\GeV is applied while scanning the cut on \nsubj. The cuts corresponding to $\tau_2/\tau_1 < 0.45$ for CHS+pruning, PUPPI $\tau_2/\tau_1 < 0.4$ for PUPPI+softdrop or $\tau_{21}^\text{DDT}<0.52$ are indicated with triangles, while the solid circles represent the efficiency and mistag rate for a mass cut only.}
\label{fig:searchII:roc}
\end{figure}
The general performance of each tagger is very similar, with the PUPPI+softdrop based taggers displaying a slightly higher signal efficiency for a given mistag rate at high \PT and CHS+pruning slightly better at low \PT.
Two better understand the difference between each tagger, we look at the tagging performance as a function of jet \PT as well as pileup, shown in Figure~\ref{fig:searchII:effvspt} and~\ref{fig:searchII:effvspu}.\par
Starting with the tagger \PT-dependence in Figure~\ref{fig:searchII:effvspt}, we observe that she signal efficiency of a PUPPI+softdrop of CHS+pruned jet mass cut is flat as a function of \PT, at around 80\%. The QCD mistagging rate drops for both groomers, with a 1-3\% lower mistag rate using PUPPI+softdrop that CHS+pruning.
\begin{figure}[h!]
\centering
\includegraphics[width=0.49\textwidth]{figures/vtagging/JME-16-003/BoostedW/WtagSigEffvsPT.pdf}
\includegraphics[width=0.49\textwidth]{figures/vtagging/JME-16-003/BoostedW/QCDBkgEffvsPT.pdf}
\caption{W-jet efficiency (left) and QCD light jet mistag rate (right) for a PUPPI+softdrop or CHS+pruned jet mass selection only (hollow circles) and the combined $m_{\mathrm{jet}}$ + (PUPPI) $\tau_2/\tau_1$ (DDT) selection (solid circles) as a function of jet \PT.}
\label{fig:searchII:effvspt}
\end{figure}
Once applying an n-subjettiness cut, the signal efficiency as well as the mistag rate for the PUPPI \nsubj and CHS \nsubj taggers drops as a function of \PT, with an average signal efficiency of around 50\% for a $\sim 2\%$ mistag rate. An interesting behavior is observed for the $\tau_{21}^{DDT}$ tagger: While the mistag rate is flat as a function of \PT, as is the purpose of decorrelated taggers, the signal efficiency improves as the \PT increases, outperforming the other taggers above 1 \TeV. \par
Turning to the tagger pileup dependence, shown in Figure~\ref{fig:searchII:effvspu}, the expected benefit from using the PUPPI algorithm is observed: The tagging efficiency for the CHS+pruning (red solid cirles) based tagger falls of steeply versus the number of primary vertices in the event, while the PUPPI+softdrop based taggers (light and dark blue solid circles) are more or less insensitive to pileup.
\begin{figure}[h!]
\centering
\includegraphics[width=0.49\textwidth]{figures/vtagging/JME-16-003/BoostedW/WtagSigEffvsNPV.pdf}
\includegraphics[width=0.49\textwidth]{figures/vtagging/JME-16-003/BoostedW/QCDBkgEffvsNPV.pdf}
\caption{W-jet efficiency (left) and QCD light jet mistag rate (right) for a PUPPI+softdrop or CHS+pruned jet mass selection only (hollow circles) and the combined $m_{\mathrm{jet}}$ + (PUPPI) $\tau_2/\tau_1$ (DDT) selection (solid circles) as a function of jet pileup.}
\label{fig:searchII:effvspu}
\end{figure}
Based on general performance, tagging stability versus pileup and due to theoretical considerations, PUPPI sofdrop mass with dedicated mass corrections applied together with PUPPI \nsubj is chosen as this analysis W-tagger. The per-jet efficiency is around 50-55\% for a 1-2\% mistag rate.\par

\subsubsection{Efficiency scale factors and mass scale/resolution measurement} 
\label{sec:searchII:wtagsf}
subsubsection{Efficiency scale factors and mass scale/resolution measurement} 
\label{sec:searchII:wtagsf}
In order to measure the W-tagging efficiency, jet mass scale and resolution for the new PUPPI+softdrop based tagger, we use the same procedure as outlined in Section~\ref{sec:searchI:vtag}. We first did an early measurement of the efficiency using 2.3 \fbinv of data collected in 2015, which was published in a jet algorithms performance note~\cite{CMS-PAS-JME-16-003} and served as the first commissioning of the new tagger. We then redid the measurement with 12.9 and 35.9 \fbinv of 2016 data, respectively, for the two analyses presented in this chapter (the latter measurement performed by a separate analysis team). The results shown in the following will be those obtained during the commissioning of the tagger, while those used in the two analyses are listed in Appendix~\ref{app:sf16}.
In order to better understand the differences between the CHS+pruning and PUPPI+softdrop based taggers, the first efficiency measurement was done in parallel for both algorithms, requiring either a softdrop or a pruned jet mass between 40 \GeV and 150 \GeV. The softdrop mass is computed after PUPPI and the jet mass corrections as described in Section~\ref{sec:searchII:masscorr} are applied, while the pruned mass is corrected with L2L3 jet energy corrections. 
The method is the same as the one outlined in detail in Section~\ref{sec:searchI:vtag} and fits to matched \ttbar MC and minor backgrounds for the PUPPI softdrop based tagger are skipped here and can be found in Appendix~\ref{app:sf16}.\newline
The PUPPI softdrop jet mass and PUPPI $\tau_{21}$ variables in data and in MC are shown in Figure~\ref{fig:searchII:ttbarcp} and can be compared to the corresponding plots for the CHS pruned jet mass and CHS \nsubj distributions in Figure~\ref{fig:searchII:ttbarcp}. The data to MC agreement as well as the observed spectra, is very similar between CHS pruning and PUPPI softdrop in this region.

\begin{figure}[ht!]
\centering
\begin{tabular}{cc}
\includegraphics[width=0.5\textwidth]{figures/vtagging/AN-16-215/Whadr_puppi_softdrop_mu.pdf}
\includegraphics[width=0.5\textwidth]{figures/vtagging/AN-16-215/Whadr_puppi_tau2tau1_mu.pdf}\\
\end{tabular}
\caption{Distribution of the PUPPI softdrop mass (left) and PUPPI n-subjettiness (right) distribution in the \ttbar control sample.}
\label{fig:searchII:ttbarcp}
\end{figure}

Following what was done in Section~\ref{sec:searchI:vtag}, we extract and compare the W-tagging efficiency, jet mass scale and resolution of the combined jet mass and \nsubj selection in data and in MC. This is done through a simultaneous fit of the the softdrop jet mass spectrum between 40 and 150 \GeV in two regions:
\begin{itemize}
\itemsep0em
  \item Pass region: $0 <  \nsubj \leq 0.40$ $\sim$ high purity
  \item Fail region: $0.40 < \nsubj \leq 0.75$ $\sim$ low purity
\end{itemize}
The corresponding fits are shown in Figure~\ref{fig:searchII:simfit}, with the corresponding extracted efficiencies from the Gaussian component of the total fit and scale factors summarized in Table~\ref{tab:searchII:WtagSFs}.
The quoted systematic uncertainties are evaluated the same was as described in Section~\ref{sec:searchI:wtagsystematic} and correspond to the uncertainty due to ME generator and due to choice of fit method.
\begin{table}[h!]
   \centering
   \footnotesize
   \begin{tabular}{|l|l|c|c|c|}
   \hline
   Category & Working point & Eff. data & Eff. simulation & Scale factor\\
   \hline
   HP& $\tau_2 / \tau_1 < 0.4$         & $0.785 \pm 0.045 $& $0.81 \pm 0.01$   &$0.97 \pm 0.06~\rm{(stat)} \pm 0.04~\rm{(sys)} \pm 0.06~\rm{(sys)}$\\
   LP& $0.45 < \tau_2 / \tau_1 < 0.75$ & $0.215 \pm 0.057 $& $0.204 \pm 0.041$ &$1.13 \pm 0.24~\rm{(stat)} \pm 0.17~\rm{(sys)}  \pm 0.12~\rm{(sys)}$\\
   \hline
   \end{tabular}
   \caption{W-tagging scale factors for both categories the high purity and low purity categories for two taggers: Pruned jet mass + \nsubj and PUPPI softdrop jet mass + PUPPI \nsubj. }
   \label{tab:searchII:WtagSFs}
\end{table}


\begin{figure}[htbp]
\centering
% \includegraphics[width=0.44\textwidth]{figures/analysis/search2/AN-16-235/plots/TotalFit__HP0v40powheg_PuppiSD.pdf}      %12.9 fbinv
% \includegraphics[width=0.44\textwidth]{figures/analysis/search2/AN-16-235/plots/TotalFit__HP0v40powheg_PuppiSD_fail.pdf} %12.9 fbinv
\includegraphics[width=0.44\textwidth]{figures/vtagging/AN-16-215/_HP0v40powheg_76X_PuppiSD_em_pTbin_200_5000.pdf}
\includegraphics[width=0.44\textwidth]{figures/vtagging/AN-16-215/_HP0v40powheg_76X_PuppiSD_em_fail_pTbin_200_5000.pdf}\\
\caption{PUPPI softdrop jet mass distribution that pass (left) and fail (right) the PUPPI $\tau_2 / \tau_1 < 0.40$ selection. Results of both the fit to data (blue) and simulation (red) are shown and the background components of the fit are shown as short-dashed lines. (!RATHER REPLACE BY 12.9INVFB MEASUREMENT)}
\label{fig:searchII:simfit}
\end{figure}

Both scale factors are comparable to unity, within uncertainties. We additionally extract the jet mass scale and jet mass resolution from the mean and width, respectively, of the Gaussian
component of the total fit in the pass region. These are summarized in Table~\ref{tab:searchII:wtagparams}.  As for pruning (Table~\ref{tab:searchI:params}), we find that the W jet mass scale is larger in simulation than in data, of roughly 2\%.
The jet mass resolution, on the other hand, is larger in data, of roughly 7\%, whereas for pruning the resolution is larger in simulation (11\%). However, both are statistically insignificant and comparable with unity within uncertainties.

\begin{table}[!htb]
 \begin{center}

 \begin{tabular}{c|c|c|c}
  Parameter & Data & Simulation & Data/Simulation \\
  \hline
  PUPPI softdrop $\langle m \rangle$ & $80.3 \pm 0.8~{\rm \GeV}$ & $81.9 \pm 0.01~{\rm \GeV}$ & $0.98 \pm 0.01$ \\%New mass corrections
  PUPPI softdrop $\sigma$            & $ 9.0 \pm 0.9~{\rm \GeV}$ &  $8.5 \pm 0.4~{\rm \GeV}$  & $1.07 \pm 0.12$ \\%New mass corrections
  \hline
 \end{tabular}
 \caption{Summary of the fitted W-mass peak fit parameters.}
 \label{tab:searchII:wtagparams}
 \end{center}
\end{table}

The W-tagging efficiency scale factors, jet mass scale and resolution affects the signal yield and are included as described in Section~\ref{sec:searchI:wtagimpact}: as a scale of the total signal yield and an uncertainty on the signal efficiency due to
a shift and broadening of the \PW-jet mass peak.
%  % \begin{figure}[htbp]
% %  \centering
% %  \begin{tabular}{cc}
% %  \includegraphics[width=0.45\textwidth]{figures/vtagging/AN-16-215/plots_76X/plots_0v45/model_data_em.pdf}
% %  \includegraphics[width=0.45\textwidth]{figures/vtagging/AN-16-215/plots_76X/plots_0v45/model_TotalMC_em.pdf}\\
% %  \includegraphics[width=0.45\textwidth]{figures/vtagging/AN-16-215/plots_76X/plots_0v45/model_data_failtau2tau1cut_em.pdf}
% %  \includegraphics[width=0.45\textwidth]{figures/vtagging/AN-16-215/plots_76X/plots_0v45/model_TotalMC_failtau2tau1cut_em.pdf}\\
% %  \end{tabular}
% %  \caption{Fit results in data (left) and simulation (right) after simultaneous fit.
% % Top: pass selection($\tau_{21} < 0.45$), bottom: fail selection($\tau_{21} > 0.45$). }
% %  \label{fig:Wtagging_sf_045}
% %  \end{figure}
% %
% %   \begin{figure}[htbp]
% %   \centering
% %   \begin{tabular}{cc}
% %   \includegraphics[width=0.45\textwidth]{figures/vtagging/AN-16-215/model_data_em.pdf}
% %   \includegraphics[width=0.45\textwidth]{figures/vtagging/AN-16-215/model_TotalMC_em.pdf}\\
% %   \includegraphics[width=0.45\textwidth]{figures/vtagging/AN-16-215/model_data_failtau2tau1cut_em.pdf}
% %   \includegraphics[width=0.45\textwidth]{figures/vtagging/AN-16-215/model_TotalMC_failtau2tau1cut_em.pdf}\\
% %   \end{tabular}
% %   \caption{Fit results in data (left) and simulation (right) after simultaneous fit.
% %  Top: pass selection(PUPPI $\tau_{21} < 0.40$), bottom: fail selection(PUPPI $\tau_{21} > 0.40$). }
% %   \label{fig:Wtagging_sf_040}
% %   \end{figure}
%

\subsection{W-tagging mistagging rate measurement} 
\label{sec:searchII:wmistag}
We additionally measure the W-tagging fake rate in data in a QCD dijet enriched region and compare this to the prediction from QCD MC using the three different combination of generators: \HERWIG{++}, \PYTHIA and \MADGRAPH+\PYTHIA.
Figure~\ref{fig:searchII:fakerate} shows the mistag rate as a function of \PT for three different taggers: CHS pruning + \nsubj, PUPPI softdrop + PUPPI \nsubj and PUPPI softdrop + \ddt.
\begin{figure}[htbp]
\centering
\includegraphics[width=0.49\textwidth]{figures/vtagging/JME-16-003/BoostedW/BkgEff_DataMC_herwig_pT.pdf}
\includegraphics[width=0.49\textwidth]{figures/vtagging/JME-16-003/BoostedW/BkgEff_DataMC_Pythia8_pT.pdf}\\
\includegraphics[width=0.49\textwidth]{figures/vtagging/JME-16-003/BoostedW/BkgEff_DataMC_Pythia8Madgraph_pT.pdf}
\caption{ 
The fraction of jets that pass the $m_{\mathrm{jet}}$ + $\tau_2/\tau_1$ selections in a dijet enriched sample for data and for simulation as a function of jet \PT. Here comparing \HERWIG{++} (left), \PYTHIA{8} (right) and \PYTHIA{8} with \MADGRAPH as matrix-element generator (left).}
\label{fig:searchII:fakerate}
\end{figure}
We find a substantial difference in the modeling of substructure variables between the different generators, most likely coming from their very different description of gluon radiation (dominant in QCD multijet events).
The best description is obtained with \HERWIG{++}, while all three generators model the tagging \PT dependence well.
We additionally study the difference in the total quark/gluon-content for the two PUPPI softdrop based taggers: \nsubj and \ddt. Figure~\ref{fig:searchII:qgfakerate} shows the stacked relative q/g content in a Pythia 8 QCD dijet sample for a cut on PUPPI \nsubj and \ddt. We see that the quark content increases as a function of jet \PT when cutting on the DDT, while it decreases when cutting on \nsubj.
\begin{figure}[htbp]
\centering
\includegraphics[width=0.49\textwidth]{figures/vtagging/JME-16-003/BoostedW/qgFakeRate_Pythia8_pT.pdf}
\caption{ 
The fraction of jets that pass the PUPPI softdrop $m_{\mathrm{jet}}$ with $\tau_2/\tau_1$ (turquoise) or \ddt (orange) selections in a dijet enriched sample. The jets from QCD MC are split into two contributions: jets originating from gluons (dotted line) and jets originating from quarks (solid line).}
\label{fig:searchII:qgfakerate}
\end{figure}
 This can be attributed to the fact that the $m/\PT$ distribution for quark and gluon jets are very different from one another, and this difference increase as the jet \PT increases. Figure~\ref{fig:searchII:mpt_qvsg} shows the $m/\PT$ for jets originating from a quark (blue) and a gluon (red) for a jet \PT of 200 \GeV (left) and 1600 \GeV. We see that the mass over \PT for gluon jets is significantly higher for gluon jets than for quark jets.
 With the \ddt tagger being defined as in Equation~\ref{eq:searchII:ddt}, the DDT will therefore act more aggressive on jets with a high $m/\PT$, effectively removing more gluon jets.
\begin{figure}[ht!]
\centering
\includegraphics[width=0.49\textwidth]{figures/analysis/search2/misc/ak07_MDPt_0200.pdf}
\includegraphics[width=0.49\textwidth]{figures/analysis/search2/misc/ak07_MDPt_1600.pdf}
\caption{The jet mass divided by the jet \PT for quark(blue) and gluon (red) jets for a jet \PT of 200 (left) and 1600 \GeV (right). Created with~\cite{Gallicchio:2011xq}.}
\label{fig:searchII:mpt_qvsg}
\end{figure}
\clearpage

\subsection{Mass and purity categorization}
The PUPPI softdrop jet mass and PUPPI \nsubj distribution after loose analysis preselections, as outlined in Section~\ref{sec:searchII:samples}, are shown in Figure~\ref{fig:searchII:wtag}.
We see some disagreement between data and MC, especially in the high-purity region (PUPPI $\ddt<0.4$), confirming what we observed in Section~\ref{sec:searchII:wmistag}
\begin{figure}[h!]
\centering
\includegraphics[width=0.49\textwidth]{figures/analysis/search2/AN-16-235/plots/qcdcp_PuppiSoftdropMass.pdf}
\includegraphics[width=0.49\textwidth]{figures/analysis/search2/AN-16-235/plots/qcdcp_puppi_tau2tau1.pdf}
\caption{PUPPI softdrop jet mass distribution (left) and PUPPI n-subjettiness $\tau_{21}$ (right) distribution for data and simulated samples. Simulated samples are scaled to match the distribution in data.}
\label{fig:searchII:wtag}
\end{figure}
As this analysis is sensitive to both heavy resonances decaying into two vector bosons and excited quark resonances $q^*$ decaying to qW and qZ, we look for events with both a single W/Z-tag and events with two W/Z-tags. 
Vector boson candidates are selected with a PUPPI softdrop jet mass of $65<m_{\mathrm{jet}}<105$~\GeV. Further, and similar to what was done in Search I, we select "high purity" (HP) W/Z jets by requiring  PUPPI $0<\tau_{21} \leq 0.40$ and "low purity" (LP) jets with $0.40<\tau_{21}\leq0.75$. The events with one W/Z-tag are classified in HP and LP events according to the two categories described previously. Events with two W/Z-tagged jets are always required to have one HP tagged jet, and are further divided into LP and HP categories depending on whether the other jet is of high or low purity. We additionally split into two mass categories in order to enhance the analysis sensitivity, with the W window defined as $65 {\GeV} < m_{pruned} < 85 {\GeV}$ and the Z boson window as $85 {\GeV} < m_{pruned} < 105 {\GeV}$. This results in ten different signal categories. They are as follows:
\begin{itemize}
\item High-purity double W/Z-tag, 3 mass categories: WW, ZZ and WZ
\item Low-purity double W/Z-tag, 3 mass categories: WW, ZZ and WZ
\item High-purity single W/Z-tag, 2 mass categories: qW and qZ
\item Low-purity single W/Z-tag, 2 mass categories: qW and qZ
\end{itemize}

\subsection{Background modeling: F-test} 
\label{sec:searchII:ftest}
With the full analysis selections and categorization defined, we move to the determination of background fit function. Following the same strategy as in Section~\ref{sec:searchI:bkg}, we determine the number of necessary parameters in order to describe the background through a Fishers F-test, comparing the same fit functions as in Section~\ref{sec:searchI:bkg}. This test is first exercised in QCD MC and then in a data sideband before the final determination in the data signal region. As the F-test method was presented in detail in the context of Search I, only a brief summary and the fits in the new single-tag categories will be presented here, while all fits and F-test results can be found in Appendix~\ref{app:2016xcheck}.\newline
A two or three parameter fit is sufficient to describe the background for all the double tag categories: a two parameter fit is sufficient for the "high-purity" WZ and ZZ categories. 
as well as the "low-purity" WW category, while the remaining analysis categories require a three parameter background fit.
From the fits to the single tag categories, shown in Figure~\ref{fig:searchII:bkgfit_sr_qv}, a three parameter fit is sufficient for all categories except the "high-purity" qW category. In this category the improvement in fit quality when increasing the number
of parameters is so large adding that adding an additional fit parameter is justified, and we continue by using a 5 parameter fit for this category. 
\begin{figure}[h!]
\centering
\includegraphics[width=0.45\textwidth]{figures/analysis/search2/AN-16-235/plots/qWHP.pdf}
\includegraphics[width=0.45\textwidth]{figures/analysis/search2/AN-16-235/plots/qWLP.pdf}\\
\includegraphics[width=0.45\textwidth]{figures/analysis/search2/AN-16-235/plots/qZHP.pdf}
\includegraphics[width=0.45\textwidth]{figures/analysis/search2/AN-16-235/plots/qZLP.pdf}\\
\caption{Background fit for the $M_{jj}$ distribution in the data signal region for the single-tag analysis. Here for the high- (left) and low-purity (right) single W/Z-tag categories qW (top) and qZ (bottom).}
\label{fig:searchII:bkgfit_sr_qv}
\end{figure}
A summary of what fit functions are used for each analysis category is listen in Table~\ref{tab:searchII:fitpars}.
\begin{table}[htb]
\centering
\begin{tabular}{|l| c | c|}
\hline
Mass category & \multicolumn{2}{|c|}{N pars.}\\
\hline
& HP & LP \\
\hline
WW & 3 & 2 \\
WZ & 2 & 3 \\
ZZ & 2 & 3 \\
qW & 5 & 3 \\
qZ & 3 & 3 \\
\hline
\end{tabular}
\caption{Fit parameters used in each analysis category}
\label{tab:searchII:fitpars}
\end{table}

\subsection{Signal modeling}
The signal is modeled from signal MC in the same way as was done in Section~\ref{sec:searchI:sig}, assuming a Gaussian core and an exponential tail. The interpolated signal shapes for $\rm{q}^* \rightarrow \rm{q}\PW$ and $\rm{q}^* \rightarrow \rm{q}\PZ$ in their most sensitive analysis categories ($\rm{q}\PW$ and $\rm{q}\PZ$,respectively) are shown in Figure \ref{fig:searchII:interpolation}. The signal shapes for the double-tag category can be compared to those in Figure~\ref{fig:searchI:sigfit}.
 
\begin{figure}[htbp]
\centering
\includegraphics[width=0.49\textwidth]{figures/analysis/search2/AN-16-235/plots/interpolation_QstarQZ_DijetMassHighPuriqZ.pdf}
\includegraphics[width=0.49\textwidth]{figures/analysis/search2/AN-16-235/plots/interpolation_QstarQW_DijetMassHighPuriqW.pdf}\\
\caption{Interpolated signal shapes for a  $q*\rightarrow qZ$ (left) and $q*\rightarrow qW$ (right) signal.}
\label{fig:searchII:interpolation}
\end{figure}


\subsection{Systematic uncertainties}
The largest sources of systematic uncertainty for this analysis is, as for Search I, related to the signal modeling and are due to the uncertainty in the tagging efficiency of the W/Z-tagger, the jet energy/mass scale, the jet energy/mass resolution and integrated luminosity. The W/Z tagging uncertainty is estimated in $\rm{t\bar{t}}$ events, as described in Section~\ref{sec:searchI:vtag}, and yield uncertainties on the scale factors for the HP and LP tagging categories.
The \pt- and $\eta$-dependent jet energy scale and resolution uncertainties on the resonance shape were approximated by a constant 2\% and 10\% uncertainty in Search I (Section~\ref{sec:searchI:wtagsystematic}) and are not expected to change for the 2016 analysis. The jet energy response and resolution uncertainty are taken into account as shape uncertainty by shifting and widening the signal resonance model,
while all other signal uncertainties only affect the yield. The list of most relevant systematic uncertainties are listed in Table~\ref{tab:searchII:VV_systematicssummary_signal}.
\begin{table}[htb]
  \centering
  \begin{tabular}{lccc}
    \hline
    Source                        & Relevant quantity      & HP+HP unc. (\%)  & HP+LP unc. (\%)\\
    \hline
    Jet energy scale                 & Resonance shape        & 2                    & 2 \\
    Jet energy resolution            & Resonance shape        & 10                   & 10 \\
    \hline
    Jet energy scale                 & Signal yield           & \multicolumn{2}{c}{$<$0.1--4.4}\\ 
    Jet energy resolution            & Signal yield           & \multicolumn{2}{c}{$<$0.1--1.1}\\
    Jet mass scale                   & Signal yield           & \multicolumn{2}{c}{0.02--1.5}\\ 
    Jet mass resolution              & Signal yield           & \multicolumn{2}{c}{1.3--6.8}\\ 
    Pileup                           & Signal yield           & \multicolumn{2}{c}{2}\\
    Integrated luminosity            & Signal yield           & \multicolumn{2}{c}{6.2}\\
    PDFs (\PWpr)                     & Signal yield		       & \multicolumn{2}{c}{4--19}\\
    PDFs (\PZpr)                     & Signal yield		       & \multicolumn{2}{c}{4--13}\\
    PDFs (\BulkG)                    & Signal yield		       & \multicolumn{2}{c}{9--77}\\
    Scales (\PWpr)                   & Signal yield		       & \multicolumn{2}{c}{1--14}\\
    Scales (\PZpr)                   & Signal yield		       & \multicolumn{2}{c}{1--13}\\
    Scales (\BulkG)                  & Signal yield		       & \multicolumn{2}{c}{8--22}\\
    \hline
    % Jet energy scale                 & Migration              & \multicolumn{2}{c}{0--14}\\
%     Jet energy resolution            & Migration              & \multicolumn{2}{c}{0--4.1}\\
    Jet mass scale                   & Migration              & \multicolumn{2}{c}{$<$0.1--16.8}\\
    Jet mass resolution              & Migration              & \multicolumn{2}{c}{$<$0.1--17.8}\\
    W-tagging \nsubj{}               & Migration              & 15.6                  & 21.9\\
    W-tagging \pt-dependence         & Migration              & 7--14                 & 5--11\\
    \hline
  \end{tabular}
  \caption{Summary of the signal systematic uncertainties for the analysis and their impact on the event yield in the signal region and on the reconstructed dijet invariant mass shape (mean and width).}
  \label{tab:searchII:VV_systematicssummary_signal}
\end{table}

\subsection{Results}   
As mentioned in the introduction to this chapter, the analysis of the 2016 dataset was done in two steps: One based on 12.9 \fbinv of early 2016 data, demonstrating the new PUPPI softdrop based tagger and single-tag analysis categories, and one topping up with the full 35.9 \fbinv dataset. The results from both will be presented in the following.

\subsubsection{Early analysis}   
\label{sec:searchII:b2g16021res}
Exclusion limits are set in the context of the bulk graviton model, the HVT model~B scenario and excited quark resonances,
assuming the resonances to have a natural width negligible with respect to the experimental resolution (as in Search I).

Figure~\ref{fig:searchII:limitCombined} shows the 95\% confidence level (CL) expected and observed exclusion limits on the signal cross section as a function of the resonance mass for the different signal hypotheses in the double-tag analysis.
The limits are compared with the cross section times the branching fraction to $\PW\PW$ and $\PZ\PZ$ for a bulk graviton with $\ktilde = 0.5$, and with the cross section
times the branching fraction to $\PW\PZ$ and $\PW\PW$ for spin-1 particles predicted by the HVT model~B for both the singlet (\PWpr or \PZpr) and triplet (\PWpr and \PZpr) hypothesis.
For the HVT model B, we exclude \PWpr(\PZpr) resonances with masses below 2.7 (2.6)~\TeV. The signal cross section uncertainties are displayed as a red checked band and result in an additional uncertainty on the resonance mass limits of 0.05 (0.04)~\TeV.
The cross section limits for $\PZpr \rightarrow \PW\PW$ and $\rm G_{bulk}\rightarrow\PW\PW$ are not identical due to the different acceptance for those two signal scenarios.

\begin{figure}[htbp]
\centering
     \includegraphics[width=0.45\textwidth]{figures/analysis/search2/B2G-16-021/figures/limits/brazilianFlag_ZprimeWW_new_combined_13TeV.pdf}
     \includegraphics[width=0.45\textwidth]{figures/analysis/search2/B2G-16-021/figures/limits/brazilianFlag_WZ_new_combined_13TeV.pdf}\\
     \includegraphics[width=0.45\textwidth]{figures/analysis/search2/B2G-16-021/figures/limits/brazilianFlag_BulkWW_new_combined_13TeV.pdf}
     \includegraphics[width=0.45\textwidth]{figures/analysis/search2/B2G-16-021/figures/limits/brazilianFlag_BulkZZ_new_combined_13TeV.pdf}
\caption{Observed (black solid) and expected (black dashed) 95\% CL upper limits on the production of a narrow-width resonance decaying to a pair of vector bosons for different signal hypotheses. 
Limits are set in the context of a spin-1 neutral \PZpr (left) and charged \PWpr (right) resonances resonance, and compared with the prediction of the HVT model~B. On the bottom, limits are set in the context of a bulk graviton decaying into $\PW\PW$ (left) and $\PZ\PZ$ (right) with $\ktilde =0.5$ and compared with the model prediction. Signal cross section uncertainties are displayed as a red checked band.
}
\label{fig:searchII:limitCombined}
\end{figure}

Figure~\ref{fig:limitCombined_qV} shows the corresponding exclusion limits for excited quarks decaying into q\PW and q\PZ.
We exclude excited quark resonances decaying into q\PW and q\PZ with masses below 5.0 and 3.9~\TeV, respectively.
The signal cross section uncertainties are displayed as a red checked band and result in an additional uncertainty on the resonance mass limits of 0.1~\TeV.

\begin{figure}[htbp]
\centering
     \includegraphics[width=0.45\textwidth]{figures/analysis/search2/B2G-16-021/figures/limits/brazilianFlag_qW_new_combined_13TeV.pdf}
     \includegraphics[width=0.45\textwidth]{figures/analysis/search2/B2G-16-021/figures/limits/brazilianFlag_qZ_new_combined_13TeV.pdf}\\
\caption{Observed (black solid) and expected (black dashed) 95\% CL upper limits on the production of an excited quark resonance
decaying into qW (left) or qZ (right). Signal cross section uncertainties are displayed as a red checked band.
}
\label{fig:searchII:limitCombined_qV}
\end{figure}



\subsubsection{Full 2016 dataset}   
\label{sec:searchII:brg17001res}

The results obtained with the full $\sim 36$ \fbinv of 2016 data are as follows:
For a \BulkG we exclude production cross sections in a range from 36.0~fb, at a resonance mass of 1.3\TeV, to 0.6~fb at resonance masses above 3.6\TeV. \PWpr(\PZpr) resonances are excluded with masses below 3.2 (2.7)\TeV for the HVT model B, in addition to \PWpr resonances with a mass between 3.3 and 3.6 \TeV. For excited quark resonances, we can exclude the production of $\rm{q}^*$ decaying to qW or qZ for masses below 5.0 and 4.7 TeV.
Figure~\ref{fig:searchII:limitCombined_VV} and~\ref{fig:searchII:limitCombined_qV} show the resulting 95\% confidence level expected and observed exclusion limits on the signal cross section as a function of the resonance mass for VV and QV resonances, respectively.

\begin{figure}[h!]
\centering
    \includegraphics[width=0.49\textwidth]{figures/analysis/search2/B2G-17-001/figures/brazilianFlag_BulkWW_VVnew_new_combined_13TeV.pdf}
    \includegraphics[width=0.49\textwidth]{figures/analysis/search2/B2G-17-001/figures/brazilianFlag_WZ_VVnew_new_combined_13TeV.pdf}\\
    \includegraphics[width=0.49\textwidth]{figures/analysis/search2/B2G-17-001/figures/brazilianFlag_BulkZZ_VVnew_new_combined_13TeV.pdf}
\caption{Observed (solid line) and expected (dashed line) 95\% CL upper limits on the production cross section of a narrow resonance decaying into two vector bosons for different signal hypotheses: A \PZpr or \BulkG resonance decaying into \WW (top left), a \PZpr decaying into \PW\PZ (top right) and a bulk graviton decaying into \ZZ (bottom).}
\label{fig:searchII:limitCombined_VV}
\end{figure}

\begin{figure}[ht!]
\centering
    \includegraphics[width=0.49\textwidth]{figures/analysis/search2/B2G-17-001/figures/brazilianFlag_qW_qVnew_new_combined_13TeV.pdf}
    \includegraphics[width=0.49\textwidth]{figures/analysis/search2/B2G-17-001/figures/brazilianFlag_qZ_qVnew_new_combined_13TeV.pdf}\\
\caption{Observed (solid line) and expected (dashed line) 95\% CL upper limits on the production of an excited quark resonance
decaying into qW (left) or qZ (right).}
\label{fig:searchII:limitCombined_qV}
\end{figure}



\clearpage

\begin{singlespace}
\setstretch{1.25}
\vspace*{\fill}
\begin{centering}
\chapter{Search III: A novel framework for multi-dimensional searches}
\label{searchIII}
\textit{
After two successful analyses of 13 TeV data, no excess had confirmed the 8 \TeV bump and the available phase space for New Physics to hide out was shrinking. However, this fact was not a source of concern to everybody. On the BSM theory front, ideas were simmering about whether it was possible that the small bumps we were observing in the dijet invariant mass spectrum were due to us catching the tail of another type of boson with a mass slightly different from that of a \PW or a \PZ boson. And that perhaps these jets were not 2-prong objects, but in reality 4-prong.
\newline
\newline
With Run 2 coming to and end, marking the beginning of a two year long shut-down, it was time to think about how we could probe alternative BSM models as effectively as possible. Our idea was therefore the following: We would build a novel framework capable of easily scanning the full softdrop jet mass and N-prong spectrum, and which, in addition, would lead to a gain in sensitivity for the standard VV all-hadronic search. We would do this by taking advantage of the fact that we were looking for bumps in a three-dimensional space: the softdrop mass of the two jets as well as their invariant mass. As a validation of the method, the method would be demonstrated in the context of the VV all-hadronic search, replacing the dijet fit method. We would then extend this to simultaneously search for resonances decaying to W(qq), Z(qq) and H(qq) and, finally, take full advantage of the framework and look for generic resonances peaking anywhere in the jet mass and dijet invariant mass spectrum.
\newline
\newline
Search III introduces a novel three-dimensional search method, allowing to simultaneously search for W/Z/H signals, and eventually non-SM bosons, in the softdrop jet mass spectrum. It is the first analysis to measure the jet mass scale and resolution simultaneously from a W(qq)+jets and Z(qq)+jets mass peak. Published with the full 2016+2017 dataset, $\sim 80 \fbinv$.}
\end{centering}
\begin{figure}[b!]
    \centering
    \includegraphics[height=6.5cm]{figures/analysis/search3/B2G-18-002/PostFitComboHPLP_Y-Proj__x___0_-1_z___0_-1.pdf}
    \vspace*{10mm}
    \caption*{\footnotesize{\textit{In progress. To be submitted to The European Physical Journal C}}}
\end{figure}
\end{singlespace}

\clearpage
\subsection{Small bumps and tribosons}
\subsection{Analysis strategy}
\subsection{Event selection}
\subsection{A mass and \PT decorrelated tagger}
\subsection{The multidimensional fit}




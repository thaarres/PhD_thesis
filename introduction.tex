\noindent The Standard Model of particle physics (SM) is one of the greatest accomplishments of fundamental science. The degree to which it can accurately predict observed phenomena is unprecedented, and it has allowed us to incorporate all of particle physics into one single equation that explains what we can see in the world around us. However, it has some shortcomings. One major problem is its failure to successfully incorporate gravity, leaving large scale phenomena unexplained. It also fails to explain why gravity is so much weaker than the electromagnetic and nuclear forces. This, together with a few other problems, has lead scientists to search for extensions to the Standard Model, commonly referred to as \emph{Beyond Standard Model physics} (BSM). These models are usually accompanied by predicted observables not included in the Standard Model, where the observation of these, or the lack thereof, is a way of falsifying or supporting the model.
\newline
\newline
In this thesis, I look for such observables by searching for new massive particles predicted by SM extensions. These particles have the property that they can decay into vector bosons, $\PW^{\pm}$ and $\PZ^{0}$, and usually have a very small interaction probability. The vector bosons are heavy and unstable and will quickly decay into leptons or quarks. In order to to counterbalance the small interaction probability associated with generating such a new heavy particle, a final state with two vector bosons decaying hadronically is required since the branching ratio for a vector boson decaying to hadrons is significantly higher than that into leptons. This final state is complicated by the presence of an overwhelming QCD multijet background and the fact that, due to the high mass of the resonance, the vector bosons are highly energetic and their quark decay products get merged into a single jet due to the small angular opening between them. The latter offers an opportunity to distinguish between vector boson jets and jets coming from a quark or a gluon due to the expected differences in jets mass and geometrical substructure within the jet. Algorithms designed to improve the jet mass resolution and resolve jet substructure are commonly referred to as \emph{jet substructure methods}. These will be a recurring topic of this thesis due to my own personal contributions to the field.
\newline
\newline
Three searches for heavy resonances decaying to dibosons in the all-hadronic final state will be presented. The first analysis to be discussed was the first of its kind to be performed at a center-of-mass energy of $\sqrt{\rm{s}}=13 \TeV$ and the first published result to take advantage of jet substructure at trigger level. The second led to the development of a novel pileup robust and perturbative safe vector boson tagging algorithm, which afterwards became the default tagging algorithm in CMS. Finally, the third search introduces a completely new way of doing diboson searches in a multi-dimensional space, allowing for the incorporation of all VV, VH and HH searches (where V = W,Z and H = Higgs boson) into one common framework as well as any generic search for resonances peaking in jet mass and dijet invariant mass. I have performed every aspect of the three analyses, making original contributions to all.
\newline
\newline
In addition to the tree searches, I will present a deep neural network for vector boson tagging intended to improve the analysis sensitivity for future searches. In this algorithm, jet clustering and substructure-like variables are embedded into the neural network layers themselves. That makes it a good starting point in the development of a generic anti-QCD tagger capable of distinguishing between the QCD background and any signal with some geometrical substructure peaking in the jet groomed mass spectrum. Such a tagger would, in combination with the multidimensional fit framework, lead to a completely new way of doing model independent searches.
\newline
\newline
This thesis is organized in five parts. Part~\ref{ch:theory} provides the theoretical background and motivation for the searches presented here. First, the Standard Model is introduced together with a discussion of its known shortcomings, followed by a chapter presenting two possible extensions to the Standard Model, both of which are probed in this thesis. Part~\ref{ch:CMS} consists of a description of the experimental setup used to collect the data which is analyzed here, as well as the different algorithms used in order to reconstruct each event. Part~\ref{ch:theory} and Part~\ref{ch:CMS} mainly consist of work done by others which has been vital for the completion of this work. The remaining three parts are dedicated to my own personal contributions. In Part~\ref{ch:diboson}, the three searches described above are presented in chronological order, each with a personal introduction motivating the analysis in question. Following this, in Part~\ref{ch:lola}, the deep neural network based vector boson tagger for future analyses will be presented. Both parts end with their own concluding summary and outlook, in Section~\ref{sec:searchIII:outlook} and Section~\ref{sec:lola:outlook}. In Part~\ref{ch:summary}, a final summary of the obtained results and a discussion of the future for the analysis is given.
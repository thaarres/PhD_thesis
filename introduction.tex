\noindent The Standard Model of particle physics (SM) is one of the greatest accomplishments of fundamental science. The degree to which it can accurately predict observed phenomena is unprecedented, and it has allowed us to incorporate all of particle physics into one single equation that explains what we can see in the world around us. Almost all. Its greatest shortcoming is its failure to successfully incorporate gravity, leaving large scale phenomena unexplained. This, together with a few other shortcomings, has lead scientists to search for extensions to the Standard Model, commonly referred to as \emph{Beyond Standard Model physics (BSM)}. These models are usually accompanied by predicted observables not included in the Standard Model, where the observation of these, or the lack thereof, is a way of falsifying or supporting the model. In this thesis, I attempt to do exactly that by searching for new massive particles predicted by SM extensions. These particles have the property that they can decay into vector bosons, $\PW^{\pm}$ and $\PZ^{0}$, and usually have a very small interaction probability. The vector bosons are heavy and unstable and will quickly decay into leptons or quarks, and in order to make up for the small interaction probability, I look for two vector bosons decaying hadronically, which occurs $\sim70\%$ of the time, significantly more frequently than leptonic decays. This final state is complicated by the presence of an overwhelming QCD multijet background, and the fact that, due to the high mass of the resonance, the vector bosons are highly energetic and their quark decay products are so collimated they get merged into a single jet. Fortunately, the latter offers an opportunity to separate vector boson jets from quark and gluon jets through the jets mass and how many prongs the jet appears to have. Methods to do this separation are referred to as \emph{jet substructure methods} and is something I will heavily focus on in this thesis, due to my own personal contributions to the field.
\newline
\newline
Three analyses will be described, where each has benefitted from novel findings in the previous: The first search was the first analysis of its kind to ever be performed at a center-of-mass energy of $\sqrt{\rm{s}}=13 \TeV$ and the first published result to take advantage of jet substructure at trigger level. The second led to the development of a novel pileup robust and perturbative safe vector boson tagging algorithm, which afterwards became the default tagging algorithm in CMS. Finally, the third search introduces a completely new way of doing diboson searches in a multi-dimensional space, allowing for the incorporation of all VV, VH and HH searches (where V = W,Z and H = Higgs boson) into one common framework as well as any generic search for resonances peaking in jet mass and dijet invariant mass. I have performed every aspect of the three analyses, making original contributions to all, despite the latter where we were a small team of three analysts dividing the workload.
\newline
\newline
In the final chapter of this thesis, I will introduce an ongoing work on a deep neural network for vector boson tagging intended to improve the analysis sensitivity for future searches and which in addition could be used to develop a generic anti-QCD tagger due to its deep encoding of jet substructure. Such a tagger would, in combination with the multidimensional fit framework, lead to a completely new way of doing model independent searches.
\newline
\newline
This thesis is organized in the following way. In Chapter~\ref{ch:theory}, I will go through the theoretical motivations behind the searches presented here through introducing the Standard Model, its known shortcomings and possible alternatives. This is followed by a description of the experimental setup used to collect the analyzed data, in Chapter~\ref{ch:CMS}, as well as the different algorithms used in order to reconstruct each event in Chapter~\ref{ch:objreco}. The remaining part of the thesis is dedicated to my own personal contributions: The three searches are presented chronologically in Chapter~\ref{ch:diboson}, each with a personal introduction motivating the search in question. Following this, and ending my thesis work, is a description of the deep neural network based vector boson tagger for future analyses, in Chapter~\ref{ch:lola}. Both chapters end with their own concluding summary and outlook, Section~{sec:searchIII:outlook} and Section~{sec:lola:outlook}.  To bring everything together, Chapter~\ref{ch:summary}, the final chapter of this thesis, provides a final conclusion of the work presented.
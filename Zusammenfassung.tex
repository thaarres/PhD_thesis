\small
\noindent In dieser Doktorarbeit werde ich drei verschiedene Analysen über neue schwere Resonanzen vorstellen, die in Paare von Vektorbosonen zerfallen, welche wiederum in Quarks zerfallen. Die analysierten Daten wurden vom CMS-Experiment am LHC während der ersten drei Jahre der Datennahme bei einer Schwerpunktsenergie von 13 \TeV gesammelt, was einer integrierten Luminosität von 2.7 (2015), 35.9 (2016) und 77.3 (2016 + 2017) \fbinv entspricht. Diese Analysen sind die ersten ihrer Art, die jemals bei einer so hohen Kollisionsenergie durchgeführt wurden. Die Endzustände des Vektorboson-Paare sind schwierig voneinander zu unterscheide, da die Bosonen hochenergetisch ("boosted") sind, was dazu führt, dass die beiden Quarks aus dem Zerfall kollimiert werden und zu einem einzigen Teilchenbündel ("jet") verschmelzen. Dies führt zu einer Dijet-Endzustands-Topologie, bei der jeder Jet eine Energieunterstruktur aufweist. Die erste Analyse, die ich vorstellen werde, war eine der ersten veröffentlichten CMS-Analysen in “boosted” Endzuständen mit 13 \TeV-Daten, und die erste, welche die Jet-Substruktur auf Trigger-Ebene nutzte. Es war eine hoch gehandelte Analyse aufgrund eines zuvor beobachteten 3.4 (1.3) $\sigma$-Überschusses um 2 \TeV im 8 \TeV-Datensatz, der von ATLAS (CMS) analysiert wurde. Ich habe die Analyse innerhalb von sechs Monaten nachdem die 13 \TeV-Datenaufnahme begonnen hatte zu einem veröffentlichten Ergebnis gebracht. Im Anschluss daran habe ich in meiner zweiten Analyse einen neuartigen Algorithmus ("PUPPI Softdrop") für das Vektorboson-Tagging optimiert, validiert und in Betrieb genommen und zusätzlich spezielle Massenkorrekturen für die Softdrop Jet-Masse entwickelt. Der Algorithmus und die entsprechenden Massenkorrekturen sind jetzt der Standard für das Vektorboson-Tagging in CMS und werden von mehreren Analysen verwendet. Es war das erste veröffentlichte Ergebnis, bei dem PUPPI-Softdrop genutzt wurde. Die dritte und letzte Analyse führt ein neuartiges multidimensionales Analyse ein, mit dem nach Resonanzen irgendwo im 3D-Spektrum der Dijet- und Jet-Massenspektren gesucht werden kann. Dieses Analyse-Architektur kann für alle Resonanzanalysen über hadronisch zerfallende Vektorbosonen oder hadronich zerfallende Higgs-Boson-Endzustände sowie generische Analysen über geboostete Objekt, die in der Jet-Masse einen Höchstwert (Peak) haben, verwendet werden. 
% Im Rahmen dieser Analyse wurde gleichzeitig eine Anpassung der W- und Z-Jet-Massenpeaks des Standardmodells $\PV(\bar{\rm{q}}\rm{q})$+jets Prozesses durchgeführt. Dies ermöglichte die erste Extraktion der Jet-Massenskala sowie deren Auflösung (und schließlich den $\PV(\bar{\rm{q}}\rm{q})$+jets Querschnitt) des V+Jets Prozesses.
Schließlich werde ich ein Deep Neural Network für das Vektorboson-Tagging vorstellen, welches Jet-Clustering- und substrukturähnliche Variablen in die neuronalen Netzwerkschichten encodiert. Dieser Algorithmus verbessert die Analyseempfindlichkeit erheblich und kann auch als Grundlage für die Entwicklung eines generischen Anti-QCD-Taggers verwendet werden. Letzteres ist von großer Bedeutung, wenn versucht wird, dieses mehrdimensionale Framework für modellunabhängige Analysen zu verwenden.